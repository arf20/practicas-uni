\documentclass{amsart}
\usepackage{sagetex}
\usepackage[utf8]{inputenc}
\newtheorem{ejer}{Ejercicio}

\def\n{\mathbb{N}}
\def\r{\mathbb{R}}
\def\z{\mathbb{Z}}
\def\q{\mathbb{Q}}
\def\c{\mathbb{C}}

\begin{document}
\begin{center}
\textrm{ \bf {AMD Curso 2023-2024}}
\vskip 0.3cm
\textrm{ \bf {Prácticas Semana 2} }
\end{center}

\begin{ejer} Estudia y resuelve si es posible el siguiente sistema de ecuaciones sobre $\r $ :
\[ x_{1} - x_{2} - 5 x_{3} - 5 x_{4} = 6 \]
\[ -x_{1} - x_{2} + 2 x_{3} + x_{4} = 1 \]
\[ -x_{2} - 2 x_{3} - 3 x_{4} = 5 \]
\[ x_{4} = -2 \]
\end{ejer}

{\it Soluci\'on.-}
% Inicio del ejercicio 1
\begin{sageblock}
A = matrix(QQ,[[1,-1,-5,-5],[-1,-1,2,1],[0,-1,-2,-3],[0,0,0,1]])
B = matrix(QQ,[[6],[1],[5],[-2]])
Ap = A.augment(B, subdivide=True)
R = Ap.echelon_form()
S = R[:, 4:]
\end{sageblock}

$$
	A = \sage{A}
$$
$$
	B = \sage{B}
$$
$$
	A' = [A|B] = \sage{Ap}
$$
$$
	R = \sage{R} = [I|S]
$$
La matriz $A$ reducida es la identidad por tanto el Sistema es Compatible Determinado, por tanto la solución S es
$$
	S = \sage{S}
$$
es decir
\begin{align*}
	x_1 &= \sage{S[0, 0]} \\
	x_2 &= \sage{S[1, 0]} \\
	x_3 &= \sage{S[2, 0]} \\ 
	x_4 &= \sage{S[3, 0]}
\end{align*}

% fin del ejercicio 1

\begin{ejer} Estudia y resuelve si es posible el siguiente sistema de ecuaciones sobre $\r $ :
\[ 2 x_{1} - x_{2} = 8 \]
\[ x_{1} - x_{3} = -3 \]
\[ x_{2} = -5 \]
\[ x_{1} = 2 \]
\[ x_{3} = 8 \]
\end{ejer}

{\it Soluci\'on.-}
% Inicio del ejercicio 2
\begin{sageblock}
A = matrix(QQ,[[2,-1,0],[1,0,-1],[0,1,0],[1,0,0],[0,0,1]])
B = matrix(QQ,[[8],[-3],[-5],[2],[8]])
Ap = A.augment(B, subdivide=True)
R = Ap.echelon_form()
\end{sageblock}

$$
	A' = [A|B] = \sage{Ap}
$$
$$
	R = \sage{R}
$$
En la cuarta fila de la matriz aumentada reducida, $0=1$ es un absurdo, por tanto esto es un Sistema Incompatible.


% Fin del ejercicio 2

\begin{ejer} Estudia y resuelve si es posible el siguiente sistema de ecuaciones sobre $\z _7$ :
\[ 3x_{1} + x_{2} + 3x_{3} + 5x_{4} + 2x_{5} = 3 \]
\[ 3x_{3} + 3x_{4} + x_{5} = 1 \]
\[ 4x_{5} = 12 \]

\end{ejer}

{\it Soluci\'on.-}

% Inicio del ejercicio 3

\begin{sageblock}
A=matrix(Zmod(7),[[3,1,3,5,2],[0,0,3,3,1],[0,0,0,0,4]])
B=matrix(Zmod(7),[3,1,12]).T
Ap = A.augment(B, subdivide=True)
R = Ap.echelon_form()
\end{sageblock}

$$
	A' = [A|B] = \sage{Ap}
$$
$$
	R = \sage{R}
$$
La matriz aumentada reducida no tiene identidad, y no contiene absurdos, por tanto debe ser un Sistema Compatible Indeterminado:

\begin{align*}
	x_1 + 5 x_2 + 3 x_4 &= 2 \\
	x_3 + x_4 &= 4 \\
	x_5 &= 3 \\ 
\end{align*}

Que parametrizando quedan dos parametros libres
\begin{align*}
	x_1 &= 2 - 3 \alpha - 5 \beta \\
	x_2 &= \beta \\
	x_3 &= 4 - \alpha \\
	x_4 &= \alpha \\
	x_5 &= 3
\end{align*}

% Fin del ejercicio 3

\begin{ejer} Estudia y resuelve si es posible el siguiente sistema de ecuaciones sobre $\r $:
\[ -x_{2} + 3 x_{3} = 3 \]
\[ x_{1} - 2 x_{2} + 5 x_{3} = 3 \]
\[ x_{1} - 3 x_{2} + 9 x_{3} = 7 \]
\[ x_{1} - 2 x_{2} + 3 x_{3} = 1 \]
\[ 2 x_{1} - 3 x_{2} + 9 x_{3} = 5 \]
\end{ejer}

{\it Soluci\'on.-}
% Inicio del ejercicio 4

\begin{sageblock}
A = matrix(QQ,[[0,-1,3],[1,-2,5],[1,-3,9],[1,-2,3],[2,-3,9]])
B = matrix(QQ,[[3],[3],[7],[1],[5]])
Ap = A.augment(B, subdivide=True)
R = Ap.echelon_form()
\end{sageblock}

$$
	A' = [A|B] = \sage{Ap}
$$
$$
	R = \sage{R}
$$
La matriz de A reducida queda como identidad, y las filas extra no son absurdos, por tanto esto es un SCD, cuya solución es
\begin{align*}
	x_1 &= -2 \\
	x_2 &= 0 \\
	x_3 &= 1 \\
\end{align*}

% Fin del ejercicio 4

\begin{ejer} Estudia y resuelve si es posible el siguiente sistema de ecuaciones sobre $\r $:
\[ x_{1} - x_{2} + x_{3} + 2 x_{4} = 2 \]
\[ x_{2} - 2 x_{3} - 5 x_{4} = -1 \]
\[ x_{2} - x_{3} - 2 x_{4} = 0 \]
\end{ejer}

{\it Soluci\'on.-}
% Inicio del ejercicio 5

\begin{sageblock}
A = matrix(QQ,[[1,-1,1,2],[0,1,-2,-5],[0,1,-1,-2]])
B = matrix(QQ,[[2],[-1],[0]])
Ap = A.augment(B, subdivide=True)
R = Ap.echelon_form()
\end{sageblock}

$$
A' = [A|B] = \sage{Ap}
$$
$$
R = \sage{R}
$$
Es SCI:
\begin{align*}
	x_1 &= 2 \\
	x_2 + x_4   &= 1 \\
	x_3 + 3 x_4 &= 1 \\
\end{align*}

\begin{align*}
	x_1 &= 2 \\
	x_2 &= 1 - \alpha \\
	x_3 &= 1 - 3 \alpha \\
	x_4 &= \alpha
\end{align*}

% Fin del ejercicio 5


\begin{ejer} Estudia y resuelve si es posible el siguiente sistema de ecuaciones sobre $\z _5$:
\[ 0 = 0 \]
\[ x_{1} + 3 x_{2} + 2 x_{3} = 0 \]
\[ -x_{1} + x_{2} + x_{3} = 1 \]
\[ -x_{1} + 3 x_{2} = 3 \]
\[ -x_{1} + 3 x_{2} = 0 \]
\end{ejer}

{\it Soluci\'on.- }
% Inicio del ejercicio 6


\begin{sageblock}
A = matrix(Zmod(5),[[0,0,0],[1,3,2],[4,1,1],[4,3,0],[4,3,0]])
B = matrix(Zmod(5),[[0],[0],[1],[3],[0]])
Ap = A.augment(B, subdivide=True)
R = Ap.echelon_form()
\end{sageblock}

$$
A' = [A|B] = \sage{Ap}
$$
$$
R = \sage{R}
$$

$0=0$ es un absurdo por tanto es SI.
% Fin del ejercicio 6

\begin{ejer} Estudia y resuelve si es posible el siguiente sistema de ecuaciones sobre $\z _5$:
\[ x_{1} - x_{2} + 3 x_{4} = 0 \]
\[ x_{2} - x_{4} = 3 \]
\[ 2 x_{1} + 2 x_{2} + x_{3} + 2 x_{4} = 4 \]
\[ 2 x_{1} + 2 x_{2} + 3 x_{3} + 3 x_{4} = 3 \]
\[ 2 x_{1} + 2 x_{2} + 3 x_{3} = 3 \]
\end{ejer}

{\it Soluci\'on.- }
% Inicio del ejercicio 7

\begin{sageblock}
A = matrix(Zmod(5),[[1,4,0,3],[0,1,0,4],[2,2,1,2],[2,2,3,3],[2,2,3,0]])
B = matrix(Zmod(5),[[0],[3],[4],[3],[3]])
Ap = A.augment(B, subdivide=True)
R = Ap.echelon_form()
\end{sageblock}

$$
A' = [A|B] = \sage{Ap}
$$
$$
R = \sage{R}
$$
La matriz reducida resulta la identidad sin absurdos y por tanto es SCD:
\begin{align*}
	x_1 &= 3 \\
	x_2 &= 3 \\
	x_3 &= 2 \\
	x_4 &= 0
\end{align*}

% Fin del ejercicio 7


\begin{ejer} Estudia y resuelve si es posible el siguiente sistema de ecuaciones sobre $\z _{11}$:
\[ 2x_{1} - x_{2} + x_{3} + x_{4} = 1 \]
\[ 2x_{1} + x_{} - x_{4} = 2 \]
\[ x_{2} + 2x_{3} + 3x_{4} = 0 \]
\end{ejer}
{\it Soluci\'on.- }
% Inicio del ejercicio 8

\begin{sageblock}
A=matrix(Zmod(11),[[2,-1,1,1],[2,1,0,-1],[0,1,2,3]])
B=matrix(Zmod(11),[1,2,0]).T
Ap = A.augment(B, subdivide=True)
R = Ap.echelon_form()
\end{sageblock}

$$
A' = [A|B] = \sage{Ap}
$$
$$
R = \sage{R}
$$

Matriz reducida es parte identidad sin absurdos, SCI, parametrización:
\begin{align*}
	x_1 &= 3 - 4 \alpha \\
	x_2 &= 7 - 2 \alpha \\
	x_3 &= 2 - 6 \alpha \\
	x_4 &= \alpha
\end{align*}

% Fin del ejercicio 8



\begin{ejer} Estudia y resuelve si es posible el siguiente sistema de ecuaciones sobre $\z _7$:
\[ x_{1} + x_{3} - x_{4} = 2 \]
\[ -x_{1} + x_{2} - x_{3} + x_{4} = 2 \]
\[ x_{2} + x_{3} - x_{4} = 2 \]
\[ x_{1} - x_{3} - x_{4} = 2 \]
\end{ejer}

{\it Soluci\'on.- }
% Inicio del ejercicio 9

\begin{sageblock}
A = matrix(Zmod(7),[[1,0,1,2],[2,1,2,1],[0,1,1,2],[1,0,2,2]])
B = matrix(Zmod(7),[[2],[2],[2],[2]])
Ap = A.augment(B, subdivide=True)
R = Ap.echelon_form()
\end{sageblock}

$$
A' = [A|B] = \sage{Ap}
$$
$$
R = \sage{R}
$$
Matriz reducida es identidad, SCD:
\begin{align*}
	x_1 &= 6 \\
	x_2 &= 6 \\
	x_3 &= 0 \\
	x_4 &= 5
\end{align*}

% Fin del ejercicio 9

\begin{ejer} Estudia y resuelve si es posible el siguiente sistema de ecuaciones sobre $\r $:
\[ -x_{1} - 5x_{3} + x_{6} = -1 \]
\[ 3x_{1} + x_{2} - x_{3} + x_{5} = 0 \]
\[ -2x_{1} + x_{4} = 3 \]

\end{ejer}

{\it Soluci\'on.- }
% Inicio del ejercicio 10

\begin{sageblock}
A=matrix(QQ,[[-1,0,-5,0,0,1],[3,1,1,0,1,0],[-2,0,0,1,0,0]])
B=matrix(QQ,[-1,0,3]).T
Ap = A.augment(B, subdivide=True)
R = Ap.echelon_form()
\end{sageblock}

$$
A' = [A|B] = \sage{Ap}
$$
$$
R = \sage{R}
$$
La matriz reducida es en parte identidad por tanto SCI, parametrización:

\begin{align*}
	x_1 &= -\frac{3}{2} + \frac{1}{2} \gamma \\
	x_2 &= 4 - \frac{7}{5} \gamma - \beta - \frac{1}{5} \alpha \\
	x_3 &= \frac{1}{2} - \frac{1}{10} \gamma + \frac{1}{5} \alpha \\
	x_4 &= \gamma \\
	x_5 &= \beta \\
	x_6 &= \alpha
\end{align*}


% Fin del ejercicio 10


\begin{ejer} Encuentra cuando sea posible la inversa de la matriz

\[ \left( \begin{array}{ccc}
2 & 3 & 1 \\
1 & 4 & 3 \\
2 & 5 & 1 \end{array} \right) \ \hbox{ en} \  \z _{2}. \] 

\end{ejer}

{\it Soluci\'on.- }
% Inicio del ejercicio 11

\begin{sageblock}
A=matrix(Zmod(2),[[2,3,1],[1,4,3],[2,5,1]])
AI = A.augment(matrix.identity(3), subdivide=True)
R = AI.echelon_form()
\end{sageblock}

$$
[A|I] = \sage{AI}
$$
$$
R = \sage{R}
$$
La matriz reducida de $[A|I]$ no queda como identidad por tanto no es invertible (regular)

% Fin del ejercicio 11

\begin{ejer} Encuentra cuando sea posible la inversa de la matriz

\[ \left( \begin{array}{cc}
1 & 2 \\
3 & 4 \end{array} \right) \ \hbox{en} \ \z _{7}. \]

\end{ejer}

{\it Soluci\'on.- }
% Inicio del ejercicio 12

\begin{sageblock}
A=matrix(Zmod(7),[[1,2],[3,4]])
AI = A.augment(matrix.identity(2), subdivide=True)
R = AI.echelon_form()
\end{sageblock}

$$
[A|I] = \sage{AI}
$$
$$
R = \sage{R}
$$
La matriz reducida de $[A|I]$ queda $[I|A^{-1}]$, así que $A^{-1}$ es
$$
A^{-1} = \sage{R[:, 2:]}
$$

% Fin del ejercicio 12

\begin{ejer} Encuentra cuando sea posible la inversa de la matriz

\[ \left( \begin{array}{cccc}
1 & 2 & 3 \\
4 & 6 & 0 \\
3 & 0 & 0  \end{array} \right) \ \hbox{en} \ \z _{5}. \]

\end{ejer}
{\bf Solución.-}
% Inicio del ejercicio 13

\begin{sageblock}
A=matrix(Zmod(5),[[1,2,3],[4,6,0],[3,0,0]])
AI = A.augment(matrix.identity(3), subdivide=True)
R = AI.echelon_form()
\end{sageblock}

$$
[A|I] = \sage{AI}
$$
$$
R = \sage{R}
$$
La matriz reducida de $[A|I]$ queda $[I|A^{-1}]$, así que $A^{-1}$ es
$$
A^{-1} = \sage{R[:, 3:]}
$$

% Fin del ejercicio 13














\end{document}
























