\documentclass{amsart}
\usepackage{sagetex}
\usepackage[utf8]{inputenc}
\newtheorem{ejer}{Ejercicio}

\def\n{\mathbb{N}}
\def\r{\mathbb{R}}
\def\z{\mathbb{Z}}
\def\q{\mathbb{Q}}
\def\c{\mathbb{C}}

\begin{document}
\begin{center}
\textrm{ \bf {AMD Curso 2023-2024}}
\vskip 0.3cm
\textrm{ \bf {Tarea Semana 3} }
\end{center}

\begin{sagesilent}
set_random_seed(23836) # Cambia este número por las 5 primeras cifras de tu DNI o NIE

n1 = ZZ.random_element(2^30)
x1 = ZZ.random_element(n1)
while(gcd(x1,n1)!=1):
  x1 = ZZ.random_element(n1)
a2 = ZZ.random_element(2^15)
b2 = ZZ.random_element(2^15)
m2 = gcd(a2,b2)*ZZ.random_element(2^10)
p3 = ZZ.random_element(2^20).next_prime()
m31 = p3*ZZ.random_element(2^15)
m32 = p3*ZZ.random_element(2^15)
g3 = gcd(m31,m32)
n3 = ZZ.random_element(m31*m32)
a3 = Zmod(m31)(n3)
b3 = Zmod(m32)(n3)
\end{sagesilent}


\vskip 1cm

\begin{ejer}  Calcula, usando la reducción de matrices, el inverso de $\sage{x1}$ en $\z_{\sage{n1}}$ .

\end{ejer}
{\it Solución.-}

% Escribe tu soluci\'on para el Ejercicio 1

\begin{sageblock}
	A = matrix(ZZ, [[x1], [n1]])
	Ap = block_matrix([[A, 1]])
	Ar = Ap.echelon_form()
	Ar = copy(Ar)
	Ar.subdivide([], 1)
	S = Ar.subdivision(0, 1)
	R = S * A
\end{sageblock}

$$
	A' = [A|I] = \sage{Ap}
$$

$$
	\sage{Ar}
$$

El mcd$(\sage{x1}, \sage{n1}) = 1$, por tanto es invertible.

$$
	\sage{S} \sage{A} = \sage{R}
$$
La expresión del mcd de $\sage{x1}$ y $\sage{n1}$
$$
	\sage{S[0][0]} \cdot \sage{A[0][0]} + (\sage{S[0][1]}) \cdot \sage{A[1][0]} = \sage{R[0][0]}
$$

$$
	\sage{S[0][0]} \cdot \sage{A[0][0]} = \sage{R[0][0]}
$$  

$$
	\sage{A[0][0]}^{-1} = \sage{S[0][0]} \text{ en } (\z_{\sage{n1}})
$$

% Fin del Ejercicio 1

\vskip 1cm

\begin{ejer} Dada la ecuación $\sage{a2}x + \sage{b2}y = \sage{m2}$, indica si tiene solución y (caso de tener) calcula todas las posibles soluciones.

\end{ejer}
{\it Solución.-}

% Escribe tu soluci\'on para el Ejercicio 2


\begin{sageblock}
	A = matrix(ZZ, [[a2], [b2]])
	Ap = block_matrix([[A, 1]])
	Ar = Ap.echelon_form()
	R = Ar[:, 1:] * A
\end{sageblock}

$$
	A' = [A|I] = \sage{Ap}
$$
reducida
$$
	\sage{Ar}
$$

mcd$(\sage{a2}, \sage{b2}) = 1$ es divisor de $145$, por tanto hay solución.

$$
	\sage{Ar[:, 1:]} \sage{A} = \sage{R}
$$

Desarrollar

\begin{align*}
	\sage{Ar[0][1]} \cdot \sage{A[0][0]} + (\sage{Ar[0][2]}) \cdot \sage{A[1][0]} &= \sage{R[0][0]} \\
	\sage{Ar[1][1]} \cdot \sage{A[0][0]} + (\sage{Ar[1][2]}) \cdot \sage{A[1][0]} &= \sage{R[1][0]} \\
\end{align*}

Multiplicar primera relación por $m$ para tener $m$, y segunda relación por $t$

\begin{align*}
	\sage{Ar[0][1]} \cdot \sage{A[0][0]} \cdot \sage{m2} + (\sage{Ar[0][2]}) \cdot \sage{A[1][0]} \cdot \sage{m2} &= \sage{m2} \\
	\sage{Ar[1][1]} \cdot \sage{A[0][0]} \cdot t + (\sage{Ar[1][2]}) \cdot \sage{A[1][0]} \cdot t &= \sage{R[1][0]} \\
\end{align*}

Sumar las dos

$$
	\sage{A[0][0]} (\sage{Ar[0][1] * m2} + \sage{Ar[1][1]} t) + \sage{A[1][0]} (\sage{Ar[0][2] * m2}  \sage{Ar[1][2]}t) = \sage{m2}
$$

Extraer $x$ e $y$

\begin{align*}
	x &= \sage{Ar[0][1] * m2} + \sage{Ar[1][1]} t \\
	y &= \sage{Ar[0][2] * m2}  \sage{Ar[1][2]}t \\
\end{align*}

% Fin del Ejercicio 2

\vskip 1cm

\begin{ejer} Obtener todas las soluciones del siguiente sistema de congruencias:
\begin{align*}
n &\equiv \sage{a3} \pmod{\sage{m31}} \\
n &\equiv \sage{b3} \pmod{\sage{m32}} \\
\end{align*}
\end{ejer}
{\it Solución.-} 

% Escribe tu soluci\'on para el Ejercicio 3

\begin{sageblock}
	A = matrix(ZZ, [[m31], [-m32]])
	Ap = block_matrix([[A, 1]])
	Ar = Ap.echelon_form()
	Ar = copy(Ar)
	Ar.subdivide([], 1)
	As = Ar[:, 1:]
	R = Ar[:, :1]
	
	mcd = R[0, 0]
	
	a3n = 1792944400
	b3n = 16054429802
	d = b3n - a3n
	
	F = d / mcd
	
	c3 = a3n + (m31 * As[0, 0] * F)
	cm3 = m31 * As[1, 0]
\end{sageblock}

$\text{mcd}(\sage{m31}, \sage{m32})$


$$
	[A|I] = \sage{Ap}
$$
reducida
$$
	\sage{Ar}
$$

$\text{mcd}(\sage{m31}, \sage{m32}) = \sage{Ar[0][0]} \not= 1$ es un caso 2

\begin{align*}
	n &= \sage{a3} + \sage{m31} x \\
	n &= \sage{b3} + \sage{m32} y \\
\end{align*}

$$
	\sage{a3} + \sage{m31} x = \sage{b3} + \sage{m32} y
$$

$$
	\sage{m31} x - \sage{m32} y = \sage{d} = d
$$

$$
	F = \frac{\sage{d}}{\sage{mcd}} = \sage{F}
$$
El mcd es divisor de $d$, por tanto hay solución

$$
	\sage{As} \cdot \sage{A} = \sage{R}
$$

\begin{align*}
	\sage{As[0, 0]} \cdot \sage{A[0, 0]} + \sage{As[0, 1]} \cdot (\sage{A[1, 0]}) &= \sage{R[0, 0]} \\
	\sage{As[1, 0]} \cdot \sage{A[0, 0]} + \sage{As[1, 1]} \cdot (\sage{A[1, 0]}) &= \sage{R[1, 0]} \\
\end{align*}

Multiplicar la de arriba para dar $d$, y abajo por $t$

\begin{align*}
	\sage{As[0, 0]} \cdot \sage{A[0, 0]} \cdot \sage{F} + \sage{As[0, 1]} \cdot (\sage{A[1, 0]}) \cdot \sage{F} &= \sage{R[0, 0] * F} \\
	\sage{As[1, 0]} \cdot \sage{A[0, 0]} \cdot t + \sage{As[1, 1]} \cdot (\sage{A[1, 0]}) \cdot t &= \sage{R[1, 0]} \\
\end{align*}

Al sumarlas

$$
	\sage{A[0, 0]} (\sage{As[0, 0] * F} + \sage{As[1, 0]} t)  \sage{A[1, 0]} (\sage{As[0, 1] * F} + \sage{As[1, 1]} t) = \sage{d}
$$

\begin{align*}
	x &= \sage{As[0, 0] * F} + \sage{As[1, 0]} t \\
	y &= \sage{As[0, 1] * F} + \sage{As[1, 1]} t \\
\end{align*}

\begin{align*}
	n &= \sage{a3} + \sage{m31} x = \sage{a3} + \sage{m31} (\sage{As[0, 0] * F} + \sage{As[1, 0]} t) \\
	&= \sage{c3} + \sage{cm3} t \\
	&= \sage{c3} \text{ (mod \sage{cm3})}
\end{align*}

% Fin del Ejercicio 3

\end{document}
