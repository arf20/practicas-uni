\documentclass{amsart}
\usepackage[utf8]{inputenc}
\usepackage[usefamily=sage]{pythontex} 
\usepackage{float}
\usepackage{tikz}
\usetikzlibrary{calc}
\usepackage[most]{tcolorbox}
\usepackage[margin = 2cm]{geometry}
\usepackage{graphicx}
\usetikzlibrary{3d}
\usepackage{tikz-3dplot}


\newtheorem{ejer}{Ejercicio}

\def\n{\mathbb{N}}
\def\r{\mathbb{R}}
\def\z{\mathbb{Z}}
\def\q{\mathbb{Q}}
\def\c{\mathbb{C}}

\title{Tarea 10 AMD. Producto Escalar}

\begin{document}

\maketitle

\begin{sagecode}
latex.matrix_delimiters("[", "]")
\end{sagecode}

\begin{ejer}
Para cada una de las rectas dadas, calcula un vector $v_1$ de la recta y dos vectores 
$v_2$ y $v_3$ perpendiculares a la recta de forma que la base $B = \{v_1,v_2,v_3\}$ tenga 
orientación positiva, normalízalos, píntalos en unos ejes coordenados y dibuja un cuadrado de 
tamaño $4\times 4$ centrado en el origen con lados paralelos a los vectores $v_2$ y $v_3$.
Para hacer los dibujos usa los siguientes ejes coordenados:

\begin{sagesub}
\begin{center}
\tdplotsetmaincoords{70}{110}
\begin{tikzpicture}[scale = 2, tdplot_main_coords,
  cara/.style={thick, color = blue, fill opacity = 0.3, fill = blue!20}]
\draw[->,thick,gray] (-2.5,0,0) -- (2.5,0,0); % Eje X
\draw[->,thick,gray] (0,-2.5,0) -- (0,2.5,0); % Eje Y
\draw[->,thick,gray] (0,0,-2.5) -- (0,0,2.5); % Eje Z
\end{tikzpicture}
\end{center}
\end{sagesub}


\begin{enumerate}
\item La recta $r$ dada por $\begin{cases} 2x+2y-z = 0 \\ x-y+3z = 0 \end{cases} $
\item La recta $r$ dada por $x = 2\lambda, y = 0, z = -\lambda$ con $\lambda \in {\mathbb R}$.
\end{enumerate}
\end{ejer}

{\it Solución}

% Inicio Ejercicio 1

\begin{sageblock}
n1 = vector(RR, [2, 2, -1])
n2 = vector(RR, [1, -1, 3])
vr = n1.cross_product(n2)

v1 = vr.normalized()
v2 = vector(RR, [vr[1], -vr[0], 0]).normalized()
v3 = v1.cross_product(v2).normalized()
B = block_matrix([[v1.column(), v2.column(), v3.column()]])

square = matrix(RR, [[0, 2, 2], [0, 2, -2], [0, -2, -2], [0, -2, 2]]).T
squareB = [B * square.column(i) for i in range(4)]
\end{sageblock}

Recta definida por dos planos cuyas normales son $n_1 = (2, 2, -1)$ $n_2 = (1, -1, 3)$
El vector de la recta $v_r$ es perpendicular a las normales, $v_r = n_1 \times n_2 = \sage{vr}$


\pagebreak

\begin{sagesub}
	\begin{center}
		\tdplotsetmaincoords{70}{110}
		\begin{tikzpicture}[scale = 2, tdplot_main_coords,
			cara/.style={thick, color = blue, fill opacity = 0.3, fill = blue!20}]
			\draw[->,thick,gray] (-2.5,0,0) -- (2.5,0,0); % Eje X
			\draw[->,thick,gray] (0,-2.5,0) -- (0,2.5,0); % Eje Y
			\draw[->,thick,gray] (0,0,-2.5) -- (0,0,2.5); % Eje Z
			
			\draw[->,thick,red] (0,0,0) -- !{v1};
			\draw[->,thick,black] (0,0,0) -- !{v2};
			\draw[->,thick,black] (0,0,0) -- !{v3};
			
			%\draw[->,thick,blue] !{squareB[0]} -- !{squareB[1]};
			%\draw[->,thick,blue] !{squareB[1]} -- !{squareB[2]};
			%\draw[->,thick,blue] !{squareB[2]} -- !{squareB[3]};
			%\draw[->,thick,blue] !{squareB[3]} -- !{squareB[0]};
			\filldraw[cara] !{squareB[0]} -- !{squareB[1]} -- !{squareB[2]} -- !{squareB[3]} -- !{squareB[0]};
		\end{tikzpicture}
	\end{center}
\end{sagesub}

% Fin del Ejercicio 1

\begin{ejer}
Dado el espacio $W = N(H) \leq \r^5$, donde

\begin{sageblock}
H=matrix(QQ,[[-1,-1,1,0,2],[2,0,1,1,0]])
\end{sageblock}

\[ H = \sage{H}.\] 

Calcula una base de $W$ y obtén una base ortonormal a partir de ella utilizando el método de Gram-Schmidt.
\end{ejer}

{\it Solución:}

% Inicio Ejercicio 2

\begin{sageblock}
HtI = block_matrix([[H.T, 1]])
HtIr = HtI.echelon_form()
HtIr = copy(HtIr)
HtIr.subdivide(2, 2)
A = HtIr.subdivision(1, 1).T
Ar = A.echelon_form()
v1 = A.column(0)
v2 = A.column(1)
v3 = A.column(2)
w1 = v1
w2 = v2 - (((v2.dot_product(w1))/(w1.dot_product(w1))) * w1)
w3 = v3 - (((v3.dot_product(w1))/(w1.dot_product(w1))) * w1) - (((v3.dot_product(w2))/(w2.dot_product(w2))) * w2)
\end{sageblock}

$$
	[H^T|I] = \sage{HtI} \to \sage{HtIr}
$$

$$
	A = \sage{A} \to \sage{Ar}
$$
A es una base de $W$


$v_1 = \sage{v1}$ $v_2 = \sage{v2}$ $v_3 = \sage{v3}$

$$w_1 = v_1 = \sage{w1}$$
$$w_2 = v_2 - \frac{v_2 \cdot w_1}{w_1 \cdot w_1} w1 = \sage{w2}$$
$$w_3 = v_3 - \frac{v_3 \cdot w_1}{w_1 \cdot w_1} w1 - \frac{v_3 \cdot w_2}{w_2 \cdot w_2} w2 = \sage{w3}$$

% Fin Ejercicio 2


\begin{ejer} 
Calcula la parábola que mejor se ajusta a los datos que se proporcionan a continuación:
\begin{sageblock}
XY = matrix(RR,[[ 0.9791609077659043 ,  -0.430715356970999 ],
                [ 1.5250083079649113 ,  0.061165657369748695 ],
                [ 2.7989027102112827 ,  -0.5015548718775755 ],
                [ 1.912158417289881 ,  0.24440901271557722 ],
                [ 0.8205530309864537 ,  -0.42261180235811696 ],
                [ 1.914008178794904 ,  -0.17482280182386326 ],
                [ 3.0850261801730925 ,  -1.0104522969296543 ],
                [ 0.15356067616723457 ,  -1.7711019287922931 ],
                [ 2.0952887890653016 ,  0.03975209722972092 ],
                [ 2.514022662217597 ,  -0.6049726172196987 ]])

X = XY.column(0)
Y = XY.column(1)
\end{sageblock}

La representación gráfica de estos puntos es:

\begin{sagesub}
\begin{center}
\begin{tikzpicture}
\foreach \p in !{set(XY.rows())}
  \filldraw[blue] \p circle (0.05);
%\draw[red, domain = !{min(X)-.2}:!{max(X)+.2}, very thick]
%  plot(\x,{3.14+2.52*\x+1.22*\x*\x});
\draw[->] (!{min(X)-2},0)--(!{max(X)+2},0) node[right]{$x$};
\draw[->] (0,!{min(Y)-2})--(0,!{max(Y)+2}) node[above]{$y$};
\end{tikzpicture}
\end{center}
\end{sagesub}


\end{ejer} 

{\it Solución:}

% Inicio Ejercicio 3

% Fin Ejercicio 3



\end{document}
