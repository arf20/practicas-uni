\documentclass{amsart}
\usepackage[utf8]{inputenc}
\usepackage[usefamily=sage]{pythontex} 
\usepackage{float}
\usepackage{tikz}
\usetikzlibrary{calc}
\usepackage[most]{tcolorbox}
\usepackage[margin = 2cm]{geometry}
\usepackage{graphicx}
\usetikzlibrary{3d}
\usepackage{tikz-3dplot}

\newtheorem{ejer}{Ejercicio}
\def\r{\mathbb{R}}

\title{Tarea de Geometría Vectorial 3D}

\begin{document}

\begin{sagecode}
latex.matrix_delimiters("[", "]")
\end{sagecode}

\maketitle

\begin{ejer}
Sea $r \leq \r^3$ una recta vectorial dada en ecuaciones implícitas
y sea $b$ un vector de $\r^3$. Dibuja un prisma con base hexagonal
que tenga como uno de los vértices de la base el vector $b$. La altura
estará situada sobre el eje $r$ y la distancia entre las bases será $3$. 
Dibuja los centros de las bases y el eje de giro viendo que pasa por el origen 
de coordenadas.

\begin{tcolorbox}[title = Datos]
\begin{sageblock}
H = matrix(RR,[[-0.171527754943673,-0.886777920259044,-0.326362780186830],
[0.0162571041668937,0.923172294747017,0.586214584852305]])
Pol.<x,y,z> = PolynomialRing(RR)
Ecuaciones = H*column_matrix(Pol,[x,y,z])
b = vector(RR,[0.717434104560504,0.503600818995866,0.274834075678966])
\end{sageblock}

\begin{align*}
r &\equiv \begin{cases} 
  \sage{Ecuaciones[0,0]} = 0 \\
  \sage{Ecuaciones[1,0]} = 0 \\
  \end{cases} \\
b &= \sage{column_matrix(b)}.
\end{align*}
\end{tcolorbox}

Te debe dar un dibujo similar a éste, pero con los valores calculados por tí:

\begin{center}
\tdplotsetmaincoords{70}{110}
\begin{tikzpicture}[scale = 3, tdplot_main_coords,
  cara/.style={thick, color = blue, fill opacity = 0.3, fill = blue!20},
  ]
\draw[->,thick,gray] (-1.5,0,0) -- (2.5,0,0); % Eje X
\draw[->,thick,gray] (0,-1.5,0) -- (0,3.5,0); % Eje Y
\draw[->,thick,gray] (0,0,-1.5) -- (0,0,1.5); % Eje Z
\draw[cara] (0.717434104560504, 0.503600818995866, 0.274834075678966) -- (0.260986986832384, 0.295012074171375, 0.829887291884511) -- (-0.0382759901376177, -0.390829498657474, 0.830449577031793) -- (0.118908150620501, -0.868082326661832, 0.275958645973531) -- (0.575355268348621, -0.659493581837341, -0.279094570232013) -- (0.874618245318622, 0.0263479909915075, -0.279656855379295) --cycle;
\draw[cara] (-1.63694189424483, 1.52964788087894, -1.27569521557950) -- (-2.09338901197295, 1.32105913605445, -0.720641999373955) -- (-2.39265198894296, 0.635217563225603, -0.720079714226673) -- (-2.23546784818484, 0.157964735221246, -1.27457064528493) -- (-1.77902073045672, 0.366553480045737, -1.82962386149048) -- (-1.47975775348672, 1.05239505287458, -1.83018614663776) --cycle;
\draw[cara] (0.717434104560504, 0.503600818995866, 0.274834075678966) -- (-1.63694189424483, 1.52964788087894, -1.27569521557950) -- (-2.09338901197295, 1.32105913605445, -0.720641999373955) -- (0.260986986832384, 0.295012074171375, 0.829887291884511) -- cycle;
\draw[cara] (0.260986986832384, 0.295012074171375, 0.829887291884511) -- (-2.09338901197295, 1.32105913605445, -0.720641999373955) -- (-2.39265198894296, 0.635217563225603, -0.720079714226673) -- (-0.0382759901376177, -0.390829498657474, 0.830449577031793) -- cycle;
\draw[cara] (-0.0382759901376177, -0.390829498657474, 0.830449577031793) -- (-2.39265198894296, 0.635217563225603, -0.720079714226673) -- (-2.23546784818484, 0.157964735221246, -1.27457064528493) -- (0.118908150620501, -0.868082326661832, 0.275958645973531) -- cycle;
\draw[cara] (0.118908150620501, -0.868082326661832, 0.275958645973531) -- (-2.23546784818484, 0.157964735221246, -1.27457064528493) -- (-1.77902073045672, 0.366553480045737, -1.82962386149048) -- (0.575355268348621, -0.659493581837341, -0.279094570232013) -- cycle;
\draw[cara] (0.575355268348621, -0.659493581837341, -0.279094570232013) -- (-1.77902073045672, 0.366553480045737, -1.82962386149048) -- (-1.47975775348672, 1.05239505287458, -1.83018614663776) -- (0.874618245318622, 0.0263479909915075, -0.279656855379295) -- cycle;
\draw[cara] (0.874618245318622, 0.0263479909915075, -0.279656855379295) -- (-1.47975775348672, 1.05239505287458, -1.83018614663776) -- (-1.63694189424483, 1.52964788087894, -1.27569521557950) -- (0.717434104560504, 0.503600818995866, 0.274834075678966) -- cycle;
\draw[fill = white] (0.717434104560504, 0.503600818995866, 0.274834075678966) circle (0.5mm) node[above right] {$b = b_0$};
\draw[fill = white] (0.260986986832384, 0.295012074171375, 0.829887291884511) circle (0.5mm) node[above right] {$b_1$};
\draw[fill = white] (-0.0382759901376177, -0.390829498657474, 0.830449577031793) circle (0.5mm) node[above right] {$b_2$};
\draw[fill = white] (0.118908150620501, -0.868082326661832, 0.275958645973531) circle (0.5mm) node[below left] {$b_3$};
\draw[fill = white] (0.575355268348621, -0.659493581837341, -0.279094570232013) circle (0.5mm) node[below left] {$b_4$};
\draw[fill = white] (0.874618245318622, 0.0263479909915075, -0.279656855379295) circle (0.5mm) node[below left] {$b_5$};
\draw[fill = white] (-1.63694189424483, 1.52964788087894, -1.27569521557950) circle (0.5mm) node[above right] {$c = c_0$};
\draw[fill = white] (-2.09338901197295, 1.32105913605445, -0.720641999373955) circle (0.5mm) node[above right] {$c_1$};
\draw[fill = white] (-2.39265198894296, 0.635217563225603, -0.720079714226673) circle (0.5mm) node[above right] {$c_2$};
\draw[fill = white] (-2.23546784818484, 0.157964735221246, -1.27457064528493) circle (0.5mm) node[below left] {$c_3$};
\draw[fill = white] (-1.77902073045672, 0.366553480045737, -1.82962386149048) circle (0.5mm) node[below left] {$c_4$};
\draw[fill = white] (-1.47975775348672, 1.05239505287458, -1.83018614663776) circle (0.5mm) node[below left] {$c_5$};
\draw[dashed] (0.627256691385754, -0.273361130749475, 0.413094541239373) -- (-2.12982535833632, 0.928186938855103, -1.40264622347544);
\draw[fill = white] (0.418171127590502, -0.182240753832983, 0.275396360826249) circle (0.5mm);
\draw[fill = white] (-1.93620487121484, 0.843806308050093, -1.27513293043222) circle (0.5mm);
\draw[fill = gray] (0,0,0) circle (0.5mm);
\end{tikzpicture}
\end{center}

\end{ejer}

{\it Solución:}

% Inicio Tarea

\begin{sageblock}
w1 = vector(RR, H.row(0))
w2 = vector(RR, H.row(1))
vr = w1.cross_product(w2)
b1 = vr.normalized()
b2 = w1.normalized()
b3 = vr.cross_product(w1).normalized()
Bproj = block_matrix([[b1.column(), b2.column(), b3.column()]])
Brot = block_matrix([[b2.column(), b3.column(), b1.column()]])

Mprojr = matrix(RR, [[1, 0, 0], [0, 0, 0], [0, 0, 0]])
O = (Bproj * Mprojr * Bproj^-1) * b

a = 2*pi/6
Mrot = matrix(RR, [[cos(a), -sin(a), 0], [sin(a), cos(a), 0], [0, 0, 1]])
V = [((Brot*Mrot^i*Brot^-1)*b) for i in range(6)]



\end{sageblock}

Las filas de H son dos vectores perpendiculares $\vec{w_1}$ y $\vec{w_2}$ a $r$, sacar $\vec{v_r}$
$$
	\vec{v_r} = \vec{w_1} \times \vec{w_2} = \sage{vr}
$$

Hacer base ortonormal con $\vec{w_1}$ y $\vec{v_r}$ que son perpendiculares entre si, y el perpendicular a esos dos
\begin{align*}
	\vec{b_1} &= \frac{\vec{v_r}}{||\vec{v_r}||} \\
	\vec{b_2} &= \frac{\vec{w_1}}{||\vec{w_1}||} \\
	\vec{b_3} &= \frac{\vec{v_r} \times \vec{w_1}}{||\vec{v_r} \times \vec{w_1}||} \\
\end{align*}
$$
	B = \begin{bmatrix}
			\vec{b_1} & \vec{b_2} & \vec{b_3} \\
		\end{bmatrix}
$$

\begin{center}
\begin{sagesub}


\tdplotsetmaincoords{70}{110}
\begin{tikzpicture}[scale = 3, tdplot_main_coords,
cara/.style={thick, color = blue, fill opacity = 0.3, fill = blue!20},]

\draw[->,thick,gray] (-1.5,0,0) -- (2.5,0,0); % Eje X
\draw[->,thick,gray] (0,-1.5,0) -- (0,3.5,0); % Eje Y
\draw[->,thick,gray] (0,0,-1.5) -- (0,0,1.5); % Eje Z

\draw[->,thick,red] (0,0,0) -- !{b1};
\draw[->,thick,red] (0,0,0) -- !{b2};
\draw[->,thick,red] (0,0,0) -- !{b3};


\draw[cara] !{V[0]} -- !{V[1]} -- !{V[2]} -- !{V[3]} -- !{V[4]} -- !{V[5]} --cycle;

\draw[fill = white] !{V[0]} circle (0.5mm) node[above right] {$b_0 = b$};
\draw[fill = white] !{V[1]} circle (0.5mm) node[above right] {$b_1$};
\draw[fill = white] !{V[2]} circle (0.5mm) node[above right] {$b_2$};
\draw[fill = white] !{V[3]} circle (0.5mm) node[above right] {$b_3$};
\draw[fill = white] !{V[4]} circle (0.5mm) node[above right] {$b_4$};
\draw[fill = white] !{V[5]} circle (0.5mm) node[above right] {$b_5$};

\draw[fill = white] !{O} circle (0.5mm);


\draw[fill = gray] (0,0,0) circle (0.5mm);
\end{tikzpicture}
\end{sagesub}
\end{center}

% Fin Tarea

\end{document}


