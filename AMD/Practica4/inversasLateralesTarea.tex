\documentclass{amsart}
\usepackage{sagetex}
\usepackage[utf8]{inputenc}
\usepackage[most]{tcolorbox}
\newtheorem{ejer}{Ejercicio}

\title{Pr\'acticas de \'Algebra y Matem\'atica Discreta. Inversas Laterales}

\begin{document}
\maketitle

\begin{sagesilent}
import random
set_random_seed(23836) # Cambia este número por las 5 primeras cifras de tu DNI o NIE

a = random.choice(range(5,21))
b = random.choice(range(22,41))
c = random.choice(range(52,61))

# los que me salió la primera vez, fijados, porque set_random_seed no es la unica seed del RNG de sage, se ve que también toma el tiempo
a = 7
b = 24
c = 52
\end{sagesilent}

\begin{tcolorbox}[colback = red!40!white]
Los ejercicios que debes realizar para completar esta tarea son al menos el Ejercicio~\sage{a},
el Ejercicio~\sage{b} y el Ejercicio~\sage{c}.
\end{tcolorbox}

\section{Inversas Laterales}

\begin{ejer} Determina si la siguiente matriz tiene inversas laterales y calc\'ulalas en caso de existir,
\[ \left[\begin{array}{rrrr}
11 & 16 & 0 & 14 \\
7 & 15 & 8 & 15 \\
1 & 10 & 6 & 15 \\
4 & 9 & 6 & 4
\end{array}\right] \in {\bf M}_{4\times 4}({\mathbb Z}_{17})\]
\end{ejer}

{\it Soluci\'on:}
% Escribe tu soluci\'on para el Ejercicio 1

\begin{sageblock}
matrix(Zmod(17),[[11,16,0,14],
[7,15,8,15],
[1,10,6,15],
[4,9,6,4]])
\end{sageblock}

% Fin del Ejercicio 1


\begin{ejer} Determina si la siguiente matriz tiene inversas laterales y calc\'ulalas en caso de existir,
\[ \left[\begin{array}{rrrrrr}
28 & 10 & 8 & 4 & 5 & 1 \\
19 & 14 & 15 & 14 & 22 & 10 \\
28 & 8 & 20 & 24 & 17 & 28 \\
10 & 16 & 16 & 3 & 25 & 20 \\
13 & 14 & 23 & 13 & 25 & 16 \\
27 & 4 & 28 & 18 & 9 & 14
\end{array}\right] \in {\bf M}_{6\times 6}({\mathbb Z}_{29})\]
\end{ejer}

{\it Soluci\'on:}
% Escribe tu soluci\'on para el Ejercicio 2

\begin{sageblock}
matrix(Zmod(29),[[28,10,8,4,5,1],
[19,14,15,14,22,10],
[28,8,20,24,17,28],
[10,16,16,3,25,20],
[13,14,23,13,25,16],
[27,4,28,18,9,14]])
\end{sageblock}

% Fin del Ejercicio 2


\begin{ejer} Determina si la siguiente matriz tiene inversas laterales y calc\'ulalas en caso de existir,
\[ \left[\begin{array}{rrrrrr}
1 & -1 & -1 & -2 & 2 & 4 \\
0 & 1 & 0 & 1 & -3 & -4 \\
0 & -1 & 1 & 2 & 6 & -4 \\
-1 & -1 & 1 & 1 & 4 & 1
\end{array}\right] \in {\bf M}_{4\times 6}({\mathbb R})\]
\end{ejer}

{\it Soluci\'on:}
% Escribe tu soluci\'on para el Ejercicio 3

\begin{sageblock}
matrix(QQ,[[1,-1,-1,-2,2,4],
[0,1,0,1,-3,-4],
[0,-1,1,2,6,-4],
[-1,-1,1,1,4,1]])
\end{sageblock}

% Fin del Ejercicio 3


\begin{ejer} Determina si la siguiente matriz tiene inversas laterales y calc\'ulalas en caso de existir,
\[ \left[\begin{array}{rrrr}
12 & 9 & 3 & 0 \\
13 & 17 & 38 & 39 \\
17 & 12 & 15 & 36 \\
16 & 30 & 37 & 40 \\
36 & 10 & 5 & 14 \\
38 & 15 & 19 & 16
\end{array}\right] \in {\bf M}_{6\times 4}({\mathbb Z}_{43})\]
\end{ejer}

{\it Soluci\'on:}
% Escribe tu soluci\'on para el Ejercicio 4

\begin{sageblock}
matrix(Zmod(43),[[12,9,3,0],
[13,17,38,39],
[17,12,15,36],
[16,30,37,40],
[36,10,5,14],
[38,15,19,16]])
\end{sageblock}

% Fin del Ejercicio 4


\begin{ejer} Determina si la siguiente matriz tiene inversas laterales y calc\'ulalas en caso de existir,
\[ \left[\begin{array}{rrrr}
28 & 8 & 22 & 14 \\
22 & 0 & 26 & 2 \\
11 & 21 & 17 & 2 \\
28 & 2 & 20 & 21
\end{array}\right] \in {\bf M}_{4\times 4}({\mathbb Z}_{29})\]
\end{ejer}

{\it Soluci\'on:}
% Escribe tu soluci\'on para el Ejercicio 5

\begin{sageblock}
matrix(Zmod(29),[[28,8,22,14],
[22,0,26,2],
[11,21,17,2],
[28,2,20,21]])
\end{sageblock}

% Fin del Ejercicio 5


\begin{ejer} Determina si la siguiente matriz tiene inversas laterales y calc\'ulalas en caso de existir,
\[ \left[\begin{array}{rrrrrr}
5 & 3 & 3 & -1 & 9 & -6 \\
-2 & -1 & -1 & 0 & -5 & 1 \\
2 & 1 & 2 & 3 & 5 & -8 \\
1 & 0 & 0 & 2 & 7 & 3 \\
0 & 0 & 0 & 2 & 3 & 1 \\
2 & 2 & 2 & -6 & -3 & -4
\end{array}\right] \in {\bf M}_{6\times 6}({\mathbb R})\]
\end{ejer}

{\it Soluci\'on:}
% Escribe tu soluci\'on para el Ejercicio 6

\begin{sageblock}
matrix(QQ,[[5,3,3,-1,9,-6],
[-2,-1,-1,0,-5,1],
[2,1,2,3,5,-8],
[1,0,0,2,7,3],
[0,0,0,2,3,1],
[2,2,2,-6,-3,-4]])
\end{sageblock}

% Fin del Ejercicio 6


\begin{ejer} Determina si la siguiente matriz tiene inversas laterales y calc\'ulalas en caso de existir,
\[ \left[\begin{array}{rrrr}
7 & 34 & 28 & 30 \\
31 & 30 & 31 & 13 \\
29 & 12 & 6 & 17 \\
8 & 28 & 11 & 23 \\
1 & 0 & 17 & 22 \\
17 & 35 & 31 & 4 \\
14 & 32 & 0 & 29 \\
9 & 17 & 15 & 32
\end{array}\right] \in {\bf M}_{8\times 4}({\mathbb Z}_{37})\]
\end{ejer}

{\it Soluci\'on:}
% Escribe tu soluci\'on para el Ejercicio 7

\begin{sageblock}
A = matrix(Zmod(37),[[7,34,28,30],
[31,30,31,13],
[29,12,6,17],
[8,28,11,23],
[1,0,17,22],
[17,35,31,4],
[14,32,0,29],
[9,17,15,32]])

Ap = block_matrix([[A, 1]])
Ar = Ap.echelon_form()

Atp = block_matrix([[A.T, 1]])
Atr = Atp.echelon_form()
Atr = copy(Atr)
Atr.subdivide([3], [3])
\end{sageblock}

Ampliada
$$
	\sage{Ap}
$$
Reducida
$$
	\sage{Ar}
$$
No sale identidad por tanto no tiene inversa por la izquierda

Transpuesta ampliada
$$
	\sage{Atp}
$$
Reducida
$$
	\sage{Atr} = \begin{bmatrix}
		I & B \\
		0 & H \\
	\end{bmatrix}
$$
Sale identidad, por tanto existe matriz inversa lateral por la izquierda, que es $B$
$$
	B = \sage{Atr[:3, 3:]}
$$


% Fin del Ejercicio 7


\begin{ejer} Determina si la siguiente matriz tiene inversas laterales y calc\'ulalas en caso de existir,
\[ \left[\begin{array}{rrrr}
3 & 3 & 2 & 4 \\
4 & 0 & 0 & 3 \\
4 & 0 & 1 & 3 \\
3 & 4 & 0 & 4 \\
1 & 1 & 2 & 2 \\
0 & 0 & 3 & 4
\end{array}\right] \in {\bf M}_{6\times 4}({\mathbb Z}_{5})\]
\end{ejer}

{\it Soluci\'on:}
% Escribe tu soluci\'on para el Ejercicio 8

\begin{sageblock}
matrix(Zmod(5),[[3,3,2,4],
[4,0,0,3],
[4,0,1,3],
[3,4,0,4],
[1,1,2,2],
[0,0,3,4]])
\end{sageblock}

% Fin del Ejercicio 8


\begin{ejer} Determina si la siguiente matriz tiene inversas laterales y calc\'ulalas en caso de existir,
\[ \left[\begin{array}{rrrrrr}
1 & -2 & 1 & 3 & 3 & 4 \\
-1 & 3 & -1 & -6 & -7 & -7 \\
-2 & 1 & -1 & 0 & 2 & -1 \\
-2 & 3 & -1 & -5 & -5 & -6
\end{array}\right] \in {\bf M}_{4\times 6}({\mathbb R})\]
\end{ejer}

{\it Soluci\'on:}
% Escribe tu soluci\'on para el Ejercicio 9

\begin{sageblock}
matrix(QQ,[[1,-2,1,3,3,4],
[-1,3,-1,-6,-7,-7],
[-2,1,-1,0,2,-1],
[-2,3,-1,-5,-5,-6]])
\end{sageblock}

% Fin del Ejercicio 9


\begin{ejer} Determina si la siguiente matriz tiene inversas laterales y calc\'ulalas en caso de existir,
\[ \left[\begin{array}{rrrr}
5 & 27 & 35 & 2 \\
3 & 16 & 31 & 6 \\
8 & 6 & 26 & 13 \\
33 & 15 & 33 & 14
\end{array}\right] \in {\bf M}_{4\times 4}({\mathbb Z}_{37})\]
\end{ejer}

{\it Soluci\'on:}
% Escribe tu soluci\'on para el Ejercicio 10

\begin{sageblock}
matrix(Zmod(37),[[5,27,35,2],
[3,16,31,6],
[8,6,26,13],
[33,15,33,14]])
\end{sageblock}

% Fin del Ejercicio 10


\begin{ejer} Determina si la siguiente matriz tiene inversas laterales y calc\'ulalas en caso de existir,
\[ \left[\begin{array}{rrrrrrrr}
0 & 0 & -1 & -1 & -1 & -4 & -2 & 3 \\
1 & 2 & -1 & 0 & -1 & -4 & 2 & 2 \\
0 & 0 & 0 & 1 & 2 & 4 & 2 & -8 \\
0 & 0 & -1 & -2 & -2 & -6 & -4 & 7
\end{array}\right] \in {\bf M}_{4\times 8}({\mathbb R})\]
\end{ejer}

{\it Soluci\'on:}
% Escribe tu soluci\'on para el Ejercicio 11

\begin{sageblock}
matrix(QQ,[[0,0,-1,-1,-1,-4,-2,3],
[1,2,-1,0,-1,-4,2,2],
[0,0,0,1,2,4,2,-8],
[0,0,-1,-2,-2,-6,-4,7]])
\end{sageblock}

% Fin del Ejercicio 11


\begin{ejer} Determina si la siguiente matriz tiene inversas laterales y calc\'ulalas en caso de existir,
\[ \left[\begin{array}{rrrrrr}
1 & 1 & 2 & 0 & 1 & 7 \\
-1 & 0 & -1 & -1 & 3 & -5 \\
-1 & -1 & -1 & -2 & 5 & -4 \\
-1 & 1 & 0 & -1 & 4 & -3
\end{array}\right] \in {\bf M}_{4\times 6}({\mathbb R})\]
\end{ejer}

{\it Soluci\'on:}
% Escribe tu soluci\'on para el Ejercicio 12

\begin{sageblock}
matrix(QQ,[[1,1,2,0,1,7],
[-1,0,-1,-1,3,-5],
[-1,-1,-1,-2,5,-4],
[-1,1,0,-1,4,-3]])
\end{sageblock}

% Fin del Ejercicio 12


\begin{ejer} Determina si la siguiente matriz tiene inversas laterales y calc\'ulalas en caso de existir,
\[ \left[\begin{array}{rrrr}
5 & -5 & -6 & -1 \\
0 & 1 & 2 & -8 \\
-4 & 4 & 5 & 0 \\
-3 & 2 & 3 & 4 \\
4 & -4 & -6 & 5 \\
-4 & 5 & 5 & -2 \\
0 & 0 & -3 & 9 \\
-2 & 2 & 0 & 6
\end{array}\right] \in {\bf M}_{8\times 4}({\mathbb R})\]
\end{ejer}

{\it Soluci\'on:}
% Escribe tu soluci\'on para el Ejercicio 13

\begin{sageblock}
matrix(QQ,[[5,-5,-6,-1],
[0,1,2,-8],
[-4,4,5,0],
[-3,2,3,4],
[4,-4,-6,5],
[-4,5,5,-2],
[0,0,-3,9],
[-2,2,0,6]])
\end{sageblock}

% Fin del Ejercicio 13


\begin{ejer} Determina si la siguiente matriz tiene inversas laterales y calc\'ulalas en caso de existir,
\[ \left[\begin{array}{rrrrrr}
1 & 1 & 0 & -2 & 1 & 5 \\
0 & 1 & 1 & -3 & 6 & 6 \\
0 & -2 & -1 & 3 & -9 & -5 \\
-1 & -2 & 0 & 3 & -4 & -7
\end{array}\right] \in {\bf M}_{4\times 6}({\mathbb R})\]
\end{ejer}

{\it Soluci\'on:}
% Escribe tu soluci\'on para el Ejercicio 14

\begin{sageblock}
matrix(QQ,[[1,1,0,-2,1,5],
[0,1,1,-3,6,6],
[0,-2,-1,3,-9,-5],
[-1,-2,0,3,-4,-7]])
\end{sageblock}

% Fin del Ejercicio 14


\begin{ejer} Determina si la siguiente matriz tiene inversas laterales y calc\'ulalas en caso de existir,
\[ \left[\begin{array}{rrrrrr}
1 & 1 & 2 & 4 & 2 & 9 \\
-1 & 0 & 0 & 1 & 1 & -1 \\
1 & -1 & -1 & -3 & 0 & 1 \\
1 & 0 & 0 & 0 & -1 & 4 \\
0 & 2 & 2 & 6 & -1 & 7 \\
0 & -1 & -2 & -3 & -4 & -2
\end{array}\right] \in {\bf M}_{6\times 6}({\mathbb R})\]
\end{ejer}

{\it Soluci\'on:}
% Escribe tu soluci\'on para el Ejercicio 15

\begin{sageblock}
matrix(QQ,[[1,1,2,4,2,9],
[-1,0,0,1,1,-1],
[1,-1,-1,-3,0,1],
[1,0,0,0,-1,4],
[0,2,2,6,-1,7],
[0,-1,-2,-3,-4,-2]])
\end{sageblock}

% Fin del Ejercicio 15


\begin{ejer} Determina si la siguiente matriz tiene inversas laterales y calc\'ulalas en caso de existir,
\[ \left[\begin{array}{rrrr}
10 & 15 & 15 & 22 \\
20 & 0 & 1 & 13 \\
7 & 8 & 10 & 3 \\
3 & 6 & 14 & 15 \\
21 & 11 & 9 & 12 \\
12 & 16 & 13 & 11
\end{array}\right] \in {\bf M}_{6\times 4}({\mathbb Z}_{23})\]
\end{ejer}

{\it Soluci\'on:}
% Escribe tu soluci\'on para el Ejercicio 16

\begin{sageblock}
matrix(Zmod(23),[[10,15,15,22],
[20,0,1,13],
[7,8,10,3],
[3,6,14,15],
[21,11,9,12],
[12,16,13,11]])
\end{sageblock}

% Fin del Ejercicio 16


\begin{ejer} Determina si la siguiente matriz tiene inversas laterales y calc\'ulalas en caso de existir,
\[ \left[\begin{array}{rrrr}
6 & 6 & 5 & 4 \\
10 & 6 & 1 & 9 \\
4 & 0 & 1 & 6 \\
3 & 7 & 10 & 4
\end{array}\right] \in {\bf M}_{4\times 4}({\mathbb Z}_{11})\]
\end{ejer}

{\it Soluci\'on:}
% Escribe tu soluci\'on para el Ejercicio 17

\begin{sageblock}
matrix(Zmod(11),[[6,6,5,4],
[10,6,1,9],
[4,0,1,6],
[3,7,10,4]])
\end{sageblock}

% Fin del Ejercicio 17


\begin{ejer} Determina si la siguiente matriz tiene inversas laterales y calc\'ulalas en caso de existir,
\[ \left[\begin{array}{rrrr}
6 & 6 & 0 & 6 \\
1 & 0 & 1 & 6 \\
1 & 2 & 0 & 2 \\
6 & 1 & 5 & 3
\end{array}\right] \in {\bf M}_{4\times 4}({\mathbb Z}_{7})\]
\end{ejer}

{\it Soluci\'on:}
% Escribe tu soluci\'on para el Ejercicio 18

\begin{sageblock}
matrix(Zmod(7),[[6,6,0,6],
[1,0,1,6],
[1,2,0,2],
[6,1,5,3]])
\end{sageblock}

% Fin del Ejercicio 18


\begin{ejer} Determina si la siguiente matriz tiene inversas laterales y calc\'ulalas en caso de existir,
\[ \left[\begin{array}{rrrrrr}
27 & 35 & 31 & 10 & 37 & 19 \\
11 & 30 & 46 & 38 & 27 & 34 \\
35 & 13 & 20 & 3 & 23 & 38 \\
6 & 45 & 40 & 25 & 23 & 19 \\
28 & 1 & 11 & 40 & 6 & 20 \\
7 & 37 & 15 & 15 & 32 & 27
\end{array}\right] \in {\bf M}_{6\times 6}({\mathbb Z}_{47})\]
\end{ejer}

{\it Soluci\'on:}
% Escribe tu soluci\'on para el Ejercicio 19

\begin{sageblock}
matrix(Zmod(47),[[27,35,31,10,37,19],
[11,30,46,38,27,34],
[35,13,20,3,23,38],
[6,45,40,25,23,19],
[28,1,11,40,6,20],
[7,37,15,15,32,27]])
\end{sageblock}

% Fin del Ejercicio 19


\begin{ejer} Determina si la siguiente matriz tiene inversas laterales y calc\'ulalas en caso de existir,
\[ \left[\begin{array}{rrrr}
3 & 10 & 9 & 8 \\
5 & 0 & 2 & 8 \\
2 & 8 & 9 & 10 \\
2 & 8 & 9 & 0
\end{array}\right] \in {\bf M}_{4\times 4}({\mathbb Z}_{11})\]
\end{ejer}

{\it Soluci\'on:}
% Escribe tu soluci\'on para el Ejercicio 20

\begin{sageblock}
matrix(Zmod(11),[[3,10,9,8],
[5,0,2,8],
[2,8,9,10],
[2,8,9,0]])
\end{sageblock}

% Fin del Ejercicio 20

\section{Vectores Linealmente Independientes y Generadores de $K^n$}

\begin{ejer} Determina si las columnas de esta matriz son vectores linealmente independientes, generadores y/o base de ${{\mathbb R}}^{6}$,
\[ A = \left[\begin{array}{rrrr}
2 & 1 & -5 & 9 \\
1 & 1 & -1 & 2 \\
0 & -1 & -2 & 2 \\
1 & 2 & 2 & -2 \\
0 & -1 & 1 & -4 \\
-1 & 1 & 5 & -4
\end{array}\right] \]
\end{ejer}

{\it Soluci\'on:}
% Escribe tu soluci\'on para el Ejercicio 21

\begin{sageblock}
matrix(QQ,[[2,1,-5,9],
[1,1,-1,2],
[0,-1,-2,2],
[1,2,2,-2],
[0,-1,1,-4],
[-1,1,5,-4]])
\end{sageblock}

% Fin del Ejercicio 21


\begin{ejer} Determina si las columnas de esta matriz son vectores linealmente independientes, generadores y/o base de ${{\mathbb R}}^{5}$,
\[ A = \left[\begin{array}{rrrr}
-1 & 2 & 8 & 1 \\
0 & 1 & 3 & 1 \\
1 & 1 & 0 & 2 \\
0 & -1 & 2 & -1 \\
0 & 1 & 5 & 1
\end{array}\right] \]
\end{ejer}

{\it Soluci\'on:}
% Escribe tu soluci\'on para el Ejercicio 22

\begin{sageblock}
matrix(QQ,[[-1,2,8,1],
[0,1,3,1],
[1,1,0,2],
[0,-1,2,-1],
[0,1,5,1]])
\end{sageblock}

% Fin del Ejercicio 22


\begin{ejer} Determina si las columnas de esta matriz son vectores linealmente independientes, generadores y/o base de ${{\mathbb R}}^{5}$,
\[ A = \left[\begin{array}{rrrrrr}
0 & 2 & 2 & -1 & -3 & 7 \\
0 & 1 & 3 & -2 & 1 & 7 \\
0 & 0 & 1 & -1 & 0 & 3 \\
1 & -1 & -1 & 0 & -5 & 2 \\
0 & 1 & 2 & -1 & 2 & 3
\end{array}\right] \]
\end{ejer}

{\it Soluci\'on:}
% Escribe tu soluci\'on para el Ejercicio 23

\begin{sageblock}
matrix(QQ,[[0,2,2,-1,-3,7],
[0,1,3,-2,1,7],
[0,0,1,-1,0,3],
[1,-1,-1,0,-5,2],
[0,1,2,-1,2,3]])
\end{sageblock}

% Fin del Ejercicio 23


\begin{ejer} Determina si las columnas de esta matriz son vectores linealmente independientes, generadores y/o base de ${{\mathbb R}}^{6}$,
\[ A = \left[\begin{array}{rrrrr}
3 & 1 & 7 & 8 & 6 \\
2 & 1 & 6 & 6 & 3 \\
0 & 2 & 9 & 5 & -5 \\
3 & 2 & 6 & 6 & 3 \\
0 & 1 & 8 & 4 & -8 \\
1 & -1 & -7 & -4 & 2
\end{array}\right] \]
\end{ejer}

{\it Soluci\'on:}
% Escribe tu soluci\'on para el Ejercicio 24

\begin{sageblock}
A = matrix(QQ,[[3,1,7,8,6],
[2,1,6,6,3],
[0,2,9,5,-5],
[3,2,6,6,3],
[0,1,8,4,-8],
[1,-1,-7,-4,2]])

Ar = A.echelon_form()

\end{sageblock}

Al reducir la matriz, sale

$$
	\sage{Ar}
$$

Cuyo rango (n de pivotes) es 5. El rango coincide con el numero de columnas (5) pero no con el numero de filas (6), por tanto los vectores columnas son linealmente independientes, es decir, inyectiva.

% Fin del Ejercicio 24


\begin{ejer} Determina si las columnas de esta matriz son vectores linealmente independientes, generadores y/o base de ${{\mathbb Z}_{29}}^{6}$,
\[ A = \left[\begin{array}{rrr}
10 & 24 & 11 \\
2 & 2 & 20 \\
11 & 5 & 0 \\
11 & 13 & 16 \\
6 & 16 & 12 \\
28 & 20 & 25
\end{array}\right] \]
\end{ejer}

{\it Soluci\'on:}
% Escribe tu soluci\'on para el Ejercicio 25

\begin{sageblock}
matrix(Zmod(29),[[10,24,11],
[2,2,20],
[11,5,0],
[11,13,16],
[6,16,12],
[28,20,25]])
\end{sageblock}

% Fin del Ejercicio 25


\begin{ejer} Determina si las columnas de esta matriz son vectores linealmente independientes, generadores y/o base de ${{\mathbb Z}_{13}}^{6}$,
\[ A = \left[\begin{array}{rrrrrr}
12 & 2 & 10 & 11 & 1 & 3 \\
11 & 3 & 11 & 12 & 2 & 3 \\
11 & 1 & 6 & 4 & 3 & 8 \\
8 & 12 & 12 & 6 & 3 & 10 \\
3 & 2 & 5 & 12 & 12 & 11 \\
10 & 4 & 10 & 7 & 12 & 9
\end{array}\right] \]
\end{ejer}

{\it Soluci\'on:}
% Escribe tu soluci\'on para el Ejercicio 26

\begin{sageblock}
matrix(Zmod(13),[[12,2,10,11,1,3],
[11,3,11,12,2,3],
[11,1,6,4,3,8],
[8,12,12,6,3,10],
[3,2,5,12,12,11],
[10,4,10,7,12,9]])
\end{sageblock}

% Fin del Ejercicio 26


\begin{ejer} Determina si las columnas de esta matriz son vectores linealmente independientes, generadores y/o base de ${{\mathbb R}}^{8}$,
\[ A = \left[\begin{array}{rrr}
-1 & 0 & 3 \\
0 & 1 & -3 \\
-1 & -3 & 9 \\
1 & 1 & -4 \\
0 & -3 & 6 \\
0 & -2 & 8 \\
-1 & -3 & 9 \\
0 & -3 & 4
\end{array}\right] \]
\end{ejer}

{\it Soluci\'on:}
% Escribe tu soluci\'on para el Ejercicio 27

\begin{sageblock}
matrix(QQ,[[-1,0,3],
[0,1,-3],
[-1,-3,9],
[1,1,-4],
[0,-3,6],
[0,-2,8],
[-1,-3,9],
[0,-3,4]])
\end{sageblock}

% Fin del Ejercicio 27


\begin{ejer} Determina si las columnas de esta matriz son vectores linealmente independientes, generadores y/o base de ${{\mathbb Z}_{7}}^{7}$,
\[ A = \left[\begin{array}{rrrr}
3 & 5 & 4 & 1 \\
2 & 6 & 1 & 0 \\
6 & 1 & 5 & 5 \\
0 & 5 & 2 & 6 \\
2 & 2 & 4 & 2 \\
5 & 1 & 5 & 4 \\
2 & 6 & 6 & 4
\end{array}\right] \]
\end{ejer}

{\it Soluci\'on:}
% Escribe tu soluci\'on para el Ejercicio 28

\begin{sageblock}
matrix(Zmod(7),[[3,5,4,1],
[2,6,1,0],
[6,1,5,5],
[0,5,2,6],
[2,2,4,2],
[5,1,5,4],
[2,6,6,4]])
\end{sageblock}

% Fin del Ejercicio 28


\begin{ejer} Determina si las columnas de esta matriz son vectores linealmente independientes, generadores y/o base de ${{\mathbb R}}^{5}$,
\[ A = \left[\begin{array}{rrrrr}
-1 & 1 & -2 & 4 & -2 \\
0 & 1 & -2 & 8 & -6 \\
0 & 1 & -1 & 4 & -4 \\
0 & 0 & -1 & 5 & -2 \\
-1 & 0 & 0 & -1 & 3
\end{array}\right] \]
\end{ejer}

{\it Soluci\'on:}
% Escribe tu soluci\'on para el Ejercicio 29

\begin{sageblock}
matrix(QQ,[[-1,1,-2,4,-2],
[0,1,-2,8,-6],
[0,1,-1,4,-4],
[0,0,-1,5,-2],
[-1,0,0,-1,3]])
\end{sageblock}

% Fin del Ejercicio 29


\begin{ejer} Determina si las columnas de esta matriz son vectores linealmente independientes, generadores y/o base de ${{\mathbb Z}_{5}}^{7}$,
\[ A = \left[\begin{array}{rrr}
4 & 1 & 4 \\
2 & 2 & 2 \\
3 & 0 & 4 \\
0 & 3 & 0 \\
4 & 3 & 3 \\
0 & 0 & 0 \\
3 & 0 & 2
\end{array}\right] \]
\end{ejer}

{\it Soluci\'on:}
% Escribe tu soluci\'on para el Ejercicio 30

\begin{sageblock}
matrix(Zmod(5),[[4,1,4],
[2,2,2],
[3,0,4],
[0,3,0],
[4,3,3],
[0,0,0],
[3,0,2]])
\end{sageblock}

% Fin del Ejercicio 30


\begin{ejer} Determina si las columnas de esta matriz son vectores linealmente independientes, generadores y/o base de ${{\mathbb Z}_{13}}^{6}$,
\[ A = \left[\begin{array}{rrrrr}
6 & 0 & 10 & 8 & 10 \\
12 & 11 & 5 & 7 & 6 \\
1 & 2 & 9 & 9 & 7 \\
7 & 8 & 7 & 4 & 3 \\
1 & 5 & 1 & 6 & 2 \\
8 & 8 & 10 & 8 & 7
\end{array}\right] \]
\end{ejer}

{\it Soluci\'on:}
% Escribe tu soluci\'on para el Ejercicio 31

\begin{sageblock}
matrix(Zmod(13),[[6,0,10,8,10],
[12,11,5,7,6],
[1,2,9,9,7],
[7,8,7,4,3],
[1,5,1,6,2],
[8,8,10,8,7]])
\end{sageblock}

% Fin del Ejercicio 31


\begin{ejer} Determina si las columnas de esta matriz son vectores linealmente independientes, generadores y/o base de ${{\mathbb Z}_{17}}^{8}$,
\[ A = \left[\begin{array}{rrr}
12 & 9 & 8 \\
5 & 10 & 4 \\
5 & 15 & 15 \\
3 & 11 & 11 \\
4 & 2 & 4 \\
0 & 15 & 12 \\
11 & 2 & 13 \\
0 & 11 & 4
\end{array}\right] \]
\end{ejer}

{\it Soluci\'on:}
% Escribe tu soluci\'on para el Ejercicio 32

\begin{sageblock}
matrix(Zmod(17),[[12,9,8],
[5,10,4],
[5,15,15],
[3,11,11],
[4,2,4],
[0,15,12],
[11,2,13],
[0,11,4]])
\end{sageblock}

% Fin del Ejercicio 32


\begin{ejer} Determina si las columnas de esta matriz son vectores linealmente independientes, generadores y/o base de ${{\mathbb Z}_{43}}^{7}$,
\[ A = \left[\begin{array}{rrrrr}
8 & 40 & 42 & 1 & 5 \\
12 & 30 & 13 & 8 & 31 \\
3 & 33 & 4 & 41 & 9 \\
15 & 5 & 21 & 28 & 19 \\
0 & 36 & 18 & 33 & 19 \\
41 & 29 & 5 & 22 & 28 \\
5 & 32 & 10 & 19 & 31
\end{array}\right] \]
\end{ejer}

{\it Soluci\'on:}
% Escribe tu soluci\'on para el Ejercicio 33

\begin{sageblock}
matrix(Zmod(43),[[8,40,42,1,5],
[12,30,13,8,31],
[3,33,4,41,9],
[15,5,21,28,19],
[0,36,18,33,19],
[41,29,5,22,28],
[5,32,10,19,31]])
\end{sageblock}

% Fin del Ejercicio 33


\begin{ejer} Determina si las columnas de esta matriz son vectores linealmente independientes, generadores y/o base de ${{\mathbb R}}^{8}$,
\[ A = \left[\begin{array}{rrrr}
-5 & -4 & -1 & 2 \\
-3 & -5 & -4 & -8 \\
0 & 0 & 1 & 5 \\
-2 & -3 & -3 & -7 \\
2 & 3 & 3 & 5 \\
0 & -2 & -3 & -5 \\
2 & 2 & 1 & 1 \\
-1 & -2 & -2 & -6
\end{array}\right] \]
\end{ejer}

{\it Soluci\'on:}
% Escribe tu soluci\'on para el Ejercicio 34

\begin{sageblock}
matrix(QQ,[[-5,-4,-1,2],
[-3,-5,-4,-8],
[0,0,1,5],
[-2,-3,-3,-7],
[2,3,3,5],
[0,-2,-3,-5],
[2,2,1,1],
[-1,-2,-2,-6]])
\end{sageblock}

% Fin del Ejercicio 34


\begin{ejer} Determina si las columnas de esta matriz son vectores linealmente independientes, generadores y/o base de ${{\mathbb R}}^{8}$,
\[ A = \left[\begin{array}{rrr}
2 & -5 & -5 \\
1 & -2 & -2 \\
-4 & 7 & 8 \\
2 & -6 & -2 \\
3 & -5 & -5 \\
0 & -2 & -7 \\
3 & -8 & -5 \\
0 & -5 & -9
\end{array}\right] \]
\end{ejer}

{\it Soluci\'on:}
% Escribe tu soluci\'on para el Ejercicio 35

\begin{sageblock}
matrix(QQ,[[2,-5,-5],
[1,-2,-2],
[-4,7,8],
[2,-6,-2],
[3,-5,-5],
[0,-2,-7],
[3,-8,-5],
[0,-5,-9]])
\end{sageblock}

% Fin del Ejercicio 35


\begin{ejer} Determina si las columnas de esta matriz son vectores linealmente independientes, generadores y/o base de ${{\mathbb R}}^{7}$,
\[ A = \left[\begin{array}{rrrr}
-2 & -1 & 4 & -8 \\
1 & 0 & -2 & -5 \\
1 & 0 & -1 & 0 \\
-1 & -2 & 3 & 5 \\
-2 & 1 & 2 & 8 \\
-1 & 0 & 2 & 0 \\
0 & 2 & 1 & 8
\end{array}\right] \]
\end{ejer}

{\it Soluci\'on:}
% Escribe tu soluci\'on para el Ejercicio 36

\begin{sageblock}
matrix(QQ,[[-2,-1,4,-8],
[1,0,-2,-5],
[1,0,-1,0],
[-1,-2,3,5],
[-2,1,2,8],
[-1,0,2,0],
[0,2,1,8]])
\end{sageblock}

% Fin del Ejercicio 36


\begin{ejer} Determina si las columnas de esta matriz son vectores linealmente independientes, generadores y/o base de ${{\mathbb R}}^{5}$,
\[ A = \left[\begin{array}{rrrrrr}
-1 & -2 & -1 & 1 & -6 & 1 \\
-1 & 0 & -4 & -3 & -5 & 2 \\
-2 & -1 & -4 & -1 & -7 & -1 \\
2 & 3 & 2 & -2 & 9 & 1 \\
0 & 2 & -3 & -6 & -2 & 8
\end{array}\right] \]
\end{ejer}

{\it Soluci\'on:}
% Escribe tu soluci\'on para el Ejercicio 37

\begin{sageblock}
matrix(QQ,[[-1,-2,-1,1,-6,1],
[-1,0,-4,-3,-5,2],
[-2,-1,-4,-1,-7,-1],
[2,3,2,-2,9,1],
[0,2,-3,-6,-2,8]])
\end{sageblock}

% Fin del Ejercicio 37


\begin{ejer} Determina si las columnas de esta matriz son vectores linealmente independientes, generadores y/o base de ${{\mathbb Z}_{47}}^{5}$,
\[ A = \left[\begin{array}{rrrrr}
42 & 44 & 29 & 24 & 16 \\
1 & 20 & 42 & 44 & 39 \\
27 & 34 & 17 & 34 & 33 \\
17 & 29 & 5 & 10 & 28 \\
11 & 9 & 30 & 41 & 45
\end{array}\right] \]
\end{ejer}

{\it Soluci\'on:}
% Escribe tu soluci\'on para el Ejercicio 38

\begin{sageblock}
matrix(Zmod(47),[[42,44,29,24,16],
[1,20,42,44,39],
[27,34,17,34,33],
[17,29,5,10,28],
[11,9,30,41,45]])
\end{sageblock}

% Fin del Ejercicio 38


\begin{ejer} Determina si las columnas de esta matriz son vectores linealmente independientes, generadores y/o base de ${{\mathbb Z}_{23}}^{6}$,
\[ A = \left[\begin{array}{rrrrrr}
2 & 2 & 15 & 8 & 4 & 20 \\
15 & 8 & 6 & 8 & 18 & 14 \\
13 & 7 & 10 & 6 & 9 & 3 \\
16 & 10 & 14 & 15 & 2 & 10 \\
4 & 0 & 0 & 22 & 15 & 11 \\
22 & 12 & 7 & 2 & 18 & 1
\end{array}\right] \]
\end{ejer}

{\it Soluci\'on:}
% Escribe tu soluci\'on para el Ejercicio 39

\begin{sageblock}
matrix(Zmod(23),[[2,2,15,8,4,20],
[15,8,6,8,18,14],
[13,7,10,6,9,3],
[16,10,14,15,2,10],
[4,0,0,22,15,11],
[22,12,7,2,18,1]])
\end{sageblock}

% Fin del Ejercicio 39


\begin{ejer} Determina si las columnas de esta matriz son vectores linealmente independientes, generadores y/o base de ${{\mathbb Z}_{43}}^{8}$,
\[ A = \left[\begin{array}{rrr}
39 & 3 & 6 \\
25 & 24 & 38 \\
17 & 20 & 18 \\
12 & 1 & 28 \\
27 & 30 & 34 \\
24 & 41 & 26 \\
0 & 19 & 32 \\
5 & 15 & 35
\end{array}\right] \]
\end{ejer}

{\it Soluci\'on:}
% Escribe tu soluci\'on para el Ejercicio 40

\begin{sageblock}
matrix(Zmod(43),[[39,3,6],
[25,24,38],
[17,20,18],
[12,1,28],
[27,30,34],
[24,41,26],
[0,19,32],
[5,15,35]])
\end{sageblock}

% Fin del Ejercicio 40

\section{Formas Implícita y Paramétrica de un Espacio Vectorial}

\begin{ejer} Sea $V$ el espacio vectorial sobre los números reales 
generado por las columnas de la matriz 
\[B = \left(\begin{array}{rrr}
2 & -83 & -2 \\
2 & 3 & -1 \\
-3 & 1 & -1 \\
-3 & 1 & 3 \\
1 & 1 & -1 \\
1 & 3 & 0
\end{array}\right)\]
Escribe $V$ en forma impl\'icita 
\end{ejer}

{\it Soluci\'on:}
% Escribe tu soluci\'on para el Ejercicio 41

\begin{sageblock}
matrix(QQ,[[2,-83,-2],
[2,3,-1],
[-3,1,-1],
[-3,1,3],
[1,1,-1],
[1,3,0]])
\end{sageblock}

% Fin del Ejercicio 41

\begin{ejer} Sea $V$ el espacio vectorial sobre los números reales 
generado por las columnas de la matriz 
\[B = \left(\begin{array}{rrrr}
-1 & 0 & -5 & 1 \\
2 & 3 & -1 & -1 \\
-1 & 2 & -1 & 1 \\
2 & 0 & -1 & -3 \\
2 & -1 & 0 & -2 \\
3 & -1 & -4 & -8 \\
-7 & 1 & -33 & -1 \\
-8 & 8 & 19 & 0
\end{array}\right)\]
Escribe $V$ en forma impl\'icita 
\end{ejer}

{\it Soluci\'on:}
% Escribe tu soluci\'on para el Ejercicio 42

\begin{sageblock}
matrix(QQ,[[-1,0,-5,1],
[2,3,-1,-1],
[-1,2,-1,1],
[2,0,-1,-3],
[2,-1,0,-2],
[3,-1,-4,-8],
[-7,1,-33,-1],
[-8,8,19,0]])
\end{sageblock}

% Fin del Ejercicio 42


\begin{ejer} Sea $V$ el espacio vectorial sobre los números reales 
generado por las columnas de la matriz 
\[B = \left(\begin{array}{rrr}
1 & 1 & 0 \\
-2 & 1 & -1 \\
-3 & 3 & -2 \\
4 & -1 & 0 \\
0 & 1 & -235857
\end{array}\right)\]
Escribe $V$ en forma impl\'icita 
\end{ejer}

{\it Soluci\'on:}
% Escribe tu soluci\'on para el Ejercicio 43

\begin{sageblock}
matrix(QQ,[[1,1,0],
[-2,1,-1],
[-3,3,-2],
[4,-1,0],
[0,1,-235857]])
\end{sageblock}

% Fin del Ejercicio 43


\begin{ejer} Sea $V$ el espacio vectorial sobre los números reales 
generado por las columnas de la matriz 
\[B = \left(\begin{array}{rrrrr}
0 & 3 & 2 & -2 & 40 \\
-5 & 0 & -4 & 1 & 1 \\
-2 & -1 & 1 & 1 & 0 \\
-2 & 0 & -1 & 5 & 1 \\
8 & -3 & 2 & 0 & 1 \\
0 & -5 & 0 & 1 & 1 \\
-1 & 1 & 1 & 0 & -1 \\
1 & -1 & -1 & 0 & -9
\end{array}\right)\]
Escribe $V$ en forma impl\'icita 
\end{ejer}

{\it Soluci\'on:}
% Escribe tu soluci\'on para el Ejercicio 44

\begin{sageblock}
matrix(QQ,[[0,3,2,-2,40],
[-5,0,-4,1,1],
[-2,-1,1,1,0],
[-2,0,-1,5,1],
[8,-3,2,0,1],
[0,-5,0,1,1],
[-1,1,1,0,-1],
[1,-1,-1,0,-9]])
\end{sageblock}

% Fin del Ejercicio 44


\begin{ejer} Sea $V$ el espacio vectorial sobre los números reales 
generado por las columnas de la matriz 
\[B = \left(\begin{array}{rrr}
2 & 6 & 8 \\
0 & 3 & 1 \\
0 & 2 & 1 \\
0 & -1 & 1 \\
-1 & 0 & 2 \\
27 & 14 & 3
\end{array}\right)\]
Escribe $V$ en forma impl\'icita 
\end{ejer}

{\it Soluci\'on:}
% Escribe tu soluci\'on para el Ejercicio 45

\begin{sageblock}
matrix(QQ,[[2,6,8],
[0,3,1],
[0,2,1],
[0,-1,1],
[-1,0,2],
[27,14,3]])
\end{sageblock}

% Fin del Ejercicio 45


\begin{ejer} Sea $V$ el espacio vectorial sobre los números reales 
generado por las columnas de la matriz 
\[B = \left(\begin{array}{rrrr}
0 & 293 & 1 & -3 \\
-1 & -1 & 1 & -1 \\
0 & 0 & 1 & 2 \\
-2 & 3 & -2 & 5 \\
-1 & 0 & 10 & -1 \\
-1 & -18 & 0 & 1 \\
2 & -1 & 1 & 1 \\
1 & -4 & 0 & 26
\end{array}\right)\]
Escribe $V$ en forma impl\'icita 
\end{ejer}

{\it Soluci\'on:}
% Escribe tu soluci\'on para el Ejercicio 46

\begin{sageblock}
matrix(QQ,[[0,293,1,-3],
[-1,-1,1,-1],
[0,0,1,2],
[-2,3,-2,5],
[-1,0,10,-1],
[-1,-18,0,1],
[2,-1,1,1],
[1,-4,0,26]])
\end{sageblock}

% Fin del Ejercicio 46


\begin{ejer} Sea $V$ el espacio vectorial sobre los números reales 
generado por las columnas de la matriz 
\[B = \left(\begin{array}{rrrr}
-3 & 5 & -1 & -1 \\
1 & -1 & -2 & -4 \\
-1 & -1 & -7 & -3 \\
1 & -1 & -1 & -6 \\
-1 & 13 & 3 & 2 \\
1 & 1 & 7 & 0 \\
0 & -2 & -2 & -1 \\
-2 & 2 & 0 & 2
\end{array}\right)\]
Escribe $V$ en forma impl\'icita 
\end{ejer}

{\it Soluci\'on:}
% Escribe tu soluci\'on para el Ejercicio 47

\begin{sageblock}
matrix(QQ,[[-3,5,-1,-1],
[1,-1,-2,-4],
[-1,-1,-7,-3],
[1,-1,-1,-6],
[-1,13,3,2],
[1,1,7,0],
[0,-2,-2,-1],
[-2,2,0,2]])
\end{sageblock}

% Fin del Ejercicio 47


\begin{ejer} Sea $V$ el espacio vectorial sobre los números reales 
generado por las columnas de la matriz 
\[B = \left(\begin{array}{rrr}
0 & -2 & 0 \\
1 & -2 & -1 \\
75 & -3 & 1 \\
0 & -1 & 15 \\
0 & 0 & 1 \\
0 & 0 & 1 \\
1 & 1 & 0
\end{array}\right)\]
Escribe $V$ en forma impl\'icita 
\end{ejer}

{\it Soluci\'on:}
% Escribe tu soluci\'on para el Ejercicio 48

\begin{sageblock}
matrix(QQ,[[0,-2,0],
[1,-2,-1],
[75,-3,1],
[0,-1,15],
[0,0,1],
[0,0,1],
[1,1,0]])
\end{sageblock}

% Fin del Ejercicio 48


\begin{ejer} Sea $V$ el espacio vectorial sobre los números reales 
generado por las columnas de la matriz 
\[B = \left(\begin{array}{rrrr}
0 & 1 & 3 & 0 \\
0 & 0 & -2 & 0 \\
0 & -6 & 1 & -2 \\
1 & 5 & -1 & 0 \\
-1 & 1 & -1 & -2 \\
-10 & 0 & 0 & 0 \\
19 & -1 & -2 & -2 \\
1 & -11 & 1 & 1
\end{array}\right)\]
Escribe $V$ en forma impl\'icita 
\end{ejer}

{\it Soluci\'on:}
% Escribe tu soluci\'on para el Ejercicio 49

\begin{sageblock}
matrix(QQ,[[0,1,3,0],
[0,0,-2,0],
[0,-6,1,-2],
[1,5,-1,0],
[-1,1,-1,-2],
[-10,0,0,0],
[19,-1,-2,-2],
[1,-11,1,1]])
\end{sageblock}

% Fin del Ejercicio 49


\begin{ejer} Sea $V$ el espacio vectorial sobre los números reales 
generado por las columnas de la matriz 
\[B = \left(\begin{array}{rrr}
0 & 0 & -2 \\
2 & 0 & -1 \\
2 & 3 & -1 \\
0 & -6 & 1 \\
0 & -3 & 0 \\
0 & 2 & -1
\end{array}\right)\]
Escribe $V$ en forma impl\'icita 
\end{ejer}

{\it Soluci\'on:}
% Escribe tu soluci\'on para el Ejercicio 50

\begin{sageblock}
matrix(QQ,[[0,0,-2],
[2,0,-1],
[2,3,-1],
[0,-6,1],
[0,-3,0],
[0,2,-1]])
\end{sageblock}

% Fin del Ejercicio 50


\begin{ejer} Sea $W$ el espacio vectorial sobre los números reales dado
como anulador por la derecha de la matriz 
\[A = \left(\begin{array}{rrrrr}
-6 & 0 & -1 & 0 & -2 \\
0 & -2 & 1 & 2 & 0 \\
1 & 0 & 0 & 0 & -3
\end{array}\right),\]
es decir, $W = N(A)$. Escribe $W$ en forma param\'etrica.
\end{ejer}

{\it Soluci\'on:}
% Escribe tu soluci\'on para el Ejercicio 51

\begin{sageblock}
matrix(QQ,[[-6,0,-1,0,-2],
[0,-2,1,2,0],
[1,0,0,0,-3]])
\end{sageblock}

% Fin del Ejercicio 51


\begin{ejer} Sea $W$ el espacio vectorial sobre los números reales dado
como anulador por la derecha de la matriz 
\[A = \left(\begin{array}{rrrrrrrr}
-1 & 0 & 1 & 0 & 0 & 0 & 0 & 6 \\
20 & 2 & 1 & 0 & 0 & -1 & 0 & -1 \\
0 & 1 & -2 & 1 & -2 & 1 & 0 & 0 \\
-1 & -2 & -1 & -1 & -43 & -2 & 1 & -1 \\
10 & 62 & 1 & 1 & 0 & -9 & 1 & 0
\end{array}\right),\]
es decir, $W = N(A)$. Escribe $W$ en forma param\'etrica.
\end{ejer}

{\it Soluci\'on:}
% Escribe tu soluci\'on para el Ejercicio 52

\begin{sageblock}
A = matrix(QQ,[[-1,0,1,0,0,0,0,6],
[20,2,1,0,0,-1,0,-1],
[0,1,-2,1,-2,1,0,0],
[-1,-2,-1,-1,-43,-2,1,-1],
[10,62,1,1,0,-9,1,0]])
\end{sageblock}

% Fin del Ejercicio 52


\begin{ejer} Sea $W$ el espacio vectorial sobre los números reales dado
como anulador por la derecha de la matriz 
\[A = \left(\begin{array}{rrrrrr}
-2 & -1 & 1 & -2 & 3 & 1 \\
21 & 1 & 2 & 1 & 1 & 26 \\
1 & 1 & 1 & 5 & 3 & -1 \\
-1 & -8 & 0 & 3 & 0 & 7
\end{array}\right),\]
es decir, $W = N(A)$. Escribe $W$ en forma param\'etrica.
\end{ejer}

{\it Soluci\'on:}
% Escribe tu soluci\'on para el Ejercicio 53

\begin{sageblock}
matrix(QQ,[[-2,-1,1,-2,3,1],
[21,1,2,1,1,26],
[1,1,1,5,3,-1],
[-1,-8,0,3,0,7]])
\end{sageblock}

% Fin del Ejercicio 53


\begin{ejer} Sea $W$ el espacio vectorial sobre los números reales dado
como anulador por la derecha de la matriz 
\[A = \left(\begin{array}{rrrrrrrrr}
-1 & -197 & 0 & 1 & 9 & 0 & 0 & 0 & -86 \\
-1 & -1 & 1 & -1 & -1 & -1 & 1 & 1 & 1 \\
-3 & -8 & 1 & 1 & -1 & -1 & 1 & 0 & -1 \\
4 & 0 & -10 & -1 & 0 & 1 & 1 & 7 & 5 \\
0 & 1 & 0 & -4 & 2 & 0 & -2 & 0 & -4
\end{array}\right),\]
es decir, $W = N(A)$. Escribe $W$ en forma param\'etrica.
\end{ejer}

{\it Soluci\'on:}
% Escribe tu soluci\'on para el Ejercicio 54

\begin{sageblock}
matrix(QQ,[[-1,-197,0,1,9,0,0,0,-86],
[-1,-1,1,-1,-1,-1,1,1,1],
[-3,-8,1,1,-1,-1,1,0,-1],
[4,0,-10,-1,0,1,1,7,5],
[0,1,0,-4,2,0,-2,0,-4]])
\end{sageblock}

% Fin del Ejercicio 54


\begin{ejer} Sea $W$ el espacio vectorial sobre los números reales dado
como anulador por la derecha de la matriz 
\[A = \left(\begin{array}{rrrrrrrrr}
-1 & 0 & 0 & -3 & -1 & -1 & 137 & 1 & 3 \\
-2 & -3 & -61 & -2 & 182 & -3 & 4 & 0 & 0 \\
1 & 1 & 2 & 1 & -1 & 1 & 1 & -13 & -1 \\
13 & 0 & -7 & 0 & 0 & 1 & -6 & -7 & 11 \\
2 & 10 & 1 & 2 & -1 & 1 & 7 & -5 & -1
\end{array}\right),\]
es decir, $W = N(A)$. Escribe $W$ en forma param\'etrica.
\end{ejer}

{\it Soluci\'on:}
% Escribe tu soluci\'on para el Ejercicio 55

\begin{sageblock}
matrix(QQ,[[-1,0,0,-3,-1,-1,137,1,3],
[-2,-3,-61,-2,182,-3,4,0,0],
[1,1,2,1,-1,1,1,-13,-1],
[13,0,-7,0,0,1,-6,-7,11],
[2,10,1,2,-1,1,7,-5,-1]])
\end{sageblock}

% Fin del Ejercicio 55


\begin{ejer} Sea $W$ el espacio vectorial sobre los números reales dado
como anulador por la derecha de la matriz 
\[A = \left(\begin{array}{rrrrrrrrr}
0 & 0 & 1 & -1 & 14 & -2 & 0 & -1 & -1 \\
-1 & -1 & -1 & 0 & 1 & -1 & -1 & 1 & -5 \\
1 & -11 & 28 & -3 & 2 & 1 & 1 & 1 & 1 \\
-1 & 1 & -1 & -7 & -12 & -1 & -9 & 1 & 1 \\
61 & 0 & 0 & 11 & -2 & -1 & 1 & -1 & -1
\end{array}\right),\]
es decir, $W = N(A)$. Escribe $W$ en forma param\'etrica.
\end{ejer}

{\it Soluci\'on:}
% Escribe tu soluci\'on para el Ejercicio 56

\begin{sageblock}
matrix(QQ,[[0,0,1,-1,14,-2,0,-1,-1],
[-1,-1,-1,0,1,-1,-1,1,-5],
[1,-11,28,-3,2,1,1,1,1],
[-1,1,-1,-7,-12,-1,-9,1,1],
[61,0,0,11,-2,-1,1,-1,-1]])
\end{sageblock}

% Fin del Ejercicio 56


\begin{ejer} Sea $W$ el espacio vectorial sobre los números reales dado
como anulador por la derecha de la matriz 
\[A = \left(\begin{array}{rrrrrrr}
-1 & 1 & -1 & -1 & -2 & 4 & 7 \\
0 & 0 & -3 & -13 & -84 & 2 & -1 \\
1 & 1 & 0 & 0 & 1 & -18 & 5
\end{array}\right),\]
es decir, $W = N(A)$. Escribe $W$ en forma param\'etrica.
\end{ejer}

{\it Soluci\'on:}
% Escribe tu soluci\'on para el Ejercicio 57

\begin{sageblock}
matrix(QQ,[[-1,1,-1,-1,-2,4,7],
[0,0,-3,-13,-84,2,-1],
[1,1,0,0,1,-18,5]])
\end{sageblock}

% Fin del Ejercicio 57


\begin{ejer} Sea $W$ el espacio vectorial sobre los números reales dado
como anulador por la derecha de la matriz 
\[A = \left(\begin{array}{rrrrrrrrr}
1 & -19 & -1 & -1 & -2 & 1 & -2 & 2 & 8 \\
-1 & -1 & 0 & -2 & 1 & 0 & -1 & 1 & -4 \\
3 & 64 & 0 & 2 & 1 & 0 & 1 & 0 & 0 \\
-1 & 4 & 1 & 2 & 2 & 2 & 1 & -1 & -5 \\
-2 & 8 & 1 & 1 & 0 & 0 & -47 & 2 & -1
\end{array}\right),\]
es decir, $W = N(A)$. Escribe $W$ en forma param\'etrica.
\end{ejer}

{\it Soluci\'on:}
% Escribe tu soluci\'on para el Ejercicio 58

\begin{sageblock}
matrix(QQ,[[1,-19,-1,-1,-2,1,-2,2,8],
[-1,-1,0,-2,1,0,-1,1,-4],
[3,64,0,2,1,0,1,0,0],
[-1,4,1,2,2,2,1,-1,-5],
[-2,8,1,1,0,0,-47,2,-1]])
\end{sageblock}

% Fin del Ejercicio 58


\begin{ejer} Sea $W$ el espacio vectorial sobre los números reales dado
como anulador por la derecha de la matriz 
\[A = \left(\begin{array}{rrrrrrr}
0 & 0 & -1 & 1 & -4 & 1 & 2 \\
6 & 0 & -8 & 1 & 0 & -68 & -1 \\
-3 & 0 & -1 & 2 & 1 & 2 & 2 \\
0 & 1 & -1 & -1 & -1 & -1 & 1
\end{array}\right),\]
es decir, $W = N(A)$. Escribe $W$ en forma param\'etrica.
\end{ejer}

{\it Soluci\'on:}
% Escribe tu soluci\'on para el Ejercicio 59

\begin{sageblock}
matrix(QQ,[[0,0,-1,1,-4,1,2],
[6,0,-8,1,0,-68,-1],
[-3,0,-1,2,1,2,2],
[0,1,-1,-1,-1,-1,1]])
\end{sageblock}

% Fin del Ejercicio 59


\begin{ejer} Sea $W$ el espacio vectorial sobre los números reales dado
como anulador por la derecha de la matriz 
\[A = \left(\begin{array}{rrrrrrr}
1 & -1 & 2 & 2 & 10 & -5 & -1 \\
-1 & -8 & 2 & -7 & -17 & -2 & 0 \\
4 & 3 & -1 & -11 & 1 & 1 & -1 \\
1 & 4 & 1 & 1 & -1 & 2 & 1 \\
1 & -2 & -1 & 2 & 1 & 5 & -2
\end{array}\right),\]
es decir, $W = N(A)$. Escribe $W$ en forma param\'etrica.
\end{ejer}

{\it Soluci\'on:}
% Escribe tu soluci\'on para el Ejercicio 60

\begin{sageblock}
matrix(QQ,[[1,-1,2,2,10,-5,-1],
[-1,-8,2,-7,-17,-2,0],
[4,3,-1,-11,1,1,-1],
[1,4,1,1,-1,2,1],
[1,-2,-1,2,1,5,-2]])
\end{sageblock}

% Fin del Ejercicio 60


\end{document}
