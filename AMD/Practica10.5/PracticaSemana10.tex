\documentclass{amsart}
\usepackage[utf8]{inputenc}
\usepackage[usefamily=sage]{pythontex} 
\usepackage{float}
\usepackage{tikz}
\usetikzlibrary{calc}
\usepackage[most]{tcolorbox}
\usepackage[margin = 2cm]{geometry}
\usepackage{graphicx}
\usetikzlibrary{3d}
\usepackage{tikz-3dplot}


\newtheorem{ejer}{Ejercicio}

\def\n{\mathbb{N}}
\def\r{\mathbb{R}}
\def\z{\mathbb{Z}}
\def\q{\mathbb{Q}}
\def\c{\mathbb{C}}

\title{AMD - Práctica de Producto Escalar}

\begin{document}

\maketitle

\begin{sagecode}
latex.matrix_delimiters("[", "]")
\end{sagecode}

\section{Ortogonalidad en ${\mathbb R}^2$ y ${\mathbb R}^3$}

\begin{ejer}
Para cada una de las rectas dadas, calcula un vector $v_1$ de la recta y un vector $v_2$ perpendicular
a la recta de forma que la base $B = \{v_1,v_2\}$ tenga orientación positiva, normalízalos y pintalos 
en unos ejes coordenados y dibuja un cuadrado de tamaño $4\times 4$ centrado en el origen con lados 
paralelos a las base calculada.

\begin{enumerate}
\item La recta $r$ dada por $2x+y = 0$, es decir $r = N([2\ 1])$.
\item La recta $r$ dada por $x = 2\lambda, y = -\lambda$ con $\lambda \in {\mathbb R}$, es decir
$r = C\left(\left[\begin{array}{c} 2 \\ -1 \end{array} \right]\right)$.
\end{enumerate}

\end{ejer}

{\it Solución:}

% Inicio Ejercicio 1

1.

\begin{sageblock}
n = vector(RR, [2, 1])
v = vector(RR, [-n[1], n[0]])
v1 = v.normalized()
v2 = n.normalized()
B = column_matrix([v1, v2])

P1 = B * vector(RR, [-2, -2])
P2 = B * vector(RR, [2, -2])
P3 = B * vector(RR, [2, 2])
P4 = B * vector(RR, [-2, 2])
\end{sageblock}

\begin{sagesub}
\begin{center}
\begin{tikzpicture}[scale = 1.25,
cara/.style={thick, color = blue, fill opacity = 0.3, fill = blue!20}]
\draw[->,thick,gray] (-3,0) -- (3,0); % Eje X
\draw[->,thick,gray] (0,-3) -- (0,3); % Eje Y
\draw[->,very thick] (0,0) -- !{v1}; 
\draw[->,very thick] (0,0) -- !{v2};
\node at !{1.2*v1} {$u_1$};
\node at !{1.2*v2} {$u_2$};
\draw[cara] !{P1} -- !{P2} -- !{P3} -- !{P4} -- cycle;
\end{tikzpicture}
\end{center}
\end{sagesub}

\begin{sageblock}
v = vector(RR, [2, -1])
n = vector(RR, [-v[1], v[0]])
v1 = v.normalized()
v2 = n.normalized()
B = column_matrix([v1, v2])

P1 = B * vector(RR, [-2, -2])
P2 = B * vector(RR, [2, -2])
P3 = B * vector(RR, [2, 2])
P4 = B * vector(RR, [-2, 2])
\end{sageblock}

\begin{sagesub}
\begin{center}
\begin{tikzpicture}[scale = 1.25,
cara/.style={thick, color = blue, fill opacity = 0.3, fill = blue!20}]
\draw[->,thick,gray] (-3,0) -- (3,0); % Eje X
\draw[->,thick,gray] (0,-3) -- (0,3); % Eje Y
\draw[->,very thick] (0,0) -- !{v1}; 
\draw[->,very thick] (0,0) -- !{v2};
\node at !{1.2*v1} {$u_1$};
\node at !{1.2*v2} {$u_2$};
\draw[cara] !{P1} -- !{P2} -- !{P3} -- !{P4} -- cycle;
\end{tikzpicture}
\end{center}
\end{sagesub}
% Fin del Ejercicio 1

\begin{ejer}
Para cada una de las rectas dadas, calcula un vector $v_1$ de la recta y un vector $v_2$ perpendicular
a la recta de forma que la base $B = \{v_1,v_2\}$ tenga orientación positiva, normalízalos y pintalos 
en unos ejes coordenados y dibuja un cuadrado de tamaño $4\times 4$ centrado en el origen con lados 
paralelos a las base calculada.

\begin{enumerate}
\item La recta $r$ dada por $3x-y = 0$.
\item La recta $r$ dada por $x = \lambda, y = 2\lambda$ con $\lambda \in {\mathbb R}$.
\item La recta $r$ dada por $x+2y = 0$.
\item La recta $r$ dada por $x = 2\lambda, y = -3\lambda$ con $\lambda \in {\mathbb R}$.
\item La recta $r$ dada por $x-y = 0$.
\item La recta $r$ dada por $x = -\lambda, y = \lambda$ con $\lambda \in {\mathbb R}$.
\item La recta $r$ dada por $-x+2y = 0$.
\item La recta $r$ dada por $x = -2\lambda, y = \lambda$ con $\lambda \in {\mathbb R}$.
\end{enumerate}
\end{ejer}

% Inicio Ejercicio 2

% Fin del Ejercicio 2

\begin{ejer}
Para cada una de las rectas dadas, calcula un vector $v_1$ de la recta y dos vectores 
$v_2$ y $v_3$ perpendiculares a la recta de forma que la base $B = \{v_1,v_2,v_3\}$ tenga 
orientación positiva, normalízalos, píntalos en unos ejes coordenados y dibuja un cuadrado de 
tamaño $4\times 4$ centrado en el origen con lados paralelos a los vectores $v_2$ y $v_3$.

\begin{enumerate}
\item La recta $r$ dada por 
$
\begin{cases}
2x+y+z = 0 \\
x-y+z = 0
\end{cases}
$ es decir $r = N\left(\left[\begin{array}{ccc} 2 & 1 & 1 \\ 1 & -1 & 1 \end{array} \right]\right)$.
\item La recta $r$ dada por $x = 2\lambda, y = -\lambda, z = \lambda$ 
con $\lambda \in {\mathbb R}$, es decir
$r = C\left(\left[\begin{array}{c} 2 \\ -1 \\ 1 \end{array} \right]\right)$.
\end{enumerate}
\end{ejer}

{\it Solución}

% Inicio Ejercicio 3

\begin{sageblock}
n1 = vector(RR, [2, 1, 1])
n2 = vector(RR, [1, -1, 1])
v = n1.cross_product(n2)
v1 = v.normalized()
v2 = n1.normalized()
v3 = v.cross_product(v2).normalized()
B = column_matrix([v1, v2, v3])

P1 = B * vector(RR, [0, -2, -2])
P2 = B * vector(RR, [0, 2, -2])
P3 = B * vector(RR, [0, 2, 2])
P4 = B * vector(RR, [0, -2, 2])
\end{sageblock}

\begin{sagesub}
\begin{center}
\begin{tikzpicture}[scale = 1.25,
cara/.style={thick, color = blue, fill opacity = 0.3, fill = blue!20}]
\draw[->,thick,gray] (-3,0) -- (3,0); % Eje X
\draw[->,thick,gray] (0,-3) -- (0,3); % Eje Y
\draw[->,very thick] (0,0) -- !{v1}; 
\draw[->,very thick] (0,0) -- !{v2};
\draw[->,very thick] (0,0) -- !{v3};
\node at !{1.2*v1} {$v_1$};
\node at !{1.2*v2} {$v_2$};
\node at !{1.2*v3} {$v_3$};
\draw[cara] !{P1} -- !{P2} -- !{P3} -- !{P4} -- cycle;
\end{tikzpicture}
\end{center}
\end{sagesub}

\begin{sageblock}
v = vector(RR, [2, -1, 1])
n1 = vector(RR, [-v[1], v[0], 0])
n2 = v.cross_product(n1)
v1 = v.normalized()
v2 = n1.normalized()
v3 = n2.normalized()
B = column_matrix([v1, v2, v3])

P1 = B * vector(RR, [0, -2, -2])
P2 = B * vector(RR, [0, 2, -2])
P3 = B * vector(RR, [0, 2, 2])
P4 = B * vector(RR, [0, -2, 2])
\end{sageblock}

\begin{sagesub}
\begin{center}
\begin{tikzpicture}[scale = 1.25,
cara/.style={thick, color = blue, fill opacity = 0.3, fill = blue!20}]
\draw[->,thick,gray] (-3,0) -- (3,0); % Eje X
\draw[->,thick,gray] (0,-3) -- (0,3); % Eje Y
\draw[->,very thick] (0,0) -- !{v1}; 
\draw[->,very thick] (0,0) -- !{v2};
\draw[->,very thick] (0,0) -- !{v3};
\node at !{1.2*v1} {$v_1$};
\node at !{1.2*v2} {$v_2$};
\node at !{1.2*v3} {$v_3$};
\draw[cara] !{P1} -- !{P2} -- !{P3} -- !{P4} -- cycle;
\end{tikzpicture}
\end{center}
\end{sagesub}


% Fin del Ejercicio 3

\begin{ejer}
Para cada una de las rectas dadas, calcula un vector $v_1$ de la recta y dos vectores 
$v_2$ y $v_3$ perpendiculares a la recta de forma que la base $B = \{v_1,v_2,v_3\}$ tenga 
orientación positiva, normalízalos, píntalos en unos ejes coordenados y dibuja un cuadrado de 
tamaño $4\times 4$ centrado en el origen con lados paralelos a los vectores $v_2$ y $v_3$.

\begin{enumerate}
\item La recta $r$ dada por $\begin{cases} x-y+z = 0 \\ -x-3y+2z = 0 \end{cases} $
\item La recta $r$ dada por $x = \lambda, y = \lambda, z = 0$ con $\lambda \in {\mathbb R}$.
\item La recta $r$ dada por $\begin{cases} x+z = 0 \\ -y+z = 0 \end{cases} $
\item La recta $r$ dada por $x = 2\lambda, y = -3\lambda, z = -\lambda$ con $\lambda \in {\mathbb R}$.
\item La recta $r$ dada por $\begin{cases} -x-y+z = 0 \\ x+y = 0 \end{cases} $
\item La recta $r$ dada por $x = -\lambda, y = \lambda, z = 2\lambda$ con $\lambda \in {\mathbb R}$.
\item La recta $r$ dada por $\begin{cases} 2x+2y-z = 0 \\ x-y+3z = 0 \end{cases} $
\item La recta $r$ dada por $x = 2\lambda, y = 0, z = -\lambda$ con $\lambda \in {\mathbb R}$.
\end{enumerate}
\end{ejer}

{\it Solución}

% Inicio Ejercicio 4

% Fin del Ejercicio 4

\section{Proyección Ortogonal}

\begin{ejer}
Dado el espacio $W = N(H)$ y el vector $v$, donde

\begin{sageblock}
H=matrix(QQ,[[1,-1,0,0],[0,0,1,0]])
v=column_matrix(QQ,[1,2,0,2])
\end{sageblock}

\[ H = \sage{H} \quad v = \sage{v}, \]
expresar el vector $v$ como suma de un vector de $W$ y otro vector de $W^\perp $.
\end{ejer}

% Inicio Ejercicio 5
{\it Solución 1:}

Nos piden que expresemos $v = v_1 + v_2$ con $v_1\in W$ y $v_2\in W^\perp $.
Como $W = N(H)$ entonces tenemos los vectores que generan $W^\perp$ porque 
sabemos que $W^{\perp} = C(H^T)$. Si llamamos $A = H^T = \sage{H.T}$,   
usando ese conjunto generador, podemos calcular directamente $v_2$, que es la 
proyección $v$ sobre el espacio $W^{\perp}$ con la fórmula de la proyección y 
luego despejar $v_1$ a partir de la fórmula $v = v_1+v_2$.

\begin{sageblock}
A=H.T
v2=A*(A.T*A)^-1*A.T*v
\end{sageblock}

$$
v_2 = A\cdot(A^T\cdot A)^{-1}\cdot A^T\cdot v = \sage{v2} \quad 
v_1 = v-v_2 = \sage{v}-\sage{v2} = \sage{v-v2}.
$$

{\it Solución 2:}

Nos piden que hagamos $v = v_1 + v_2$ donde $v_1\in W$ y $v_2\in W^{\perp}$. 
Obtenemos una base de $W$ y 
calculamos en primer lugar la proyección de $v$ sobre $W$. 

\begin{sageblock}
M=block_matrix(1,2,[H.T,1])
R=copy(M.echelon_form())
R.subdivide(2,2)
D = R.subdivision(1,1).T
v1 = D*(D.T*D)^-1*D.T*v 
\end{sageblock}

Como $W = N(H)$ la base la obtenemos reduciendo por filas la matriz $H$ ampliada con la identidad:
$$ \sage{M} \to \sage{R} $$
La base de $W$ es $D= \sage{D}$ y entonces podemos calcular $v_1$ con la 
fórmula de la proyección sobre $W$ y despejar $v_2$ de la igualdad $v = v_1+v_2$: 
$$
v_1 = D\cdot (D^T\cdot D)^{-1}\cdot D^T\cdot v = \sage{v1} \qquad 
v_2 = v-v_1 = \sage{v}-\sage{v1} = \sage{v-v1}.
$$

% Fin Ejercicio 5

\section{Método de Gram-Schmidt}

\begin{ejer}
Dado el espacio $W = N(H) \leq \r^4$, donde

\begin{sageblock}
H=matrix(QQ,[[2,1,0,2],[1,-1,1,0],[2,0,1,-1]])
\end{sageblock}

\[ H = \sage{H}.\] 

Determinar una base ortogonal del espacio $W^\perp$.
\end{ejer}
{\it Solución:}

% Inicio Ejercicio 6

Como $W = N (H)$ entonces $W^\perp = C(H^T)$, las columnas de $H^T$ son generadoras de $W^\perp$. 
Reduciendo la matriz también vemos que son linealmente independientes por lo que son una base $\{u_1,u_2,u_3\}$
a la que le aplicaremos Gram-Schmidt. 

$$ H^T = \sage{H.T} \to \sage{H.T.echelon_form()} $$

\begin{sageblock}
u1,u2,u3 = H.T.columns() # Nos las da en formato vector y ahí podemos usar el producto escalar.
v1=u1
v2=u2-((u2*v1)/(v1*v1))*v1
v3=u3-((u3*v1)/(v1*v1))*v1-((u3*v2)/(v2*v2))*v2
B=column_matrix([v1,v2,v3])
\end{sageblock}

El método de Gram-Schmidt nos dice que la base ortonormal es 
\begin{align*}
v_1 &= u_1 \\
v_2 &= u_2-\frac{u_2\cdot v_1}{v_1\cdot v_1} \ v_1 \\
v_3 &= u_3-\frac{u_3\cdot v_1}{v_1\cdot v_1} \ v_1 - \frac{u_3\cdot v_2}{v_2\cdot v_2} \ v_2 \\
\end{align*}

$$ v_1=\sage{B[:,0]} \qquad v_2=\sage{B[:,1]} \qquad v_3=\sage{B[:,2]} $$

% Fin Ejercicio 6

\vspace{0.5cm}
\begin{ejer} Calcula una base ortonormal del subespacio $U=C(B) \leq \r^5$ y la proyección ortogonal sobre $U$ del vector 
$v\in \r^5$ siendo
$$
    B=
    \left[
    \begin{array}{rrr}
    2 & 1 & 1 \\
    1 & 1 & -1 \\
    1 & 1 & 0 \\
    1 & 3 & 1 \\
    0 & 1 & 0
    \end{array}
    \right],
    \qquad\qquad
    v=
    \left[
    \begin{array}{r}
    1 \\
    0 \\
    1 \\
    -1 \\
    1
    \end{array}
    \right].
$$
\end{ejer}
{\it Solución:}

% Inicio Ejercicio 7

\begin{sageblock}
B=matrix(QQ,[[2,1,1],[1,1,-1],[1,1,0],[1,3,1],[0,1,0]])
v=vector([1,0,1,-1,1])
R=B.echelon_form()
\end{sageblock}

En primer lugar, comprobamos que las columnas de $B$ son linealmente independientes
(y, por tanto, una base de $U=C(B)$). Para ello calculamos la reducida completa por filas de~$B$, que es:

$$ \sage{R}.$$

Como hay $3$ pivotes, el rango de $B$ es $3$ y las columnas son independientes.

\begin{sageblock}
v1=B.column(0)
v2=B.column(1)
v3=B.column(2)
u1=v1
u2=v2-((v2*u1)/(u1*u1))*u1
u3=v3-((v3*u1)/(u1*u1))*u1-((v3*u2)/(u2*u2))*u2
w1=u1/norm(u1)
w2=u2/norm(u2)
w3=u3/norm(u3)
B1=column_matrix([u1,u2,u3])
B2=column_matrix([w1,w2,w3])
\end{sageblock}

Aplicamos el método de Gram-Schmidt a los vectores

$$ v_1=\sage{B[:,0]} \qquad v_2=\sage{B[:,1]} \qquad v_3=\sage{B[:,2]} $$


Obteniendo una base ortogonal formada por las columnas de la matriz

$$ B_1 = \sage{B1} $$
	
Una vez normalizada se obtiene la base ortonormal dada por las columnas de la matriz:

$$ B_2 = \sage{B2} $$

Como mera comprobación $B_2$ es una matriz ortogonal:

$$ B_2^T\cdot B_2 = \sage{B2.T*B2}.$$

La proyección de $v$ sobre $U$ es

$$ {\rm proy}_U(v) = B\cdot ((B^T\cdot B)^{-1}\cdot B^T)\cdot v = \sage{B*(B.T*B)^-1*B.T*column_matrix([v])}$$

Si utilizamos la base ortonormal $B_2$ en vez de la base $B$, de manera alternativa también se puede calcular la proyección así:

$$
	{\rm proy}_U(v) = B_2\cdot B_2^T\cdot v = (v*w_1)w_1+(v*w_2)w_2+(v*w_3)w_3 = \sage{B2*B2.T*column_matrix([v])}.
$$

% Fin Ejercicio 7

\vspace{0.5cm}
\begin{ejer}
Calcula una base ortonormal del subespacio $U=N(H) \leq \mathbb{R}^6$ y la proyección ortogonal sobre $U$ del vector $v\in \mathbb{R}^6$ siendo
$$
    H=
    \left[
    \begin{array}{rrrrrr}
    1 & 0 & 0 & 0 & 1 & 1 \\
    0 & 1 & -1 & 1 & 0 & 0
    \end{array}
    \right],
    \qquad\qquad
    v=
    \left[
    \begin{array}{r}
    1 \\
    2 \\
    0 \\
    1 \\
    0 \\
    -1 
    \end{array}
    \right].
$$
\end{ejer}

{\it Solución:}

% Inicio Ejercicio 8

\begin{sageblock}
H=matrix(QQ,[[1,0,0,0,1,1],[0,1,-1,1,0,0]])
HTI=block_matrix(1,2,[H.T,1])
R=HTI.echelon_form()
R=copy(R)
R.subdivide(2,2)
B=R.subdivision(1,1).T
\end{sageblock}

Debemos obtener una base del espacio $U = N(H)$, para ello obtenemos unas ecuaciones paramétricas de $U$. Ampliamos la traspuesta de $H$ con la matriz identidad

$$ [H^T | I] = \sage{HTI} $$

Reducimos por filas y obtenemos

$$ \sage{R} $$

Una base del espacio $U$ son las columnas de la matriz

$$ B = \sage{B} $$

\begin{sageblock}
v1=B.column(0)
v2=B.column(1)
v3=B.column(2)
v4=B.column(3)
u1=v1
u2=v2-((v2*u1)/(u1*u1))*u1
u3=v3-((v3*u1)/(u1*u1))*u1-((v3*u2)/(u2*u2))*u2
u4=v4-((v4*u1)/(u1*u1))*u1-((v4*u2)/(u2*u2))*u2-((v4*u3)/(u3*u3))*u3
w1=u1/norm(u1)
w2=u2/norm(u2)
w3=u3/norm(u3)
w4=u4/norm(u4)
B1=column_matrix([u1,u2,u3,u4])
B2=column_matrix([w1,w2,w3,w4])
v=vector([1,2,0,1,0,-1])
\end{sageblock}

Aplicamos el método de Gram-Schmidt a los vectores

$$ v_1=\sage{B[:,0]} \qquad v_2=\sage{B[:,1]} \qquad v_3=\sage{B[:,2]} \qquad v_4=\sage{B[:,3]} $$

y obtenemos la base ortogonal dada por las columnas de la matriz

$$ B_1 = \sage{B1} $$
	
Una vez normalizada se obtiene la base ortonormal dada por las columnas de la matriz:

$$ B_2 = \sage{B2} $$

Como mera comprobación $B_2$ es una matriz ortogonal:

$$ B_2^T\cdot B_2 = \sage{B2.T*B2}.$$

La proyección de $v$ sobre $U$ es

$$ {\rm proy}_U(v) = B\cdot ((B^T\cdot B)^{-1}\cdot B^T)\cdot v = \sage{B*(B.T*B)^-1*B.T*column_matrix([v])}$$

Si utilizamos la base ortonormal $B_2$ en vez de la base $B$, de manera alternativa también se puede calcular la proyección así:

$$
	{\rm proy}_U(v) = B_2\cdot B_2^T\cdot v = (v*w_1)w_1+(v*w_2)w_2+(v*w_3)w_3+(v*w_4)w_4  = \sage{B2*B2.T*column_matrix([v])}.
$$

% Fin Ejercicio 8

\section{Mínimos Cuadrados}

\begin{ejer} 
Calcula la recta de regresión asociada a los datos que se proporcionan a continuación:
\begin{sageblock}
XY = matrix(RR,[[ 2.385560477076296 ,  -2.395822899318959 ],
                [ 1.905248601109367 ,  -1.6229129728676281 ],
                [ 2.1431869782379556 ,  -2.0645676597180205 ],
                [ 1.3544213528768796 ,  -0.7196308967482228 ],
                [ 1.523305672725867 ,  -0.5993214201342096 ],
                [ 1.9786571047006658 ,  -1.6669634951798296 ],
                [ 2.6018225949955114 ,  -2.866463639001059 ],
                [ 2.000064502538637 ,  -1.9491976003788676 ],
                [ 2.018559544597864 ,  -1.5956613748963764 ],
                [ 2.2810102232428395 ,  -2.5347565098507556 ]])

X = XY.column(0)
Y = XY.column(1)
\end{sageblock}

La representación gráfica de estos puntos es:

\begin{sagesub}
\begin{center}
\begin{tikzpicture}
\foreach \p in !{set(XY.rows())}
  \filldraw[blue] \p circle (0.05);
%\draw[red, domain = !{min(X)-.2}:!{max(X)+.2}, very thick]
%  plot(\x,{3.14+2.52*\x});
\draw[->] (!{min(X)-2},0)--(!{max(X)+2},0) node[right]{$x$};
\draw[->] (0,!{min(Y)-2})--(0,!{max(Y)+2}) node[above]{$y$};
\end{tikzpicture}
\end{center}
\end{sagesub}


\end{ejer} 

{\it Solución:}

% Escribe su solución para el Ejercicio 9

\begin{sageblock}
B = matrix([[1 for x in X],
            [x for x in X]]).T
C = (B.T*B)^-1 * B.T * Y
\end{sageblock}

Vamos a construir una solución del tipo $y = c_0 + c_1 x$, para ello tomamos
la matriz que tiene como columnas $1$ y $x$, es decir
$$ B = \sage{B} $$
Si tomamos $C$ la columna con las variables $c_0$ y $c_1$, tenemos un sistema 
de ecuaciones $B C = Y$ que podemos resolver por mínimos cuadrados con la fórmula
$C = (B^T B)^{-1} B^T Y$. De esta forma obtenemos 
$$ C = \sage{C}.$$
Si representamos gráficamente la solución obtenemos:

\begin{sagesub}
\begin{center}
\begin{tikzpicture}
\foreach \p in !{set(XY.rows())}
  \filldraw[blue] \p circle (0.05);
\draw[red, domain = !{min(X)-.2}:!{max(X)+.2}, very thick]
  plot(\x,{!{C[0]}+!{C[1]}*\x});
\draw[->] (!{min(X)-2},0)--(!{max(X)+2},0) node[right]{$x$};
\draw[->] (0,!{min(Y)-2})--(0,!{max(Y)+2}) node[above]{$y$};
\end{tikzpicture}
\end{center}
\end{sagesub}

% Fin del Ejercicio 9

\begin{ejer} 
Calcula la recta de regresión asociada a los datos que se proporcionan a continuación:
\begin{sageblock}
XY = matrix(RR,[[ 0.577634973301552 ,  -0.021892353516998586 ],
                [ 2.2346179266047286 ,  0.4344125028412995 ],
                [ 3.4846606572341514 ,  -0.3181800856916049 ],
                [ 0.8085216507157904 ,  0.3558535279864201 ],
                [ 2.436171618880355 ,  -0.011007029039438766 ],
                [ 2.529682379873726 ,  0.11015658313238955 ],
                [ 4.303663555225892 ,  0.13678910258637164 ],
                [ 4.434181331472274 ,  -0.09759478810728146 ],
                [ 2.1031791958182837 ,  0.2667520474182835 ],
                [ 0.5662953947120373 ,  0.12428487018538861 ]])

X = XY.column(0)
Y = XY.column(1)
\end{sageblock}

La representación gráfica de estos puntos es:

\begin{sagesub}
\begin{center}
\begin{tikzpicture}
\foreach \p in !{set(XY.rows())}
  \filldraw[blue] \p circle (0.05);
%\draw[red, domain = !{min(X)-.2}:!{max(X)+.2}, very thick]
%  plot(\x,{3.14+2.52*\x});
\draw[->] (!{min(X)-2},0)--(!{max(X)+2},0) node[right]{$x$};
\draw[->] (0,!{min(Y)-2})--(0,!{max(Y)+2}) node[above]{$y$};
\end{tikzpicture}
\end{center}
\end{sagesub}


\end{ejer} 

{\it Solución:}

% Escribe su solución para el Ejercicio 10

\begin{sageblock}
B = matrix([[1 for x in X],
            [x for x in X]]).T
C = (B.T*B)^-1 * B.T * Y
\end{sageblock}

Vamos a construir una solución del tipo $y = c_0 + c_1 x$, para ello tomamos
la matriz que tiene como columnas $1$ y $x$, es decir
$$ B = \sage{B} $$
Si tomamos $C$ la columna con las variables $c_0$ y $c_1$, tenemos un sistema 
de ecuaciones $B C = Y$ que podemos resolver por mínimos cuadrados con la fórmula
$C = (B^T B)^{-1} B^T Y$. De esta forma obtenemos 
$$ C = \sage{C}.$$
Si representamos gráficamente la solución obtenemos:

\begin{sagesub}
\begin{center}
\begin{tikzpicture}
\foreach \p in !{set(XY.rows())}
  \filldraw[blue] \p circle (0.05);
\draw[red, domain = !{min(X)-.2}:!{max(X)+.2}, very thick]
  plot(\x,{!{C[0]}+!{C[1]}*\x});
\draw[->] (!{min(X)-2},0)--(!{max(X)+2},0) node[right]{$x$};
\draw[->] (0,!{min(Y)-2})--(0,!{max(Y)+2}) node[above]{$y$};
\end{tikzpicture}
\end{center}
\end{sagesub}

% Fin del Ejercicio 10

\begin{ejer} 
Calcula la parábola que mejor se ajusta a los datos que se proporcionan a continuación:
\begin{sageblock}
XY = matrix(RR,[[ 2.7612642659418345 ,  -1.1156708175321115 ],
                [ 2.103443783724906 ,  0.31383231157572405 ],
                [ 1.902670552753022 ,  0.25032326843590824 ],
                [ 2.1869201945783363 ,  0.24762559905865678 ],
                [ 1.6648725595631342 ,  0.5985413716434942 ],
                [ 2.180936578280802 ,  0.10312969275192364 ],
                [ 2.166571283712363 ,  0.47415888561041936 ],
                [ 1.4134229078250833 ,  0.05736296276262719 ],
                [ 1.895504173706609 ,  -0.036508422559938136 ],
                [ 2.2224525503221937 ,  -0.03560723558130427 ]])

X = XY.column(0)
Y = XY.column(1)
\end{sageblock}

La representación gráfica de estos puntos es:

\begin{sagesub}
\begin{center}
\begin{tikzpicture}
\foreach \p in !{set(XY.rows())}
  \filldraw[blue] \p circle (0.05);
%\draw[red, domain = !{min(X)-.2}:!{max(X)+.2}, very thick]
%  plot(\x,{3.14+2.52*\x+1.22*\x*\x});
\draw[->] (!{min(X)-2},0)--(!{max(X)+2},0) node[right]{$x$};
\draw[->] (0,!{min(Y)-2})--(0,!{max(Y)+2}) node[above]{$y$};
\end{tikzpicture}
\end{center}
\end{sagesub}


\end{ejer} 

{\it Solución:}

% Escribe su solución para el Ejercicio 11

\begin{sageblock}
B = matrix([[1 for x in X],
            [x for x in X],
            [x^2 for x in X]]).T
C = (B.T*B)^-1 * B.T * Y
\end{sageblock}

Vamos a construir una solución del tipo $y = c_0 + c_1 x + c_2 x^2$, para ello tomamos
la matriz que tiene como columnas $1,x$ y $x^2$, es decir
$$ B = \sage{B} $$
Si tomamos $C$ la columna con las variables $c_0,c_1$ y $c_2$, tenemos un sistema 
de ecuaciones $B C = Y$ que podemos resolver por mínimos cuadrados con la fórmula
$C = (B^T B)^{-1} B^T Y$. De esta forma obtenemos 
$$ C = \sage{C}.$$
Si representamos gráficamente la solución obtenemos:

\begin{sagesub}
\begin{center}
\begin{tikzpicture}
\foreach \p in !{set(XY.rows())}
  \filldraw[blue] \p circle (0.05);
\draw[red, domain = !{min(X)-.2}:!{max(X)+.2}, very thick]
  plot(\x,{!{C[0]}+!{C[1]}*\x+!{C[2]}*\x*\x});
\draw[->] (!{min(X)-2},0)--(!{max(X)+2},0) node[right]{$x$};
\draw[->] (0,!{min(Y)-2})--(0,!{max(Y)+2}) node[above]{$y$};
\end{tikzpicture}
\end{center}
\end{sagesub}

% Fin del Ejercicio 11

\begin{ejer} 
Calcula el polinomio cúbico que mejor se ajusta a los datos que se proporcionan a 
continuación:
\begin{sageblock}
XY = matrix(RR,[[ 2.4287087204306785 ,  -0.43197724654693614 ],
                [ 2.5890848700434073 ,  -0.1594124267242222 ],
                [ 3.818747862986166 ,  2.2687403350956052 ],
                [ 2.884116866851538 ,  -0.019315878916267435 ],
                [ 3.890299468309755 ,  2.8459293720843775 ],
                [ 3.219690337713705 ,  0.09004858390854256 ],
                [ 2.023838230298453 ,  -0.7236683135884444 ],
                [ 1.728399800189709 ,  -1.719850517119307 ],
                [ 3.830415822389739 ,  2.4528159160605782 ],
                [ 1.94577565082535 ,  -0.5302661204039961 ]])

X = XY.column(0)
Y = XY.column(1)
\end{sageblock}

La representación gráfica de estos puntos es:

\begin{sagesub}
\begin{center}
\begin{tikzpicture}
\foreach \p in !{set(XY.rows())}
  \filldraw[blue] \p circle (0.05);
%\draw[red, domain = !{min(X)-.2}:!{max(X)+.2}, very thick]
%  plot(\x,{3.14+2.52*\x+1.22*\x*\x-0.22*\x*\x*\x});
\draw[->] (!{min(X)-2},0)--(!{max(X)+2},0) node[right]{$x$};
\draw[->] (0,!{min(Y)-2})--(0,!{max(Y)+2}) node[above]{$y$};
\end{tikzpicture}
\end{center}
\end{sagesub}


\end{ejer} 

{\it Solución:}

% Escribe su solución para el Ejercicio 12

\begin{sageblock}
B = matrix([[1 for x in X],
            [x for x in X],
            [x^2 for x in X],
            [x^3 for x in X]]).T
C = (B.T*B)^-1 * B.T * Y
\end{sageblock}

Vamos a construir una solución del tipo $y = c_0 + c_1 x + c_2 x^2 + c_3 x^3$, para 
ello tomamos la matriz que tiene como columnas $1,x,x^2$ y $x^3$, es decir
$$ B = \sage{B} $$
Si tomamos $C$ la columna con las variables $c_0,c_1,c_2$ y $c_3$, tenemos un sistema 
de ecuaciones $B C = Y$ que podemos resolver por mínimos cuadrados con la fórmula
$C = (B^T B)^{-1} B^T Y$. De esta forma obtenemos 
$$ C = \sage{C}.$$
Si representamos gráficamente la solución obtenemos:

\begin{sagesub}
\begin{center}
\begin{tikzpicture}
\foreach \p in !{set(XY.rows())}
  \filldraw[blue] \p circle (0.05);
\draw[red, domain = !{min(X)-.2}:!{max(X)+.2}, very thick]
  plot(\x,{!{C[0]}+!{C[1]}*\x+!{C[2]}*\x*\x+!{C[3]}*\x*\x*\x});
\draw[->] (!{min(X)-2},0)--(!{max(X)+2},0) node[right]{$x$};
\draw[->] (0,!{min(Y)-2})--(0,!{max(Y)+2}) node[above]{$y$};
\end{tikzpicture}
\end{center}
\end{sagesub}

% Fin del Ejercicio 12

\end{document}
