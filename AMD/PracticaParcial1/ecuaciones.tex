\documentclass{amsart}
%\usepackage[usefamily=sage]{pythontex} 
\usepackage{sagetex}
\usepackage[utf8]{inputenc}
\usepackage[margin = 2cm]{geometry}
%\usepackage[spanish]{babel}
\usepackage[most]{tcolorbox}
\newtheorem{ejer}{Ejercicio}
\newtheorem{ejem}{Ejemplo}
\newtheorem{defn}{Definición}
\newtheorem{prop}{Proposición}

\title{Ecuaciones Matriciales}
\author{Leandro Marín \\ Facultad de Informática. Universidad de Murcia}

\begin{document}
\maketitle

Vamos a volver a mirar la resolución de sistemas de ecuaciones de una forma más 
genérica, lo cual nos permitirá resolver problemas de ecuaciones matriciales.

\begin{ejer}
Resuelve el siguiente sistema de ecuaciones sobre ${\mathbb Z}_5$

\begin{sageblock}
A = matrix(Zmod(5),[[4,2,1,0,1,4],
[1,3,4,0,0,1],
[4,2,2,2,0,4],
[3,4,3,2,2,3],
[1,3,1,4,0,1]])
B = matrix(Zmod(5),[[4],[0],[4],[2],[3]])
Pol = PolynomialRing(Zmod(5),x,6)
X = column_matrix(Pol,Pol.gens())
Ecuaciones = A*X
\end{sageblock}


\begin{align*}
\sage{Ecuaciones[0,0]} &= \sage{B[0,0]} \\
\sage{Ecuaciones[1,0]} &= \sage{B[1,0]} \\
\sage{Ecuaciones[2,0]} &= \sage{B[2,0]} \\
\sage{Ecuaciones[3,0]} &= \sage{B[3,0]} \\
\sage{Ecuaciones[4,0]} &= \sage{B[4,0]} \\
\end{align*}
\end{ejer}

{\it Solución:}

% Escribe tu solución para el Ejercicio 1

\begin{sageblock}
	AB = block_matrix([[A, B]])
	ABr = AB.echelon_form()
\end{sageblock}

$$
	[A|B] = \sage{AB} \to \sage{ABr}
$$

No sale identidad por tanto es un SCI



% Fin del ejercicio 1

\begin{ejer}
Calcular todas las matrices $X$ sobre el cuerpo ${\mathbb Z}_5$ tales que 
$$ \left(\begin{array}{rrrr}
4 & 2 & 4 & 1 \\
2 & 4 & 4 & 3 \\
1 & 3 & 0 & 1
\end{array}\right) X 
= \left(\begin{array}{rr}
0 & 4 \\
1 & 1 \\
1 & 0
\end{array}\right).$$
\end{ejer}

{\it Solución:} 

% Escribe tu solución para el Ejercicio 2

\begin{sageblock}
A = matrix(Zmod(5),[[4,2,4,1],[2,4,4,3],[1,3,0,1]])
B = matrix(Zmod(5),[[0,4],[1,1],[1,0]])
\end{sageblock}

Lo primero que tenemos que determinar es el tamaño de la matriz $X$, que deberá tener
$4$ filas para poder multiplicarse por la izquierda con la matriz que nos proponen
y dos columnas para que el resultado sea una matriz $3 \times 2$. 

\begin{sageblock}
Pol = PolynomialRing(Zmod(5),8,'x')
X = matrix(Pol,4,2,Pol.gens())
\end{sageblock}

El sistema que tenemos es por lo tanto:

$$ \sage{A} \sage{X} = \sage{B}$$

Para resolverlo vamos a reducir la matriz $A$ obteniendo la matriz de paso $P$.

\begin{sageblock}
M = block_matrix([[A,1]])
R = M.echelon_form()
P = R.subdivision(0,1)
PA = P*A
PB = P*B
\end{sageblock}

Lo que nos da el sistema en forma simplificada

$$ \sage{PA} \sage{X} = \sage{PB} $$

A la matriz reducida le falta un pivote y si lo añadimos, tenemos que añadir una
nueva fila a la parte derecha, eso corresponderá a dos parámetros libres:

\begin{sageblock}
Pivotes = matrix(Zmod(5),[[0,0,0,1]])
Pol2.<a,b> = PolynomialRing(Zmod(5))
Parametros = matrix(Pol2,[[a,b]])
PAE = block_matrix([[PA],[Pivotes]])
PBE = block_matrix([[PB],[Parametros]])
\end{sageblock}

$$ \sage{PAE} \sage{X} = \sage{PBE} $$

La matriz que multiplica a las incóginitas ya es invertible porque tiene todos
sus pivotes y podemos despejar
$$ \sage{X} = \sage{PAE}^{-1} \sage{PBE} = \sage{PAE^-1 * PBE}$$
donde $a,b \in {\mathbb Z}_5$, que es la solución al problema.

% Fin del Ejercicio 2

\begin{ejer}
Calcula todas las matrices $X$ sobre ${\mathbb Z}_5$ tales que $AXC = B$ siendo

\begin{sageblock}
A =  matrix(Zmod(5),[[2,1],
[2,1],
[2,1],
[3,4]])
B =  matrix(Zmod(5),[[4,3,2],
[4,3,2],
[4,3,2],
[1,2,3]])
C =  matrix(Zmod(5),[[4,3,2],
[1,3,2],
[4,1,4],
[2,3,2],
[4,3,2]])
\end{sageblock}

$$ A = \sage{A} \quad C = \sage{C} \quad B = \sage{B} $$
\end{ejer}

{\it Solución:} 

% Escribe tu solución para el Ejercicio 3

Empecemos determinando el tamaño de $X$. Puesto que $X$ se tiene que poder
multiplicar por la derecha por $A$ deberá tener $2$ filas y como se tiene que
poder multiplicar por la izquierda por $C$ tendrá que tener $5$ columnas.

\begin{sageblock}
Pol = PolynomialRing(Zmod(5),10,'x')
X = matrix(Pol,2,5,Pol.gens())
\end{sageblock}

La ecuación nos queda:
$$ \sage{A} \sage{X} \sage{C} = \sage{B}$$

Empecemos despejando la parte izquierda. Para ello hacemos la reducción

\begin{sageblock}
M1 = block_matrix([[A,1]])
R1 = M1.echelon_form()
P1 = R1.subdivision(0,1)
P1A = P1*A
P1B = P1*B
P1A.subdivide(1,[])
P1B.subdivide(1,[])
\end{sageblock}

$$\sage{P1A} \sage{X} \sage{C} = \sage{P1B} $$

Tenemos tres filas de ceros en la primera matriz que se corresponderá con tres 
fila de ceros en la parte derecha. Como esas filas están efectivamente en la parte
derecha, el sistema de momento es compatible. 
Vamos a eliminar esas fila de ceros tanto en la parte
izquierda como en la derecha puesto que no aportan información adicional.

\begin{sageblock}
P1A0 = P1A.subdivision(0,0)
P1B0 = P1B.subdivision(0,0)
\end{sageblock}

$$\sage{P1A0} \sage{X} \sage{C} = \sage{P1B0} $$

Antes de empezar a meter parámetros libres, debemos despejar también las filas
de ceros de la parte derecha, para ello tomaremos traspuestas y pondremos
el sistema de la forma 

$$\sage{C.T} \sage{X.T} \sage{P1A0.T} = \sage{P1B0.T} $$

Repetiremos el proceso por este lado reduciendo la matriz de la izquierda.

\begin{sageblock}
A = C.T
B = P1B0.T
C = P1A0.T
M1 = block_matrix([[A,1]])
R1 = M1.echelon_form()
P1 = R1.subdivision(0,1)
P1A = P1*A
P1B = P1*B
P1A.subdivide(2,[])
P1B.subdivide(2,[])
\end{sageblock}

$$\sage{P1A} \sage{X.T} \sage{C} = \sage{P1B} $$

Como podemos observar, en este caso también nos sale una fila de ceros en la 
parte inferior de ambos miembros (si no fuera así el sistema sería incompatible).
Vamos a eliminar esta información redundante y dejar la ecuación del siguiente modo:

\begin{sageblock}
P1A0 = P1A.subdivision(0,0)
P1B0 = P1B.subdivision(0,0)
\end{sageblock}

$$\sage{P1A0} \sage{X.T} \sage{C} = \sage{P1B0} $$

Para añadir los pivotes que nos faltan en la parte izquierda necesitamos 
tres filas nuevas y para añadir los pivotes en la parte derecha necesitaríamos
una columna nueva. Eso hace que tengamos que convertir la matriz de la parte 
derecha en una matriz de 5 filas y 2 columnas usando parámetros libres. El 
número de parámetros es $10-2 = 8$ parámetros. 

\begin{sageblock}
Pivotes1 = matrix(Zmod(5),[[0,0,1,0,0],[0,0,0,1,0],[0,0,0,0,1]])
Pivotes2 = matrix(Zmod(5),[[0],[1]])
Pol2.<a,b,c,d,e,f,g,h> = PolynomialRing(Zmod(5))
BB = matrix(Pol2,[[P1B0[0,0],a],
                  [P1B0[1,0],b],
                  [c,d],
                  [e,f],
                  [g,h]])
BB.subdivide(2,1)
P1A0E = block_matrix([[P1A0],[Pivotes1]])
CE = block_matrix([[C,Pivotes2]])
Sol = (P1A0E^-1 * BB * CE^-1).T
\end{sageblock}

$$\sage{P1A0E} \sage{X.T} \sage{CE} = \sage{BB} $$

De ahí podemos ya despejar el valor de $X$ que será

$$\sage{X.T} = 
\sage{P1A0E}^{-1} \sage{BB} \sage{CE}^{-1} $$
$$ = \sage{Sol.T}$$
donde $a,b,c,d,e,f,g$ y $h$ son parámetros libres. Tomando traspuestas nos sale
$$ X = \sage{X} =$$ $$ \sage{Sol} $$
y podemos comprobar que el resultado es correcto haciendo el producto 
original:

\begin{sageblock}
A =  matrix(Zmod(5),[[2,1],
[2,1],
[2,1],
[3,4]])
B =  matrix(Zmod(5),[[4,3,2],
[4,3,2],
[4,3,2],
[1,2,3]])
C =  matrix(Zmod(5),[[4,3,2],
[1,3,2],
[4,1,4],
[2,3,2],
[4,3,2]])
\end{sageblock}

$$ AX = \sage{A} \sage{Sol}$$ $$ = \sage{A*Sol}; $$
$$ AXC = \sage{A*Sol} \sage{C} = \sage{A*Sol*C} = B. $$
% Fin del Ejercicio 3

\begin{ejer}
Calcula todas las matrices $X$ sobre ${\mathbb Z}_5$ tales que $AX = B$ siendo

\begin{sageblock}
A =  matrix(Zmod(5),[[4,2,0,1,1],
[4,2,0,1,1],
[2,1,3,2,3]])
B =  matrix(Zmod(5),[[1,3],
[1,3],
[2,4]])
\end{sageblock}

$$ A = \sage{A} \quad B = \sage{B} $$
\end{ejer}

{\it Solución:} 

% Escribe tu solución para el Ejercicio 4

% Fin del Ejercicio 4

\begin{ejer}
Calcula todas las matrices $X$ sobre ${\mathbb Z}_5$ tales que $AX = B$ siendo

\begin{sageblock}
A =  matrix(Zmod(5),[[2,4,1,1,4],
[0,3,2,2,4],
[2,4,0,2,0],
[3,4,1,1,0]])
B =  matrix(Zmod(5),[[3,0],
[4,2],
[0,0],
[1,2]])
\end{sageblock}

$$ A = \sage{A} \quad B = \sage{B} $$
\end{ejer}

{\it Solución:} 

% Escribe tu solución para el Ejercicio 5

% Fin del Ejercicio 5

\begin{ejer}
Calcula todas las matrices $X$ sobre ${\mathbb Z}_5$ tales que $AX = B$ siendo

\begin{sageblock}
A =  matrix(Zmod(5),[[3,2,1,1],
[1,4,0,1],
[2,3,4,4],
[4,0,4,4]])
B =  matrix(Zmod(5),[[4,4],
[3,1],
[1,1],
[1,0]])
\end{sageblock}

$$ A = \sage{A} \quad B = \sage{B} $$
\end{ejer}

{\it Solución:} 

% Escribe tu solución para el Ejercicio 6

% Fin del Ejercicio 6

\begin{ejer}
Calcula todas las matrices $X$ sobre ${\mathbb Z}_5$ tales que $XC = B$ siendo

\begin{sageblock}
B =  matrix(Zmod(5),[[1,4,1],
[3,3,0]])
C =  matrix(Zmod(5),[[0,1,2],
[1,2,2],
[1,1,0]])
\end{sageblock}

$$ C = \sage{C} \quad B = \sage{B} $$
\end{ejer}

{\it Solución:} 

% Escribe tu solución para el Ejercicio 7

% Fin del Ejercicio 7

\begin{ejer}
Calcula todas las matrices $X$ sobre ${\mathbb Z}_5$ tales que $XC = B$ siendo

\begin{sageblock}
B =  matrix(Zmod(5),[[4,3,1],
[3,1,2],
[1,4,3]])
C =  matrix(Zmod(5),[[4,1,2],
[2,1,2],
[4,4,3]])
\end{sageblock}

$$ C = \sage{C} \quad B = \sage{B} $$
\end{ejer}

{\it Solución:} 

% Escribe tu solución para el Ejercicio 8

% Fin del Ejercicio 8

\begin{ejer}
Calcula todas las matrices $X$ sobre ${\mathbb Z}_5$ tales que $AXC = B$ siendo

\begin{sageblock}
A =  matrix(Zmod(5),[[1,4,0,1],
[4,1,2,1],
[0,1,3,1],
[0,3,3,2]])
B =  matrix(Zmod(5),[[2,0,2],
[2,2,0],
[1,3,3],
[1,3,3]])
C =  matrix(Zmod(5),[[1,3,3],
[3,1,2],
[3,3,0]])
\end{sageblock}

$$ A = \sage{A} \quad C = \sage{C} \quad B = \sage{B} $$
\end{ejer}

{\it Solución:} 

% Escribe tu solución para el Ejercicio 9

% Fin del Ejercicio 9

\end{document}

