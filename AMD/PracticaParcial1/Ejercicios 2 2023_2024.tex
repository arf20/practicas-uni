

\documentclass{amsart}
\usepackage{sagetex}
\usepackage[utf8]{inputenc}
\newtheorem{ejer}{Ejercicio}

\def\n{\mathbb{N}}
\def\r{\mathbb{R}}
\def\z{\mathbb{Z}}
\def\q{\mathbb{Q}}
\def\c{\mathbb{C}}

\begin{document}
\begin{center}
\textrm{ \bf {Ejercicios Hoja 2}}
\vskip 0.3cm
\textrm{ \bf {Curso 2023/2024} }
\end{center}


\begin{ejer} Hallar una base $\mathcal{B}$ d el espacio $\r ^3$ tal que se verifiquen las siguientes igualdades:
\[ \left( \begin{array}{r}
1 \\ -1 \\ 1             
\end{array} \right) = \left( \begin{array}{r}
2 \\ 3 \\ -1       
\end{array} \right)_{\mathcal{B}}, \ \left( \begin{array}{r}
1 \\ 0 \\ 1             
\end{array} \right) = \left( \begin{array}{r}
1 \\ -2 \\ 2       
\end{array} \right)_{\mathcal{B}}  \hbox{ y } \left( \begin{array}{r}
0 \\ 1 \\ 1             
\end{array} \right) = \left( \begin{array}{r}
-1 \\ 1 \\ 0       
\end{array} \right)_{\mathcal{B}} \]
\end{ejer}

\begin{sageblock}
	AC = matrix(QQ, [[1, 1, 0], [-1, 0, 1], [1, 1, 1]])
	AB = matrix(QQ, [[2, 1, -1], [3, -2, 1], [-1, 2, 0]])
	B = AC * AB.inverse()
\end{sageblock}

$$
	B \cdot \sage{AB} = \sage{AC}
$$

$$
	B = \sage{AC} \cdot {\sage{AB}}^{-1}
$$

$$
	B = \sage{B}
$$

\begin{ejer} Dadas las aplicaciones
\begin{enumerate}
\item[a)] $f_1: \r ^2 \longrightarrow \r ^4$ definida por $f_1 \left( \begin{array}{c}
x_1 \\ x_2             
\end{array} \right) = \left( \begin{array}{c}
x_1-x_2 \\ 2x_1+x_2 \\ x_1+2x_2 \\ 3x_1      
\end{array} \right) $
\item[b)] $f_2: \z ^4_3 \longrightarrow \z ^2_3$ definida por $f_2 \left( \begin{array}{c}
x_1 \\ x_2 \\ x_3 \\ x_4            
\end{array} \right) = \left( \begin{array}{c}
x_1-2x_2+x_3+2x_4 \\ 2x_1+x_3-x_4             
\end{array} \right)$ .
\item[c)] $f_3: \r ^2 \longrightarrow \r ^3$ definida por $f_1 \left( \begin{array}{c}
x_1 \\ x_2             
\end{array} \right) = \left( \begin{array}{c}
x_1-x_2 \\ 2x_1+x_2 \\ 1      
\end{array} \right) $
\item[d)] $f_4:\hbox{P}_3(\r ) \longrightarrow  \hbox {P}_2(\r )$ definida por $f_4(a_0+a_1x+a_2x^2+a_3x^3) = a_1+2a_2x+3a_3x^2$.
\end{enumerate} 
\vskip 0.2 cm
Estudiar si son aplicaciones lineales y, en su caso determinar su matriz asociada $\mathcal{M}(f)$.
\end{ejer}

\begin{sageblock}
	Mf1 = matrix(QQ, [[1, -1], [2, 1], [1, 2], [3, 0]])
	Mf2 = matrix(QQ, [[1, -2, 1, 2], [2, 0, 1, -1]])
	Mf4 = matrix(QQ, [[1, 2, 3, 0]])
\end{sageblock}

a.
$$
	M(f_1) = \sage{Mf1}
$$
b.
$$
	M(f_2) = \sage{Mf2}
$$
c. No es lineal porque tiene elementos no $x$
d.
$$
	M(f_4) = \sage{Mf4}
$$


\begin{ejer} Dado el sistema de vectores, sobre el cuerpo $\z _5$, 
\[\mathcal{B} = \left\lbrace \left( \begin{array}{r}
1 \\ 1 \\ -1            
\end{array} \right), 
\left( \begin{array}{c}
           
1 \\ 1 \\ 1           
\end{array} \right), 
\left( \begin{array}{c}
           
0 \\ 1 \\ 1           
\end{array} \right) \right\rbrace \] Comprobar si son base y caso de serlo calcular 
las matrices de paso $\mathcal{P}_{\mathcal{B}C_{3}}$ y $\mathcal{P}_{C_{3}\mathcal{B}}$

({\bf Nota.-} La matriz $C_3$ denota la base can\'onica de $\z ^3_5$.)
\end{ejer}

\begin{sageblock}
	B = matrix(Zmod(5), [[1, 1, 0], [1, 1, 1], [-1, 1, 1]])
	Br = B.echelon_form()
	Bi = B.inverse()
\end{sageblock}

$$
	B = \sage{B} \to  \sage{Br}
$$
B es base.

$$
	P_{BC_3} = C_3^{-1} B = B = \sage{B}
$$

$$
	P_{C_3B} = B^{-1} C_3 = B^{-1} = \sage{Bi}
$$

\begin{ejer} Dados los siguientes conjuntos de vectores, sobre el cuerpo $\z _5$,  

\[\mathcal{B}_1 = \left\lbrace \left( \begin{array}{c}
1 \\ 0 \\ 1            
\end{array} \right), 
\left( \begin{array}{c}
           
1 \\ 1 \\ 0           
\end{array} \right), 
\left( \begin{array}{c} 
           
0 \\ 1 \\ 1           
\end{array} \right) \right\rbrace \]
\[\mathcal{B}_2 = \left\lbrace \left( \begin{array}{c}
0 \\ 0 \\ 1            
\end{array} \right), 
\left( \begin{array}{r}
           
-1 \\ 1 \\ 0           
\end{array} \right), 
\left( \begin{array}{c}
           
1 \\ 1 \\ 1           
\end{array} \right) \right\rbrace  \]

Comprobar si son base y, caso de serlo calcula la matriz de paso $\mathcal{P}_{B_1B_2}$.
\end{ejer}

\begin{sageblock}
	B1 = matrix(Zmod(5), [[1, 1, 0], [0, 1, 1], [1, 0, 1]])
	B2 = matrix(Zmod(5), [[0, -1, 1], [0, 1, 1], [1, 0, 1]])
	B1r = B1.echelon_form()
	B2r = B2.echelon_form()
	B1i = B1.inverse()
	B2i = B2.inverse()
	PB1B2 = B2i * B1
\end{sageblock}

$$
	B_1 = \sage{B1} \to \sage{B1r}
$$

$$
	B_2 = \sage{B2} \to \sage{B2r}
$$

Los dos son bases.

$$
	\mathcal{P}_{B_1B_2} = B_2^{-1} B_1 = \sage{PB1B2}
$$
 
\begin{ejer} Sea la aplicaci\'on lineal:
\[ f: \z ^2_5 \longrightarrow \z ^4_5 \ \ \hbox{definida por} \ \
f \left( \begin{array}{c}
x_1 \\ x_2             
\end{array} \right) = \left( \begin{array}{c}
x_1+2x_2 \\ 2x_1-3x_2 \\ 4x_1+2x_2 \\ 0      
\end{array} \right) \] y las bases:

 \[\mathcal{B}_1 = \left\lbrace \left( \begin{array}{c}
1 \\ 1            
\end{array} \right), 
\left( \begin{array}{c}
0 \\ 1            
\end{array} \right) \right\rbrace  \ \ \hbox{de}  \ \z ^2_5 \]

\[\mathcal{B}_2 = \left\lbrace \left( \begin{array}{c}
1 \\ 0 \\ 1 \\ 1            
\end{array} \right), 
\left( \begin{array}{c}
0 \\ 1 \\ 0 \\ 1           
\end{array} \right), 
\left( \begin{array}{c}
1 \\ 1 \\ 0 \\ 1           
\end{array} \right),
\left( \begin{array}{c}
0 \\ 0 \\ 0 \\ 1            
\end{array} \right) \right\rbrace \ \hbox{de}  \ \z ^4_5 \] Calcular las siguientes matrices:
\begin{itemize}
\item[a)] $\mathcal{M}_{B_1C_4}(f)$.
\item[b)] $\mathcal{M}_{C_2B_2}(f)$.
\item[c)] $\mathcal{M}_{B_1B_2}(f)$.
\end{itemize}

({\bf Nota.-} Las matrices $C_2$ y $C_4$ denotan las bases can\'onicas de $\z ^2_5$ y $\z ^4_5$ respectivamente.)
\end{ejer}

\begin{ejer} Dadas las aplicaciones lineales:
\[\r ^3  \buildrel g \over{\longrightarrow } \r ^2  \ \ \hbox{y} \ \r ^2  \buildrel f \over{\longrightarrow } \r ^3 \] definidas por:

\[g \left( \begin{array}{c}
x_1 \\ x_2 \\ x_3             
\end{array} \right) = \left( \begin{array}{c}
x_1+x_2+x_3 \\ x_1+3x_2-x_3             
\end{array} \right)  \ \ \hbox{y} \ \ f \left( \begin{array}{c}
x_1 \\ x_2             
\end{array} \right) = \left( \begin{array}{c}
2x_1-x_2 \\ x_1+3x_2 \\ -x_1-x_2             
\end{array} \right) \] y las bases
\[\mathcal{B}_1 = \left\lbrace \left( \begin{array}{r}
1 \\ 1            
\end{array} \right), 
\left( \begin{array}{r}
-1 \\ 1            
\end{array} \right) \right\rbrace \ \ \hbox{de}  \ \r ^2 \]
\[\mathcal{B}_2 = \left\lbrace \left( \begin{array}{r}
1 \\ 1 \\ 1            
\end{array} \right), 
\left( \begin{array}{r}
-1 \\ 1 \\ -1           
\end{array} \right), 
\left( \begin{array}{r}
0 \\ 0 \\ 2         
\end{array} \right) \right\rbrace \ \ \hbox{de}  \ \r ^3 \]
\[\mathcal{B}_3 = \left\lbrace \left( \begin{array}{r}
1 \\ -1 \\ 2            
\end{array} \right), 
\left( \begin{array}{r}
1 \\ 0 \\ -1           
\end{array} \right), 
\left( \begin{array}{r}
2 \\ -3 \\ 2         
\end{array} \right) \right\rbrace \ \ \hbox{de}  \ \r ^3 \]
Calcular: 
\begin{itemize}
\item[a)] $\mathcal{M}_{B_1B_2}(f)$
\item[b)] $\mathcal{M}_{B_3B_1}(g)$
\item[c)] $\mathcal{M}_{B_2B_3}(f\circ g)$
\end{itemize}
\end{ejer}

\begin{ejer} Calcular el n\'ucleo, $\hbox{Ker}(f)$, y el espacio imagen, $\hbox{Im}(f)$, de las siguientes aplicaciones lineales:
\begin{itemize}
\item[a)] la aplicaci\'on dada por la matriz

\[ \left( \begin{array}{cccc}
4 & 3 & 3 & 1 \\ 
2 & 3 & 3 & 1 \\
2 & 4 & 4 & 2 \\
4 & 1 & 2 & 3          
\end{array} \right) \in \hbox{M}_{4\times 4}(\z _5)\]


\item[b)] la aplicaci\'on lineal $f: \hbox{P}_2(\r ) \longrightarrow \r ^4$ definida por: 
\[f(p(x)) = (p(0),p(1),p(2),p(3))\] 

({\bf Nota.} $p(x)$ denota en general al polinomio 
$p(x) = a_0+a_1x+a_2x^2 \in  \hbox{P}_2(\r )$)

\end{itemize}
\end{ejer}

\begin{ejer} Sean las bases de $\z ^3_5$, 

\[\mathcal{B}_1 = \left\lbrace \left( \begin{array}{r}
2 \\ 4 \\ 2            
\end{array} \right), 
\left( \begin{array}{c}
           
0 \\ 1 \\ 1           
\end{array} \right), 
\left( \begin{array}{c}
           
4 \\ 4 \\ 3           
\end{array} \right) \right\rbrace \hbox{ y } 
\mathcal{B}_2 = \left\lbrace \left( \begin{array}{r}
1 \\ 1 \\ 1            
\end{array} \right), 
\left( \begin{array}{c}
           
0 \\ 2 \\ 3           
\end{array} \right), 
\left( \begin{array}{c}
           
0 \\ 0 \\ 3           
\end{array} \right) \right\rbrace \] y la aplicaci\'on lineal $f: \z ^3_5 \longrightarrow \z ^3_5$ cuya matriz asociada en bases $\mathcal{B}_2$ y $\mathcal{B}_1$ es:
\[ \mathcal{M}_{\mathcal{B}_2 \mathcal{B}_1}(f) = \left( \begin{array}{crc}
1 & 2 & 3  \\ 
2 & -1 & 1 \\
1 & 1 & 3           
\end{array} \right) \]
\begin{itemize}
\item[a)] Calcula la matriz del cambio de base $\mathcal{P}_{\mathcal{B}_2 \mathcal{B}_1}$

\item[b)] Dado el espacio $U = \left\lbrace  \left( \begin{array}{r}
x_1 \\ x_2 \\ x_3            
\end{array} \right)_{\mathcal{B}_1} | \  x_1-2x_2+x_3 = 0   \right\rbrace $. Hallar una base de $U$ cuyos vectores est\'en expresados en base can\'onica. 

\item[c)] Calcular $f\left( \begin{array}{r}
1 \\ 2 \\ 3            
\end{array} \right) $

\end{itemize}
\end{ejer}

\begin{ejer} Dado el espacio $V = C(B)$ siendo $B$ la matriz, sobre el cuerpo $\r $, 
\[B = \left( \begin{array}{cccc}
1 & 2 & 1 & 1 \\            
2 & 4 & 2 & 4 \\          
1 & 1 & 0 & 2 \\
3 & 7 & 4 & 6           
\end{array} \right) \]
\begin{itemize}
\item[a)] Estudia si los vectores son generadores, linealmente independientes o base.
\item[b)] Si no son base extrae una base de entre ellos.
\item[c)] Calcula unas ecuaciones impl\'{\i}citas de $V$.
\item[d)] Ampliar la base del espacio $V$, encontrada en el apartado b), a\~nadi\'endole los vectores que se necesiten para obtener una base del espacio $\r ^4$.
\end{itemize}
\end{ejer}

\begin{ejer} Dados los espacios $U = N(B)$ y $V=N(C)$ siendo $B$ y $C$  las matrices, sobre el cuerpo $\z ^5_5 $, 
\[B = \left( \begin{array}{rrrrr}
2 & -1 & 2 & 3 & -2 \\            
1 & 0 & -1 & 2 & 3 \\          
2 & 1 & 1 & 1 & -2           
\end{array} \right) \ \hbox{ y } \ C = \left( \begin{array}{rrrrr}
-3 & 4 & 2 & -2 & 3 \\            
2 & 0 & 3 & 4 & 1 \\          
1 & 3 & 3 & 3 & 4           
\end{array} \right) \] Se pide:
\begin{itemize}
\item[a)] Estudiar si ambos espacios son iguales o si alguno est\'a contenido en el otro.
\item[b)] Ecuaciones impl\'{\i}citas del espacio suma $U+V$ y del espacio intersecci\'on $U\cap V$.
\item[c)] Bases de los espacios $U+V$ y  $U\cap V$.
\end{itemize}
\end{ejer}


\begin{ejer} Dados los espacios de ${P}_3(\r )$
\[U = <1-x+x^2+x^3, x-x^2, -1+x^2, 2+x-2x^2+x^3> \] y 
\[V = <1-x-x^2+2x^3, 2x-2x^2, 3-3x^2, 3+2x-5x^2> .\] Se pide:
\begin{itemize}
\item[a)] Base y dimensi\'on de $U$ y $V$.
\item[b)] Ecuaciones impl\'{i}citas de $U$ y $V$.
\item[c)] Estudia si son iguales o si alguno est\'a contenido en el otro.
\item[d)] Encuentra un polinomio que est\'e en $V$ pero no en $U$. Comprueba que ese polinomio junto con los de la base de $U$ forman una base de $\hbox {P}_3(\r )$.
\item[e)] Ecuaciones param\'etricas de $U+V$ y de $U\cap V$. 

\end{itemize}


\end{ejer}






\end{document}