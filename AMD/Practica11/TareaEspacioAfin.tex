\documentclass{amsart}
\usepackage[utf8]{inputenc}
\usepackage[usefamily=sage]{pythontex} 
\usepackage{float}
\usepackage{tikz}
\usetikzlibrary{calc}
\usepackage[most]{tcolorbox}
\usepackage[margin = 2cm]{geometry}
\usepackage{graphicx}
\usetikzlibrary{3d}
\usepackage{tikz-3dplot}
%\usepackage{pgfplots}
%\pgfplotsset{compat=1.15}

\newtheorem{ejer}{Ejercicio}
\def\r{\mathbb{R}}

\author{Nombre Apellido Apellido}

\title{AMD - Tarea de Espacio Afín}

\begin{document}

\begin{sagecode}
latex.matrix_delimiters("[", "]")
\end{sagecode}

\maketitle

\begin{ejer}
Dado el polígono de vértices $C_0,C_1,C_2,C_3,C_4$ en el plano, determina la posición de los $P_0,P_1,\cdots,P_{9}$ con respecto a dicho polígono, es decir,
para cada uno de ellos determinar si están dentro, fuera o en el borde del
polígono (Sugerencia: Descomponer el polígono en varios triángulos y resolver cada problema por separado).

\begin{tcolorbox}[title = Datos]

\begin{sageblock}
C =  column_matrix(RR,[[-0.172651912033068,0.425473646411074],
[-0.927177836444634,0.730828051174769],
[-0.455968358491447,-0.920601173524937],
[0.406985519949459,-0.882736889806611],
[0.876887465140915,0.916803361839041]])
P =  column_matrix(RR,[
[0.629081226542386,0.741279405121107],
[-0.448184676007223,-0.303402020267078],
[-0.541604965947982,0.633876454181660],
[-0.364100606388708,0.257256981672285],
[0.429547739604162,0.444885711632425],
[0.407307318948960,0.762730650438023],
[-0.992167831095530,-0.468225919921707],
[0.813501066850795,0.380278966484519],
[0.194118465888270,-0.947683683347809],
[0.691495569525582,-0.786581561257583]])
\end{sageblock}

$$
C_0 = \sage{C[:,0]},
C_1 = \sage{C[:,1]},  
C_2 = \sage{C[:,2]},
$$
$$  
C_3 = \sage{C[:,3]},
C_4 = \sage{C[:,4]},
$$


$$
P_0 = \sage{P[:,0]},
P_1 = \sage{P[:,1]},  
P_2 = \sage{P[:,2]},
$$
$$  
P_3 = \sage{P[:,3]},
P_4 = \sage{P[:,4]},
P_5 = \sage{P[:,5]},
$$
$$  
P_6 = \sage{P[:,6]},  
P_7 = \sage{P[:,7]},
$$
$$
P_8 = \sage{P[:,8]},
P_9 = \sage{P[:,9]}
$$
\end{tcolorbox}
\end{ejer}

{\it Solución:}

% Inicio Ejercicio 1

\begin{sageblock}
CV = C.columns()
PV = P.columns()

h = matrix([1]*10)
hP = block_matrix(2, 1, [h, P]);

T0 = matrix(RR,  [
[1, 	1, 		1],
[CV[0][0], CV[1][0], 	CV[2][0]],
[CV[0][1], CV[1][1], 	CV[2][1]]]);
T1 = matrix(RR,  [
[1, 	1, 		1],
[CV[0][0], CV[2][0], 	CV[3][0]],
[CV[0][1], CV[2][1], 	CV[3][1]]]);
T2 = matrix(RR,  [
[1, 	1, 		1],
[CV[0][0], CV[3][0], 	CV[4][0]],
[CV[0][1], CV[3][1], 	CV[4][1]]]);

T0iP = T0^-1 * hP
T1iP = T1^-1 * hP
T2iP = T2^-1 * hP
\end{sageblock}

\begin{sagesub}
	\begin{tikzpicture}[scale = 3, 		cara/.style={thick, color = blue, fill opacity = 0.3, fill = blue!20},]
		\draw !{C[1]};
		\draw[fill = white] !{CV[0]} circle (0.5mm) node[above right] {$0$};
		\draw[fill = white] !{CV[1]} circle (0.5mm) node[above right] {$1$};
		\draw[fill = white] !{CV[2]} circle (0.5mm) node[above right] {$2$};
		\draw[fill = white] !{CV[3]} circle (0.5mm) node[above right] {$3$};
		\draw[fill = white] !{CV[4]} circle (0.5mm) node[above right] {$4$};
	\end{tikzpicture}
\end{sagesub}

Matrices de triangulos 012, 023 y 034

\begin{align*}
	T_0 &= \sage{T0} \\
	T_1 &= \sage{T1} \\
	T_2 &= \sage{T2} \\
\end{align*}

Matrices de puntos transpuestas para que no se salgan del papel, cada fila es un punto
$$
(T_0 \cdot P)^T = \sage{T0iP.T}
$$
$$
(T_1 \cdot P)^T = \sage{T1iP.T}
$$
$$
(T_2 \cdot P)^T = \sage{T2iP.T}
$$
% Fin Ejercicio 1

Puntos que están dentro del poligono (porque sus coordenadas baricentricas están entre 0 y 1 en almenos uno de los triangulos): 0, 1, 3, 4; Por tanto los que están fuera son 2, 5, 6, 7, 8, 9.

\end{document}



