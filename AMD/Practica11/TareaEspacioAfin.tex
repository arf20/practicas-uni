\documentclass{amsart}
\usepackage[utf8]{inputenc}
\usepackage[usefamily=sage]{pythontex} 
\usepackage{float}
\usepackage{tikz}
\usetikzlibrary{calc}
\usepackage[most]{tcolorbox}
\usepackage[margin = 2cm]{geometry}
\usepackage{graphicx}
\usetikzlibrary{3d}
\usepackage{tikz-3dplot}
%\usepackage{pgfplots}
%\pgfplotsset{compat=1.15}

\newtheorem{ejer}{Ejercicio}
\def\r{\mathbb{R}}

\author{Nombre Apellido Apellido}

\title{AMD - Tarea de Espacio Afín}

\begin{document}

\begin{sagecode}
latex.matrix_delimiters("[", "]")
\end{sagecode}

\maketitle

\begin{ejer}
Dado el polígono de vértices $C_0,C_1,C_2,C_3,C_4$ en el plano, determina la posición de los $P_0,P_1,\cdots,P_{9}$ con respecto a dicho polígono, es decir,
para cada uno de ellos determinar si están dentro, fuera o en el borde del
polígono (Sugerencia: Descomponer el polígono en varios triángulos y resolver cada problema por separado).

\begin{tcolorbox}[title = Datos]

\begin{sageblock}
C =  column_matrix(RR,[[-0.172651912033068,0.425473646411074],
[-0.927177836444634,0.730828051174769],
[-0.455968358491447,-0.920601173524937],
[0.406985519949459,-0.882736889806611],
[0.876887465140915,0.916803361839041]]
CV = P.columns()
P =  column_matrix(RR,[
[0.629081226542386,0.741279405121107],
[-0.448184676007223,-0.303402020267078],
[-0.541604965947982,0.633876454181660],
[-0.364100606388708,0.257256981672285],
[0.429547739604162,0.444885711632425],
[0.407307318948960,0.762730650438023],
[-0.992167831095530,-0.468225919921707],
[0.813501066850795,0.380278966484519],
[0.194118465888270,-0.947683683347809],
[0.691495569525582,-0.786581561257583]])
C0 = vector(C[:,0])
C1 = vector(C[:,1])
C2 = vector(C[:,2])
C3 = vector(C[:,3])
C4 = vector(C[:,4])
\end{sageblock}

$$
C_0 = \sage{C[:,0]},
C_1 = \sage{C[:,1]},  
C_2 = \sage{C[:,2]},
$$
$$  
C_3 = \sage{C[:,3]},
C_4 = \sage{C[:,4]},
$$


$$
P_0 = \sage{P[:,0]},
P_1 = \sage{P[:,1]},  
P_2 = \sage{P[:,2]},
$$
$$  
P_3 = \sage{P[:,3]},
P_4 = \sage{P[:,4]},
P_5 = \sage{P[:,5]},
$$
$$  
P_6 = \sage{P[:,6]},  
P_7 = \sage{P[:,7]},
$$
$$
P_8 = \sage{P[:,8]},
P_9 = \sage{P[:,9]}
$$
\end{tcolorbox}
\end{ejer}

{\it Solución:}

% Inicio Ejercicio 1

\begin{sagesub}
	\begin{tikzpicture}[scale = 3, 		cara/.style={thick, color = blue, fill opacity = 0.3, fill = blue!20},]
		\draw !{C[1]};
		\draw[fill = white] !{C0} circle (0.5mm) node[above right] {$0$};
		\draw[fill = white] !{C1} circle (0.5mm) node[above right] {$1$};
		\draw[fill = white] !{C2} circle (0.5mm) node[above right] {$2$};
		\draw[fill = white] !{C3} circle (0.5mm) node[above right] {$3$};
		\draw[fill = white] !{C4} circle (0.5mm) node[above right] {$4$};
	\end{tikzpicture}
\end{sagesub}

Triangulos 012, 023 y 034

\begin{sageblock}
R0 = matrix(RR,  [
[1, 	1, 		1],
[C0[0], C1[0], 	C2[0]],
[C0[1], C1[1], 	C2[1]]]);
R1 = matrix(RR,  [
[1, 	1, 		1],
[C0[0], C2[0], 	C3[0]],
[C0[1], C2[1], 	C3[1]]]);
R2 = matrix(RR,  [
[1, 	1, 		1],
[C0[0], C3[0], 	C4[0]],
[C0[1], C3[1], 	C4[1]]]);
\end{sageblock}

\begin{align*}
	R_0 &= \sage{R0} \\
	R_1 &= \sage{R1} \\
	R_2 &= \sage{R2} \\
\end{align*}

$$
	\sage{P.columns()}
$$

% Fin Ejercicio 1

\end{document}



