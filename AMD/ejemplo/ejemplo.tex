\documentclass{amsart}
\usepackage{sagetex}
\newtheorem{ejercicio}{Ejercicio}

\begin{document}

\begin{ejercicio}
Calcula la matriz inversa de la matriz 
$A = \left(\begin{array}{ccc} 1 & 2 & 1/2 \\ 3 & 1 & 0 \\ 2 & 3 & 1 \end{array}\right)$ 
definida sobre el cuerpo de los números racionales. 
\end{ejercicio}
{\it Solución.}

% Escribe aquí la solución a tu ejercicio

\begin{sageblock}
	A = matrix(QQ, [[1, 2, 1/2], [3, 1, 0], [2, 3, 1]])
	Ap = A.augment(matrix.identity(3), subdivide=True)
	Apr = Ap.echelon_form()
\end{sageblock}

La matriz ampliada por la identidad
$$
	[A|I] = \sage{Ap}
$$
reducida es
$$
	\sage{Apr} = [I|P]
$$
la identidad, por tanto A es invertible, y es
$$
	\sage{A.inverse()}
$$

% Fin del ejercicio

\end{document}
