\documentclass{amsart}
\usepackage{sagetex}
\usepackage[utf8]{inputenc}
\usepackage{float}
\newtheorem{ejer}{Ejercicio}
\newtheorem{defn}{Definición}
\newtheorem{prop}{Proposición}
\newtheorem{ejem}{Ejemplo}

\def\n{\mathbb{N}}
\def\r{\mathbb{R}}
\def\z{\mathbb{Z}}
\def\q{\mathbb{Q}}



\begin{document}

\begin{ejer}
Dadas las matrices 
\[ A =
	\left( \begin{array}{rrr}
		2 & 5 & -3 \\
		-1 & 2 & 11
	\end{array}
	\right) \ \ \hbox{y} \ \   B =
	\left( \begin{array}{rrr}
		2 & -1 & 3 \\
		2 & 4 & -7
	\end{array}
	\right) 
	\]
sobre el cuerpo de los racionales $\q$, realizar las siguientes operaciones: $2\cdot A$, $A+B$, $3\cdot A -2\cdot B$ y $A\cdot B^T$

\end{ejer}
{\it Soluci\'on:}
% Escribe tu soluci\'on para el Ejercicio 1

\begin{sageblock}
A = matrix(QQ, [[2, 5, -3], [-1, 2, 11]])
B = matrix(QQ, [[-1, 2, 11], [2, 4, -7]])
\end{sageblock}

$2 * A$ es $\sage{2 * A}$, $A + B$ es $\sage{A + B}$, $3 * A$ es $\sage{3 * A}$, y $A * B^T$ es $\sage{A * B.T}$

% Fin del Ejercicio 1

\begin{ejer}
Dada la matriz 
\[ C =
	\left( \begin{array}{rrrrr}
		-5 & 2 & -2  & 1 & 3 \\
		1 & 2 & -3 & 2 & -1 \\ 
		2 & -3 & 1 & -4 & 2
	\end{array}
	\right) \]
sobre el cuerpo $\z _{13}$. Se pide:
\begin{itemize}
\item Determinar su matriz reducida
\item Extraer las dos últimas columnas de dicha matriz reducida
\end{itemize}	
\end{ejer}
{\it Soluci\'on:}
% Escribe tu soluci\'on para el Ejercicio 2

\begin{sageblock}
	C = matrix(QQ, [[-5, 2, -2, 1, 3], [1, 2, -3, 2, -1], [2, -3, 1, -4, 2]])
\end{sageblock}

La matriz reducida de $C$ es $\sage{C.echelon_form()}$

Las ultimas dos columnas de $C$ reducida son $\sage{C.echelon_form()[:, 3:]}$

% Fin del Ejercicio 2

\begin{ejer}
Dada la matriz 
\[ A =
	\left( \begin{array}{rrrr}
		3 & 2 & 1 & 4 \\
		4 & 4 & 3 & 3 \\
		2 & 5 & 1 & 5 \\
		3 & 2 & 0 & 1 
	\end{array}
	\right) 
	\]
sobre $\z _5$, encuentra si es posible su inversa.
\end{ejer}
{\it Soluci\'on:}
% Escribe tu soluci\'on para el Ejercicio 3

\begin{sageblock}
	C = matrix(Zmod(5), [[3, 2, 1, 4], [4, 4, 3, 3], [2, 5, 1, 5], [3, 2, 0, 1]])
\end{sageblock}

La inversa en $\z_5$ existe y es $\sage{C.inverse()}$


% Fin del Ejercicio 3



\end{document}
