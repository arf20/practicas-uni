\documentclass[12pt]{amsart}
%\usepackage[usefamily=sage]{pythontex} 
\usepackage{sagetex}
\usepackage[utf8]{inputenc}
\usepackage[margin = 2cm]{geometry}
%\usepackage[spanish]{babel}
\usepackage[most]{tcolorbox}
\newtheorem{ejer}{Ejercicio}

\title{Álgebra y Matemática Discreta \\ Tarea de Aplicaciones Lineales }

\begin{document}
\maketitle

\begin{sageblock}
miDNI = 36363   # introduce aqui las 5 ultimas cifras de tu DNI o NIE
primo = ZZ(miDNI^2).next_prime()
EJ = [1+20*i+ZZ(Zmod(20)(primo^(i+2))) for i in range(4)]

# fijar primeros valores
EJ[0] = 10
EJ[1] = 38
EJ[2] = 42
EJ[3] = 74
\end{sageblock}

\begin{tcolorbox}[colback = orange!60!white]
Los ejercicios que tienes que realizar son al menos el Ejercicio~\sage{EJ[0]}, Ejercicio~\sage{EJ[1]}, Ejercicio~\sage{EJ[2]} y Ejercicio~\sage{EJ[3]}.
\end{tcolorbox}


\begin{ejer} Sea $K$ el cuerpo de 47 elementos.
\newline
\noindent\begin{minipage}{\textwidth}
\begin{tcolorbox}[colback = green!20!white,title=Versión Aplicación]
Calcula la aplicaci\'on lineal $f:K^{3} \to K^{3}$ que cumple las siguientes condiciones: 
\[f\left(\left[\begin{array}{r}
3 \\
28 \\
45
\end{array}\right]\right) = \left[\begin{array}{r}
1 \\
0 \\
0
\end{array}\right] \quad f\left(\left[\begin{array}{r}
7 \\
15 \\
17
\end{array}\right]\right) = \left[\begin{array}{r}
8 \\
44 \\
11
\end{array}\right] \quad f\left(\left[\begin{array}{r}
1 \\
15 \\
0
\end{array}\right]\right) = \left[\begin{array}{r}
11 \\
23 \\
20
\end{array}\right] \quad 
\]\end{tcolorbox}
\end{minipage} \newline
\noindent\begin{minipage}{\textwidth}
\begin{tcolorbox}[colback = blue!20!white,title=Versión Sistema Matricial]
Resuelve el siguiente sistema de ecuaciones matriciales despejando el valor de la matriz $M$
\[M \left[\begin{array}{r}
3 \\
28 \\
45
\end{array}\right] = \left[\begin{array}{r}
1 \\
0 \\
0
\end{array}\right] \quad M \left[\begin{array}{r}
7 \\
15 \\
17
\end{array}\right] = \left[\begin{array}{r}
8 \\
44 \\
11
\end{array}\right] \quad M \left[\begin{array}{r}
1 \\
15 \\
0
\end{array}\right] = \left[\begin{array}{r}
11 \\
23 \\
20
\end{array}\right] \quad 
\]\end{tcolorbox}
\end{minipage} \newline
\noindent\begin{minipage}{\textwidth} 
\begin{tcolorbox}[colback = red!20!white,title=Versión Ecuación Matricial]
Resuelve la ecuación matricial siguiente despejando el valor de la matriz $M$
\[M \left[\begin{array}{rrr}
3 & 7 & 1 \\
28 & 15 & 15 \\
45 & 17 & 0
\end{array}\right] = \left[\begin{array}{rrr}
1 & 8 & 11 \\
0 & 44 & 23 \\
0 & 11 & 20
\end{array}\right] \quad 
\]
\end{tcolorbox}
\end{minipage}%
\end{ejer}


{\it Soluci\'on:}
% Escribe tu soluci\'on para el Ejercicio 1

% Fin del Ejercicio 1


\begin{ejer} Sea $K$ el cuerpo de 7 elementos.
\newline
\noindent\begin{minipage}{\textwidth}
\begin{tcolorbox}[colback = green!20!white,title=Versión Aplicación]
Calcula la aplicaci\'on lineal $f:K^{3} \to K^{4}$ que cumple las siguientes condiciones: 
\[f\left(\left[\begin{array}{r}
0 \\
6 \\
1
\end{array}\right]\right) = \left[\begin{array}{r}
1 \\
0 \\
5 \\
5
\end{array}\right] \quad f\left(\left[\begin{array}{r}
4 \\
6 \\
5
\end{array}\right]\right) = \left[\begin{array}{r}
4 \\
5 \\
6 \\
0
\end{array}\right] \quad f\left(\left[\begin{array}{r}
1 \\
0 \\
3
\end{array}\right]\right) = \left[\begin{array}{r}
4 \\
2 \\
4 \\
4
\end{array}\right] \quad 
\]\end{tcolorbox}
\end{minipage} \newline
\noindent\begin{minipage}{\textwidth}
\begin{tcolorbox}[colback = blue!20!white,title=Versión Sistema Matricial]
Resuelve el siguiente sistema de ecuaciones matriciales despejando el valor de la matriz $M$
\[M \left[\begin{array}{r}
0 \\
6 \\
1
\end{array}\right] = \left[\begin{array}{r}
1 \\
0 \\
5 \\
5
\end{array}\right] \quad M \left[\begin{array}{r}
4 \\
6 \\
5
\end{array}\right] = \left[\begin{array}{r}
4 \\
5 \\
6 \\
0
\end{array}\right] \quad M \left[\begin{array}{r}
1 \\
0 \\
3
\end{array}\right] = \left[\begin{array}{r}
4 \\
2 \\
4 \\
4
\end{array}\right] \quad 
\]\end{tcolorbox}
\end{minipage} \newline
\noindent\begin{minipage}{\textwidth} 
\begin{tcolorbox}[colback = red!20!white,title=Versión Ecuación Matricial]
Resuelve la ecuación matricial siguiente despejando el valor de la matriz $M$
\[M \left[\begin{array}{rrr}
0 & 4 & 1 \\
6 & 6 & 0 \\
1 & 5 & 3
\end{array}\right] = \left[\begin{array}{rrr}
1 & 4 & 4 \\
0 & 5 & 2 \\
5 & 6 & 4 \\
5 & 0 & 4
\end{array}\right] \quad 
\]
\end{tcolorbox}
\end{minipage}%
\end{ejer}


{\it Soluci\'on:}
% Escribe tu soluci\'on para el Ejercicio 2

% Fin del Ejercicio 2


\begin{ejer} Sea $K$ el cuerpo de 31 elementos.
\newline
\noindent\begin{minipage}{\textwidth}
\begin{tcolorbox}[colback = green!20!white,title=Versión Aplicación]
Calcula la aplicaci\'on lineal $f:K^{3} \to K^{2}$ que cumple las siguientes condiciones: 
\[f\left(\left[\begin{array}{r}
19 \\
19 \\
6
\end{array}\right]\right) = \left[\begin{array}{r}
4 \\
2
\end{array}\right] \quad f\left(\left[\begin{array}{r}
10 \\
16 \\
27
\end{array}\right]\right) = \left[\begin{array}{r}
6 \\
17
\end{array}\right] \quad f\left(\left[\begin{array}{r}
18 \\
19 \\
20
\end{array}\right]\right) = \left[\begin{array}{r}
28 \\
11
\end{array}\right] \quad 
\]\end{tcolorbox}
\end{minipage} \newline
\noindent\begin{minipage}{\textwidth}
\begin{tcolorbox}[colback = blue!20!white,title=Versión Sistema Matricial]
Resuelve el siguiente sistema de ecuaciones matriciales despejando el valor de la matriz $M$
\[M \left[\begin{array}{r}
19 \\
19 \\
6
\end{array}\right] = \left[\begin{array}{r}
4 \\
2
\end{array}\right] \quad M \left[\begin{array}{r}
10 \\
16 \\
27
\end{array}\right] = \left[\begin{array}{r}
6 \\
17
\end{array}\right] \quad M \left[\begin{array}{r}
18 \\
19 \\
20
\end{array}\right] = \left[\begin{array}{r}
28 \\
11
\end{array}\right] \quad 
\]\end{tcolorbox}
\end{minipage} \newline
\noindent\begin{minipage}{\textwidth} 
\begin{tcolorbox}[colback = red!20!white,title=Versión Ecuación Matricial]
Resuelve la ecuación matricial siguiente despejando el valor de la matriz $M$
\[M \left[\begin{array}{rrr}
19 & 10 & 18 \\
19 & 16 & 19 \\
6 & 27 & 20
\end{array}\right] = \left[\begin{array}{rrr}
4 & 6 & 28 \\
2 & 17 & 11
\end{array}\right] \quad 
\]
\end{tcolorbox}
\end{minipage}%
\end{ejer}


{\it Soluci\'on:}
% Escribe tu soluci\'on para el Ejercicio 3

% Fin del Ejercicio 3


\begin{ejer} Sea $K$ el cuerpo de los n\'umeros reales.
\newline
\noindent\begin{minipage}{\textwidth}
\begin{tcolorbox}[colback = green!20!white,title=Versión Aplicación]
Calcula la aplicaci\'on lineal $f:K^{3} \to K^{4}$ que cumple las siguientes condiciones: 
\[f\left(\left[\begin{array}{r}
2 \\
1 \\
1
\end{array}\right]\right) = \left[\begin{array}{r}
-2 \\
0 \\
1 \\
0
\end{array}\right] \quad f\left(\left[\begin{array}{r}
1 \\
1 \\
0
\end{array}\right]\right) = \left[\begin{array}{r}
0 \\
-\frac{1}{2} \\
-1 \\
-2
\end{array}\right] \quad f\left(\left[\begin{array}{r}
5 \\
0 \\
6
\end{array}\right]\right) = \left[\begin{array}{r}
0 \\
\frac{1}{2} \\
-1 \\
0
\end{array}\right] \quad 
\]\end{tcolorbox}
\end{minipage} \newline
\noindent\begin{minipage}{\textwidth}
\begin{tcolorbox}[colback = blue!20!white,title=Versión Sistema Matricial]
Resuelve el siguiente sistema de ecuaciones matriciales despejando el valor de la matriz $M$
\[M \left[\begin{array}{r}
2 \\
1 \\
1
\end{array}\right] = \left[\begin{array}{r}
-2 \\
0 \\
1 \\
0
\end{array}\right] \quad M \left[\begin{array}{r}
1 \\
1 \\
0
\end{array}\right] = \left[\begin{array}{r}
0 \\
-\frac{1}{2} \\
-1 \\
-2
\end{array}\right] \quad M \left[\begin{array}{r}
5 \\
0 \\
6
\end{array}\right] = \left[\begin{array}{r}
0 \\
\frac{1}{2} \\
-1 \\
0
\end{array}\right] \quad 
\]\end{tcolorbox}
\end{minipage} \newline
\noindent\begin{minipage}{\textwidth} 
\begin{tcolorbox}[colback = red!20!white,title=Versión Ecuación Matricial]
Resuelve la ecuación matricial siguiente despejando el valor de la matriz $M$
\[M \left[\begin{array}{rrr}
2 & 1 & 5 \\
1 & 1 & 0 \\
1 & 0 & 6
\end{array}\right] = \left[\begin{array}{rrr}
-2 & 0 & 0 \\
0 & -\frac{1}{2} & \frac{1}{2} \\
1 & -1 & -1 \\
0 & -2 & 0
\end{array}\right] \quad 
\]
\end{tcolorbox}
\end{minipage}%
\end{ejer}


{\it Soluci\'on:}
% Escribe tu soluci\'on para el Ejercicio 4

% Fin del Ejercicio 4


\begin{ejer} Sea $K$ el cuerpo de 47 elementos.
\newline
\noindent\begin{minipage}{\textwidth}
\begin{tcolorbox}[colback = green!20!white,title=Versión Aplicación]
Calcula la aplicaci\'on lineal $f:K^{3} \to K^{2}$ que cumple las siguientes condiciones: 
\[f\left(\left[\begin{array}{r}
18 \\
38 \\
26
\end{array}\right]\right) = \left[\begin{array}{r}
16 \\
15
\end{array}\right] \quad f\left(\left[\begin{array}{r}
40 \\
14 \\
44
\end{array}\right]\right) = \left[\begin{array}{r}
7 \\
16
\end{array}\right] \quad f\left(\left[\begin{array}{r}
22 \\
36 \\
38
\end{array}\right]\right) = \left[\begin{array}{r}
32 \\
43
\end{array}\right] \quad 
\]\end{tcolorbox}
\end{minipage} \newline
\noindent\begin{minipage}{\textwidth}
\begin{tcolorbox}[colback = blue!20!white,title=Versión Sistema Matricial]
Resuelve el siguiente sistema de ecuaciones matriciales despejando el valor de la matriz $M$
\[M \left[\begin{array}{r}
18 \\
38 \\
26
\end{array}\right] = \left[\begin{array}{r}
16 \\
15
\end{array}\right] \quad M \left[\begin{array}{r}
40 \\
14 \\
44
\end{array}\right] = \left[\begin{array}{r}
7 \\
16
\end{array}\right] \quad M \left[\begin{array}{r}
22 \\
36 \\
38
\end{array}\right] = \left[\begin{array}{r}
32 \\
43
\end{array}\right] \quad 
\]\end{tcolorbox}
\end{minipage} \newline
\noindent\begin{minipage}{\textwidth} 
\begin{tcolorbox}[colback = red!20!white,title=Versión Ecuación Matricial]
Resuelve la ecuación matricial siguiente despejando el valor de la matriz $M$
\[M \left[\begin{array}{rrr}
18 & 40 & 22 \\
38 & 14 & 36 \\
26 & 44 & 38
\end{array}\right] = \left[\begin{array}{rrr}
16 & 7 & 32 \\
15 & 16 & 43
\end{array}\right] \quad 
\]
\end{tcolorbox}
\end{minipage}%
\end{ejer}


{\it Soluci\'on:}
% Escribe tu soluci\'on para el Ejercicio 5

% Fin del Ejercicio 5


\begin{ejer} Sea $K$ el cuerpo de los n\'umeros reales.
\newline
\noindent\begin{minipage}{\textwidth}
\begin{tcolorbox}[colback = green!20!white,title=Versión Aplicación]
Calcula la aplicaci\'on lineal $f:K^{2} \to K^{4}$ que cumple las siguientes condiciones: 
\[f\left(\left[\begin{array}{r}
1 \\
-5
\end{array}\right]\right) = \left[\begin{array}{r}
\frac{1}{2} \\
0 \\
-\frac{1}{2} \\
0
\end{array}\right] \quad f\left(\left[\begin{array}{r}
2 \\
-9
\end{array}\right]\right) = \left[\begin{array}{r}
-2 \\
-1 \\
0 \\
-1
\end{array}\right] \quad 
\]\end{tcolorbox}
\end{minipage} \newline
\noindent\begin{minipage}{\textwidth}
\begin{tcolorbox}[colback = blue!20!white,title=Versión Sistema Matricial]
Resuelve el siguiente sistema de ecuaciones matriciales despejando el valor de la matriz $M$
\[M \left[\begin{array}{r}
1 \\
-5
\end{array}\right] = \left[\begin{array}{r}
\frac{1}{2} \\
0 \\
-\frac{1}{2} \\
0
\end{array}\right] \quad M \left[\begin{array}{r}
2 \\
-9
\end{array}\right] = \left[\begin{array}{r}
-2 \\
-1 \\
0 \\
-1
\end{array}\right] \quad 
\]\end{tcolorbox}
\end{minipage} \newline
\noindent\begin{minipage}{\textwidth} 
\begin{tcolorbox}[colback = red!20!white,title=Versión Ecuación Matricial]
Resuelve la ecuación matricial siguiente despejando el valor de la matriz $M$
\[M \left[\begin{array}{rr}
1 & 2 \\
-5 & -9
\end{array}\right] = \left[\begin{array}{rr}
\frac{1}{2} & -2 \\
0 & -1 \\
-\frac{1}{2} & 0 \\
0 & -1
\end{array}\right] \quad 
\]
\end{tcolorbox}
\end{minipage}%
\end{ejer}


{\it Soluci\'on:}
% Escribe tu soluci\'on para el Ejercicio 6

% Fin del Ejercicio 6


\begin{ejer} Sea $K$ el cuerpo de los n\'umeros reales.
\newline
\noindent\begin{minipage}{\textwidth}
\begin{tcolorbox}[colback = green!20!white,title=Versión Aplicación]
Calcula la aplicaci\'on lineal $f:K^{2} \to K^{3}$ que cumple las siguientes condiciones: 
\[f\left(\left[\begin{array}{r}
1 \\
0
\end{array}\right]\right) = \left[\begin{array}{r}
0 \\
\frac{1}{2} \\
0
\end{array}\right] \quad f\left(\left[\begin{array}{r}
-2 \\
1
\end{array}\right]\right) = \left[\begin{array}{r}
-\frac{1}{2} \\
\frac{1}{2} \\
0
\end{array}\right] \quad 
\]\end{tcolorbox}
\end{minipage} \newline
\noindent\begin{minipage}{\textwidth}
\begin{tcolorbox}[colback = blue!20!white,title=Versión Sistema Matricial]
Resuelve el siguiente sistema de ecuaciones matriciales despejando el valor de la matriz $M$
\[M \left[\begin{array}{r}
1 \\
0
\end{array}\right] = \left[\begin{array}{r}
0 \\
\frac{1}{2} \\
0
\end{array}\right] \quad M \left[\begin{array}{r}
-2 \\
1
\end{array}\right] = \left[\begin{array}{r}
-\frac{1}{2} \\
\frac{1}{2} \\
0
\end{array}\right] \quad 
\]\end{tcolorbox}
\end{minipage} \newline
\noindent\begin{minipage}{\textwidth} 
\begin{tcolorbox}[colback = red!20!white,title=Versión Ecuación Matricial]
Resuelve la ecuación matricial siguiente despejando el valor de la matriz $M$
\[M \left[\begin{array}{rr}
1 & -2 \\
0 & 1
\end{array}\right] = \left[\begin{array}{rr}
0 & -\frac{1}{2} \\
\frac{1}{2} & \frac{1}{2} \\
0 & 0
\end{array}\right] \quad 
\]
\end{tcolorbox}
\end{minipage}%
\end{ejer}


{\it Soluci\'on:}
% Escribe tu soluci\'on para el Ejercicio 7

% Fin del Ejercicio 7


\begin{ejer} Sea $K$ el cuerpo de 29 elementos.
\newline
\noindent\begin{minipage}{\textwidth}
\begin{tcolorbox}[colback = green!20!white,title=Versión Aplicación]
Calcula la aplicaci\'on lineal $f:K^{2} \to K^{2}$ que cumple las siguientes condiciones: 
\[f\left(\left[\begin{array}{r}
27 \\
24
\end{array}\right]\right) = \left[\begin{array}{r}
3 \\
5
\end{array}\right] \quad f\left(\left[\begin{array}{r}
7 \\
17
\end{array}\right]\right) = \left[\begin{array}{r}
3 \\
18
\end{array}\right] \quad 
\]\end{tcolorbox}
\end{minipage} \newline
\noindent\begin{minipage}{\textwidth}
\begin{tcolorbox}[colback = blue!20!white,title=Versión Sistema Matricial]
Resuelve el siguiente sistema de ecuaciones matriciales despejando el valor de la matriz $M$
\[M \left[\begin{array}{r}
27 \\
24
\end{array}\right] = \left[\begin{array}{r}
3 \\
5
\end{array}\right] \quad M \left[\begin{array}{r}
7 \\
17
\end{array}\right] = \left[\begin{array}{r}
3 \\
18
\end{array}\right] \quad 
\]\end{tcolorbox}
\end{minipage} \newline
\noindent\begin{minipage}{\textwidth} 
\begin{tcolorbox}[colback = red!20!white,title=Versión Ecuación Matricial]
Resuelve la ecuación matricial siguiente despejando el valor de la matriz $M$
\[M \left[\begin{array}{rr}
27 & 7 \\
24 & 17
\end{array}\right] = \left[\begin{array}{rr}
3 & 3 \\
5 & 18
\end{array}\right] \quad 
\]
\end{tcolorbox}
\end{minipage}%
\end{ejer}


{\it Soluci\'on:}
% Escribe tu soluci\'on para el Ejercicio 8

% Fin del Ejercicio 8


\begin{ejer} Sea $K$ el cuerpo de 13 elementos.
\newline
\noindent\begin{minipage}{\textwidth}
\begin{tcolorbox}[colback = green!20!white,title=Versión Aplicación]
Calcula la aplicaci\'on lineal $f:K^{2} \to K^{3}$ que cumple las siguientes condiciones: 
\[f\left(\left[\begin{array}{r}
0 \\
2
\end{array}\right]\right) = \left[\begin{array}{r}
7 \\
4 \\
7
\end{array}\right] \quad f\left(\left[\begin{array}{r}
6 \\
9
\end{array}\right]\right) = \left[\begin{array}{r}
1 \\
10 \\
10
\end{array}\right] \quad 
\]\end{tcolorbox}
\end{minipage} \newline
\noindent\begin{minipage}{\textwidth}
\begin{tcolorbox}[colback = blue!20!white,title=Versión Sistema Matricial]
Resuelve el siguiente sistema de ecuaciones matriciales despejando el valor de la matriz $M$
\[M \left[\begin{array}{r}
0 \\
2
\end{array}\right] = \left[\begin{array}{r}
7 \\
4 \\
7
\end{array}\right] \quad M \left[\begin{array}{r}
6 \\
9
\end{array}\right] = \left[\begin{array}{r}
1 \\
10 \\
10
\end{array}\right] \quad 
\]\end{tcolorbox}
\end{minipage} \newline
\noindent\begin{minipage}{\textwidth} 
\begin{tcolorbox}[colback = red!20!white,title=Versión Ecuación Matricial]
Resuelve la ecuación matricial siguiente despejando el valor de la matriz $M$
\[M \left[\begin{array}{rr}
0 & 6 \\
2 & 9
\end{array}\right] = \left[\begin{array}{rr}
7 & 1 \\
4 & 10 \\
7 & 10
\end{array}\right] \quad 
\]
\end{tcolorbox}
\end{minipage}%
\end{ejer}


{\it Soluci\'on:}
% Escribe tu soluci\'on para el Ejercicio 9

% Fin del Ejercicio 9


\begin{ejer} Sea $K$ el cuerpo de 19 elementos.
\newline
\noindent\begin{minipage}{\textwidth}
\begin{tcolorbox}[colback = green!20!white,title=Versión Aplicación]
Calcula la aplicaci\'on lineal $f:K^{2} \to K^{4}$ que cumple las siguientes condiciones: 
\[f\left(\left[\begin{array}{r}
9 \\
13
\end{array}\right]\right) = \left[\begin{array}{r}
17 \\
1 \\
6 \\
12
\end{array}\right] \quad f\left(\left[\begin{array}{r}
10 \\
4
\end{array}\right]\right) = \left[\begin{array}{r}
0 \\
10 \\
3 \\
18
\end{array}\right] \quad 
\]\end{tcolorbox}
\end{minipage} \newline
\noindent\begin{minipage}{\textwidth}
\begin{tcolorbox}[colback = blue!20!white,title=Versión Sistema Matricial]
Resuelve el siguiente sistema de ecuaciones matriciales despejando el valor de la matriz $M$
\[M \left[\begin{array}{r}
9 \\
13
\end{array}\right] = \left[\begin{array}{r}
17 \\
1 \\
6 \\
12
\end{array}\right] \quad M \left[\begin{array}{r}
10 \\
4
\end{array}\right] = \left[\begin{array}{r}
0 \\
10 \\
3 \\
18
\end{array}\right] \quad 
\]\end{tcolorbox}
\end{minipage} \newline
\noindent\begin{minipage}{\textwidth} 
\begin{tcolorbox}[colback = red!20!white,title=Versión Ecuación Matricial]
Resuelve la ecuación matricial siguiente despejando el valor de la matriz $M$
\[M \left[\begin{array}{rr}
9 & 10 \\
13 & 4
\end{array}\right] = \left[\begin{array}{rr}
17 & 0 \\
1 & 10 \\
6 & 3 \\
12 & 18
\end{array}\right] \quad 
\]
\end{tcolorbox}
\end{minipage}%
\end{ejer}


{\it Soluci\'on:}
% Escribe tu soluci\'on para el Ejercicio 10

\begin{sageblock}
	A = matrix(Zmod(19), [[9, 10], [13, 4]])
	B = matrix(Zmod(19), [[9, 10], [13, 4], [12, 18]])
	Ap = block_matrix([[A, 1]])
	Ai = Ap.echelon_form().subdivision(0, 1)
\end{sageblock}

$$
	M = \sage{B} \sage{A}^{-1}
$$

$$
	M = \sage{B} \sage{Ai}
$$

$$
	M = \sage{B * Ai}
$$

% Fin del Ejercicio 10


\begin{ejer} Sea $K$ el cuerpo de los n\'umeros reales.
\newline
\noindent\begin{minipage}{\textwidth}
\begin{tcolorbox}[colback = green!20!white,title=Versión Aplicación]
Calcula la aplicaci\'on lineal $f:K^{3} \to K^{4}$ que cumple las siguientes condiciones: 
\[f\left(\left[\begin{array}{r}
-1 \\
2 \\
1
\end{array}\right]\right) = \left[\begin{array}{r}
0 \\
0 \\
2 \\
-1
\end{array}\right] \quad f\left(\left[\begin{array}{r}
0 \\
-1 \\
-1
\end{array}\right]\right) = \left[\begin{array}{r}
-1 \\
0 \\
0 \\
0
\end{array}\right] \quad f\left(\left[\begin{array}{r}
2 \\
-8 \\
-5
\end{array}\right]\right) = \left[\begin{array}{r}
0 \\
-2 \\
0 \\
-1
\end{array}\right] \quad 
\]\end{tcolorbox}
\end{minipage} \newline
\noindent\begin{minipage}{\textwidth}
\begin{tcolorbox}[colback = blue!20!white,title=Versión Sistema Matricial]
Resuelve el siguiente sistema de ecuaciones matriciales despejando el valor de la matriz $M$
\[M \left[\begin{array}{r}
-1 \\
2 \\
1
\end{array}\right] = \left[\begin{array}{r}
0 \\
0 \\
2 \\
-1
\end{array}\right] \quad M \left[\begin{array}{r}
0 \\
-1 \\
-1
\end{array}\right] = \left[\begin{array}{r}
-1 \\
0 \\
0 \\
0
\end{array}\right] \quad M \left[\begin{array}{r}
2 \\
-8 \\
-5
\end{array}\right] = \left[\begin{array}{r}
0 \\
-2 \\
0 \\
-1
\end{array}\right] \quad 
\]\end{tcolorbox}
\end{minipage} \newline
\noindent\begin{minipage}{\textwidth} 
\begin{tcolorbox}[colback = red!20!white,title=Versión Ecuación Matricial]
Resuelve la ecuación matricial siguiente despejando el valor de la matriz $M$
\[M \left[\begin{array}{rrr}
-1 & 0 & 2 \\
2 & -1 & -8 \\
1 & -1 & -5
\end{array}\right] = \left[\begin{array}{rrr}
0 & -1 & 0 \\
0 & 0 & -2 \\
2 & 0 & 0 \\
-1 & 0 & -1
\end{array}\right] \quad 
\]
\end{tcolorbox}
\end{minipage}%
\end{ejer}


{\it Soluci\'on:}
% Escribe tu soluci\'on para el Ejercicio 11

% Fin del Ejercicio 11


\begin{ejer} Sea $K$ el cuerpo de 31 elementos.
\newline
\noindent\begin{minipage}{\textwidth}
\begin{tcolorbox}[colback = green!20!white,title=Versión Aplicación]
Calcula la aplicaci\'on lineal $f:K^{3} \to K^{3}$ que cumple las siguientes condiciones: 
\[f\left(\left[\begin{array}{r}
2 \\
30 \\
5
\end{array}\right]\right) = \left[\begin{array}{r}
21 \\
2 \\
3
\end{array}\right] \quad f\left(\left[\begin{array}{r}
1 \\
0 \\
0
\end{array}\right]\right) = \left[\begin{array}{r}
6 \\
28 \\
17
\end{array}\right] \quad f\left(\left[\begin{array}{r}
9 \\
10 \\
13
\end{array}\right]\right) = \left[\begin{array}{r}
23 \\
14 \\
12
\end{array}\right] \quad 
\]\end{tcolorbox}
\end{minipage} \newline
\noindent\begin{minipage}{\textwidth}
\begin{tcolorbox}[colback = blue!20!white,title=Versión Sistema Matricial]
Resuelve el siguiente sistema de ecuaciones matriciales despejando el valor de la matriz $M$
\[M \left[\begin{array}{r}
2 \\
30 \\
5
\end{array}\right] = \left[\begin{array}{r}
21 \\
2 \\
3
\end{array}\right] \quad M \left[\begin{array}{r}
1 \\
0 \\
0
\end{array}\right] = \left[\begin{array}{r}
6 \\
28 \\
17
\end{array}\right] \quad M \left[\begin{array}{r}
9 \\
10 \\
13
\end{array}\right] = \left[\begin{array}{r}
23 \\
14 \\
12
\end{array}\right] \quad 
\]\end{tcolorbox}
\end{minipage} \newline
\noindent\begin{minipage}{\textwidth} 
\begin{tcolorbox}[colback = red!20!white,title=Versión Ecuación Matricial]
Resuelve la ecuación matricial siguiente despejando el valor de la matriz $M$
\[M \left[\begin{array}{rrr}
2 & 1 & 9 \\
30 & 0 & 10 \\
5 & 0 & 13
\end{array}\right] = \left[\begin{array}{rrr}
21 & 6 & 23 \\
2 & 28 & 14 \\
3 & 17 & 12
\end{array}\right] \quad 
\]
\end{tcolorbox}
\end{minipage}%
\end{ejer}


{\it Soluci\'on:}
% Escribe tu soluci\'on para el Ejercicio 12

% Fin del Ejercicio 12


\begin{ejer} Sea $K$ el cuerpo de los n\'umeros reales.
\newline
\noindent\begin{minipage}{\textwidth}
\begin{tcolorbox}[colback = green!20!white,title=Versión Aplicación]
Calcula la aplicaci\'on lineal $f:K^{3} \to K^{2}$ que cumple las siguientes condiciones: 
\[f\left(\left[\begin{array}{r}
1 \\
0 \\
-1
\end{array}\right]\right) = \left[\begin{array}{r}
\frac{1}{2} \\
2
\end{array}\right] \quad f\left(\left[\begin{array}{r}
0 \\
1 \\
-2
\end{array}\right]\right) = \left[\begin{array}{r}
1 \\
0
\end{array}\right] \quad f\left(\left[\begin{array}{r}
5 \\
-5 \\
6
\end{array}\right]\right) = \left[\begin{array}{r}
0 \\
1
\end{array}\right] \quad 
\]\end{tcolorbox}
\end{minipage} \newline
\noindent\begin{minipage}{\textwidth}
\begin{tcolorbox}[colback = blue!20!white,title=Versión Sistema Matricial]
Resuelve el siguiente sistema de ecuaciones matriciales despejando el valor de la matriz $M$
\[M \left[\begin{array}{r}
1 \\
0 \\
-1
\end{array}\right] = \left[\begin{array}{r}
\frac{1}{2} \\
2
\end{array}\right] \quad M \left[\begin{array}{r}
0 \\
1 \\
-2
\end{array}\right] = \left[\begin{array}{r}
1 \\
0
\end{array}\right] \quad M \left[\begin{array}{r}
5 \\
-5 \\
6
\end{array}\right] = \left[\begin{array}{r}
0 \\
1
\end{array}\right] \quad 
\]\end{tcolorbox}
\end{minipage} \newline
\noindent\begin{minipage}{\textwidth} 
\begin{tcolorbox}[colback = red!20!white,title=Versión Ecuación Matricial]
Resuelve la ecuación matricial siguiente despejando el valor de la matriz $M$
\[M \left[\begin{array}{rrr}
1 & 0 & 5 \\
0 & 1 & -5 \\
-1 & -2 & 6
\end{array}\right] = \left[\begin{array}{rrr}
\frac{1}{2} & 1 & 0 \\
2 & 0 & 1
\end{array}\right] \quad 
\]
\end{tcolorbox}
\end{minipage}%
\end{ejer}


{\it Soluci\'on:}
% Escribe tu soluci\'on para el Ejercicio 13

% Fin del Ejercicio 13


\begin{ejer} Sea $K$ el cuerpo de los n\'umeros reales.
\newline
\noindent\begin{minipage}{\textwidth}
\begin{tcolorbox}[colback = green!20!white,title=Versión Aplicación]
Calcula la aplicaci\'on lineal $f:K^{2} \to K^{2}$ que cumple las siguientes condiciones: 
\[f\left(\left[\begin{array}{r}
1 \\
-2
\end{array}\right]\right) = \left[\begin{array}{r}
0 \\
2
\end{array}\right] \quad f\left(\left[\begin{array}{r}
-4 \\
9
\end{array}\right]\right) = \left[\begin{array}{r}
-2 \\
-1
\end{array}\right] \quad 
\]\end{tcolorbox}
\end{minipage} \newline
\noindent\begin{minipage}{\textwidth}
\begin{tcolorbox}[colback = blue!20!white,title=Versión Sistema Matricial]
Resuelve el siguiente sistema de ecuaciones matriciales despejando el valor de la matriz $M$
\[M \left[\begin{array}{r}
1 \\
-2
\end{array}\right] = \left[\begin{array}{r}
0 \\
2
\end{array}\right] \quad M \left[\begin{array}{r}
-4 \\
9
\end{array}\right] = \left[\begin{array}{r}
-2 \\
-1
\end{array}\right] \quad 
\]\end{tcolorbox}
\end{minipage} \newline
\noindent\begin{minipage}{\textwidth} 
\begin{tcolorbox}[colback = red!20!white,title=Versión Ecuación Matricial]
Resuelve la ecuación matricial siguiente despejando el valor de la matriz $M$
\[M \left[\begin{array}{rr}
1 & -4 \\
-2 & 9
\end{array}\right] = \left[\begin{array}{rr}
0 & -2 \\
2 & -1
\end{array}\right] \quad 
\]
\end{tcolorbox}
\end{minipage}%
\end{ejer}


{\it Soluci\'on:}
% Escribe tu soluci\'on para el Ejercicio 14

% Fin del Ejercicio 14


\begin{ejer} Sea $K$ el cuerpo de 7 elementos.
\newline
\noindent\begin{minipage}{\textwidth}
\begin{tcolorbox}[colback = green!20!white,title=Versión Aplicación]
Calcula la aplicaci\'on lineal $f:K^{2} \to K^{4}$ que cumple las siguientes condiciones: 
\[f\left(\left[\begin{array}{r}
3 \\
6
\end{array}\right]\right) = \left[\begin{array}{r}
6 \\
0 \\
3 \\
4
\end{array}\right] \quad f\left(\left[\begin{array}{r}
6 \\
3
\end{array}\right]\right) = \left[\begin{array}{r}
5 \\
6 \\
6 \\
4
\end{array}\right] \quad 
\]\end{tcolorbox}
\end{minipage} \newline
\noindent\begin{minipage}{\textwidth}
\begin{tcolorbox}[colback = blue!20!white,title=Versión Sistema Matricial]
Resuelve el siguiente sistema de ecuaciones matriciales despejando el valor de la matriz $M$
\[M \left[\begin{array}{r}
3 \\
6
\end{array}\right] = \left[\begin{array}{r}
6 \\
0 \\
3 \\
4
\end{array}\right] \quad M \left[\begin{array}{r}
6 \\
3
\end{array}\right] = \left[\begin{array}{r}
5 \\
6 \\
6 \\
4
\end{array}\right] \quad 
\]\end{tcolorbox}
\end{minipage} \newline
\noindent\begin{minipage}{\textwidth} 
\begin{tcolorbox}[colback = red!20!white,title=Versión Ecuación Matricial]
Resuelve la ecuación matricial siguiente despejando el valor de la matriz $M$
\[M \left[\begin{array}{rr}
3 & 6 \\
6 & 3
\end{array}\right] = \left[\begin{array}{rr}
6 & 5 \\
0 & 6 \\
3 & 6 \\
4 & 4
\end{array}\right] \quad 
\]
\end{tcolorbox}
\end{minipage}%
\end{ejer}


{\it Soluci\'on:}
% Escribe tu soluci\'on para el Ejercicio 15

% Fin del Ejercicio 15


\begin{ejer} Sea $K$ el cuerpo de los n\'umeros reales.
\newline
\noindent\begin{minipage}{\textwidth}
\begin{tcolorbox}[colback = green!20!white,title=Versión Aplicación]
Calcula la aplicaci\'on lineal $f:K^{3} \to K^{4}$ que cumple las siguientes condiciones: 
\[f\left(\left[\begin{array}{r}
-3 \\
2 \\
1
\end{array}\right]\right) = \left[\begin{array}{r}
-2 \\
2 \\
-1 \\
0
\end{array}\right] \quad f\left(\left[\begin{array}{r}
7 \\
-5 \\
2
\end{array}\right]\right) = \left[\begin{array}{r}
1 \\
2 \\
\frac{1}{2} \\
2
\end{array}\right] \quad f\left(\left[\begin{array}{r}
-8 \\
5 \\
8
\end{array}\right]\right) = \left[\begin{array}{r}
-1 \\
-\frac{1}{2} \\
\frac{1}{2} \\
\frac{1}{2}
\end{array}\right] \quad 
\]\end{tcolorbox}
\end{minipage} \newline
\noindent\begin{minipage}{\textwidth}
\begin{tcolorbox}[colback = blue!20!white,title=Versión Sistema Matricial]
Resuelve el siguiente sistema de ecuaciones matriciales despejando el valor de la matriz $M$
\[M \left[\begin{array}{r}
-3 \\
2 \\
1
\end{array}\right] = \left[\begin{array}{r}
-2 \\
2 \\
-1 \\
0
\end{array}\right] \quad M \left[\begin{array}{r}
7 \\
-5 \\
2
\end{array}\right] = \left[\begin{array}{r}
1 \\
2 \\
\frac{1}{2} \\
2
\end{array}\right] \quad M \left[\begin{array}{r}
-8 \\
5 \\
8
\end{array}\right] = \left[\begin{array}{r}
-1 \\
-\frac{1}{2} \\
\frac{1}{2} \\
\frac{1}{2}
\end{array}\right] \quad 
\]\end{tcolorbox}
\end{minipage} \newline
\noindent\begin{minipage}{\textwidth} 
\begin{tcolorbox}[colback = red!20!white,title=Versión Ecuación Matricial]
Resuelve la ecuación matricial siguiente despejando el valor de la matriz $M$
\[M \left[\begin{array}{rrr}
-3 & 7 & -8 \\
2 & -5 & 5 \\
1 & 2 & 8
\end{array}\right] = \left[\begin{array}{rrr}
-2 & 1 & -1 \\
2 & 2 & -\frac{1}{2} \\
-1 & \frac{1}{2} & \frac{1}{2} \\
0 & 2 & \frac{1}{2}
\end{array}\right] \quad 
\]
\end{tcolorbox}
\end{minipage}%
\end{ejer}


{\it Soluci\'on:}
% Escribe tu soluci\'on para el Ejercicio 16

% Fin del Ejercicio 16


\begin{ejer} Sea $K$ el cuerpo de 17 elementos.
\newline
\noindent\begin{minipage}{\textwidth}
\begin{tcolorbox}[colback = green!20!white,title=Versión Aplicación]
Calcula la aplicaci\'on lineal $f:K^{2} \to K^{2}$ que cumple las siguientes condiciones: 
\[f\left(\left[\begin{array}{r}
4 \\
14
\end{array}\right]\right) = \left[\begin{array}{r}
3 \\
6
\end{array}\right] \quad f\left(\left[\begin{array}{r}
11 \\
9
\end{array}\right]\right) = \left[\begin{array}{r}
6 \\
8
\end{array}\right] \quad 
\]\end{tcolorbox}
\end{minipage} \newline
\noindent\begin{minipage}{\textwidth}
\begin{tcolorbox}[colback = blue!20!white,title=Versión Sistema Matricial]
Resuelve el siguiente sistema de ecuaciones matriciales despejando el valor de la matriz $M$
\[M \left[\begin{array}{r}
4 \\
14
\end{array}\right] = \left[\begin{array}{r}
3 \\
6
\end{array}\right] \quad M \left[\begin{array}{r}
11 \\
9
\end{array}\right] = \left[\begin{array}{r}
6 \\
8
\end{array}\right] \quad 
\]\end{tcolorbox}
\end{minipage} \newline
\noindent\begin{minipage}{\textwidth} 
\begin{tcolorbox}[colback = red!20!white,title=Versión Ecuación Matricial]
Resuelve la ecuación matricial siguiente despejando el valor de la matriz $M$
\[M \left[\begin{array}{rr}
4 & 11 \\
14 & 9
\end{array}\right] = \left[\begin{array}{rr}
3 & 6 \\
6 & 8
\end{array}\right] \quad 
\]
\end{tcolorbox}
\end{minipage}%
\end{ejer}


{\it Soluci\'on:}
% Escribe tu soluci\'on para el Ejercicio 17

% Fin del Ejercicio 17


\begin{ejer} Sea $K$ el cuerpo de 47 elementos.
\newline
\noindent\begin{minipage}{\textwidth}
\begin{tcolorbox}[colback = green!20!white,title=Versión Aplicación]
Calcula la aplicaci\'on lineal $f:K^{3} \to K^{2}$ que cumple las siguientes condiciones: 
\[f\left(\left[\begin{array}{r}
11 \\
35 \\
42
\end{array}\right]\right) = \left[\begin{array}{r}
10 \\
24
\end{array}\right] \quad f\left(\left[\begin{array}{r}
29 \\
24 \\
10
\end{array}\right]\right) = \left[\begin{array}{r}
41 \\
46
\end{array}\right] \quad f\left(\left[\begin{array}{r}
20 \\
34 \\
23
\end{array}\right]\right) = \left[\begin{array}{r}
11 \\
29
\end{array}\right] \quad 
\]\end{tcolorbox}
\end{minipage} \newline
\noindent\begin{minipage}{\textwidth}
\begin{tcolorbox}[colback = blue!20!white,title=Versión Sistema Matricial]
Resuelve el siguiente sistema de ecuaciones matriciales despejando el valor de la matriz $M$
\[M \left[\begin{array}{r}
11 \\
35 \\
42
\end{array}\right] = \left[\begin{array}{r}
10 \\
24
\end{array}\right] \quad M \left[\begin{array}{r}
29 \\
24 \\
10
\end{array}\right] = \left[\begin{array}{r}
41 \\
46
\end{array}\right] \quad M \left[\begin{array}{r}
20 \\
34 \\
23
\end{array}\right] = \left[\begin{array}{r}
11 \\
29
\end{array}\right] \quad 
\]\end{tcolorbox}
\end{minipage} \newline
\noindent\begin{minipage}{\textwidth} 
\begin{tcolorbox}[colback = red!20!white,title=Versión Ecuación Matricial]
Resuelve la ecuación matricial siguiente despejando el valor de la matriz $M$
\[M \left[\begin{array}{rrr}
11 & 29 & 20 \\
35 & 24 & 34 \\
42 & 10 & 23
\end{array}\right] = \left[\begin{array}{rrr}
10 & 41 & 11 \\
24 & 46 & 29
\end{array}\right] \quad 
\]
\end{tcolorbox}
\end{minipage}%
\end{ejer}


{\it Soluci\'on:}
% Escribe tu soluci\'on para el Ejercicio 18

% Fin del Ejercicio 18


\begin{ejer} Sea $K$ el cuerpo de los n\'umeros reales.
\newline
\noindent\begin{minipage}{\textwidth}
\begin{tcolorbox}[colback = green!20!white,title=Versión Aplicación]
Calcula la aplicaci\'on lineal $f:K^{2} \to K^{2}$ que cumple las siguientes condiciones: 
\[f\left(\left[\begin{array}{r}
-1 \\
-2
\end{array}\right]\right) = \left[\begin{array}{r}
-2 \\
2
\end{array}\right] \quad f\left(\left[\begin{array}{r}
-3 \\
-7
\end{array}\right]\right) = \left[\begin{array}{r}
1 \\
0
\end{array}\right] \quad 
\]\end{tcolorbox}
\end{minipage} \newline
\noindent\begin{minipage}{\textwidth}
\begin{tcolorbox}[colback = blue!20!white,title=Versión Sistema Matricial]
Resuelve el siguiente sistema de ecuaciones matriciales despejando el valor de la matriz $M$
\[M \left[\begin{array}{r}
-1 \\
-2
\end{array}\right] = \left[\begin{array}{r}
-2 \\
2
\end{array}\right] \quad M \left[\begin{array}{r}
-3 \\
-7
\end{array}\right] = \left[\begin{array}{r}
1 \\
0
\end{array}\right] \quad 
\]\end{tcolorbox}
\end{minipage} \newline
\noindent\begin{minipage}{\textwidth} 
\begin{tcolorbox}[colback = red!20!white,title=Versión Ecuación Matricial]
Resuelve la ecuación matricial siguiente despejando el valor de la matriz $M$
\[M \left[\begin{array}{rr}
-1 & -3 \\
-2 & -7
\end{array}\right] = \left[\begin{array}{rr}
-2 & 1 \\
2 & 0
\end{array}\right] \quad 
\]
\end{tcolorbox}
\end{minipage}%
\end{ejer}


{\it Soluci\'on:}
% Escribe tu soluci\'on para el Ejercicio 19

% Fin del Ejercicio 19


\begin{ejer} Sea $K$ el cuerpo de los n\'umeros reales.
\newline
\noindent\begin{minipage}{\textwidth}
\begin{tcolorbox}[colback = green!20!white,title=Versión Aplicación]
Calcula la aplicaci\'on lineal $f:K^{3} \to K^{4}$ que cumple las siguientes condiciones: 
\[f\left(\left[\begin{array}{r}
-1 \\
0 \\
-2
\end{array}\right]\right) = \left[\begin{array}{r}
0 \\
2 \\
1 \\
-2
\end{array}\right] \quad f\left(\left[\begin{array}{r}
0 \\
1 \\
2
\end{array}\right]\right) = \left[\begin{array}{r}
-\frac{1}{2} \\
2 \\
-1 \\
0
\end{array}\right] \quad f\left(\left[\begin{array}{r}
-1 \\
-3 \\
-9
\end{array}\right]\right) = \left[\begin{array}{r}
-1 \\
-1 \\
\frac{1}{2} \\
1
\end{array}\right] \quad 
\]\end{tcolorbox}
\end{minipage} \newline
\noindent\begin{minipage}{\textwidth}
\begin{tcolorbox}[colback = blue!20!white,title=Versión Sistema Matricial]
Resuelve el siguiente sistema de ecuaciones matriciales despejando el valor de la matriz $M$
\[M \left[\begin{array}{r}
-1 \\
0 \\
-2
\end{array}\right] = \left[\begin{array}{r}
0 \\
2 \\
1 \\
-2
\end{array}\right] \quad M \left[\begin{array}{r}
0 \\
1 \\
2
\end{array}\right] = \left[\begin{array}{r}
-\frac{1}{2} \\
2 \\
-1 \\
0
\end{array}\right] \quad M \left[\begin{array}{r}
-1 \\
-3 \\
-9
\end{array}\right] = \left[\begin{array}{r}
-1 \\
-1 \\
\frac{1}{2} \\
1
\end{array}\right] \quad 
\]\end{tcolorbox}
\end{minipage} \newline
\noindent\begin{minipage}{\textwidth} 
\begin{tcolorbox}[colback = red!20!white,title=Versión Ecuación Matricial]
Resuelve la ecuación matricial siguiente despejando el valor de la matriz $M$
\[M \left[\begin{array}{rrr}
-1 & 0 & -1 \\
0 & 1 & -3 \\
-2 & 2 & -9
\end{array}\right] = \left[\begin{array}{rrr}
0 & -\frac{1}{2} & -1 \\
2 & 2 & -1 \\
1 & -1 & \frac{1}{2} \\
-2 & 0 & 1
\end{array}\right] \quad 
\]
\end{tcolorbox}
\end{minipage}%
\end{ejer}


{\it Soluci\'on:}
% Escribe tu soluci\'on para el Ejercicio 20

% Fin del Ejercicio 20


\begin{ejer} Sea $K$ el cuerpo de 19 elementos.
\newline
\noindent\begin{minipage}{\textwidth}
\begin{tcolorbox}[colback = green!20!white,title=Versión Aplicaciones]
Determina si la aplicaci\'on lineal asociada a la matriz $A$ es inyectiva, sobreyectiva, biyectiva o ninguna de esas cosas, siendo $A$ la matriz \end{tcolorbox}
\end{minipage} \newline
\noindent\begin{minipage}{\textwidth}
\begin{tcolorbox}[colback = blue!20!white,title=Versión Vectores]
Determina si las columnas de la matriz $A$ son linealmente idependientes, generadores o base de $K^{3}$, siendo $A$ la matriz \end{tcolorbox}
\end{minipage} \newline
\noindent\begin{minipage}{\textwidth} 
\begin{tcolorbox}[colback = red!20!white,title=Versión Inversas]
Determina si la matriz $A$ tiene inversa por la izquierda, por la derecha, por los dos lados o por ninguno, siendo $A$ la matriz 
\end{tcolorbox}
\end{minipage}
\[ A = \left[\begin{array}{rrrr}
1 & 9 & 8 & 9 \\
0 & 1 & 6 & 5 \\
16 & 0 & 6 & 6
\end{array}\right] \in {\bf M}_{3\times 4}({\mathbb Z}_{19})\]
\end{ejer}

{\it Soluci\'on:}
% Escribe tu soluci\'on para el Ejercicio 21

\begin{sageblock}
matrix(Zmod(19),[[1,9,8,9],
[0,1,6,5],
[16,0,6,6]])
\end{sageblock}

% Fin del Ejercicio 21


\begin{ejer} Sea $K$ el cuerpo de 43 elementos.
\newline
\noindent\begin{minipage}{\textwidth}
\begin{tcolorbox}[colback = green!20!white,title=Versión Aplicaciones]
Determina si la aplicaci\'on lineal asociada a la matriz $A$ es inyectiva, sobreyectiva, biyectiva o ninguna de esas cosas, siendo $A$ la matriz \end{tcolorbox}
\end{minipage} \newline
\noindent\begin{minipage}{\textwidth}
\begin{tcolorbox}[colback = blue!20!white,title=Versión Vectores]
Determina si las columnas de la matriz $A$ son linealmente idependientes, generadores o base de $K^{5}$, siendo $A$ la matriz \end{tcolorbox}
\end{minipage} \newline
\noindent\begin{minipage}{\textwidth} 
\begin{tcolorbox}[colback = red!20!white,title=Versión Inversas]
Determina si la matriz $A$ tiene inversa por la izquierda, por la derecha, por los dos lados o por ninguno, siendo $A$ la matriz 
\end{tcolorbox}
\end{minipage}
\[ A = \left[\begin{array}{rrrrr}
33 & 4 & 7 & 36 & 32 \\
36 & 7 & 34 & 14 & 20 \\
13 & 37 & 14 & 39 & 42 \\
32 & 38 & 26 & 0 & 3 \\
26 & 36 & 37 & 42 & 39
\end{array}\right] \in {\bf M}_{5\times 5}({\mathbb Z}_{43})\]
\end{ejer}

{\it Soluci\'on:}
% Escribe tu soluci\'on para el Ejercicio 22

\begin{sageblock}
matrix(Zmod(43),[[33,4,7,36,32],
[36,7,34,14,20],
[13,37,14,39,42],
[32,38,26,0,3],
[26,36,37,42,39]])
\end{sageblock}

% Fin del Ejercicio 22


\begin{ejer} Sea $K$ el cuerpo de 31 elementos.
\newline
\noindent\begin{minipage}{\textwidth}
\begin{tcolorbox}[colback = green!20!white,title=Versión Aplicaciones]
Determina si la aplicaci\'on lineal asociada a la matriz $A$ es inyectiva, sobreyectiva, biyectiva o ninguna de esas cosas, siendo $A$ la matriz \end{tcolorbox}
\end{minipage} \newline
\noindent\begin{minipage}{\textwidth}
\begin{tcolorbox}[colback = blue!20!white,title=Versión Vectores]
Determina si las columnas de la matriz $A$ son linealmente idependientes, generadores o base de $K^{4}$, siendo $A$ la matriz \end{tcolorbox}
\end{minipage} \newline
\noindent\begin{minipage}{\textwidth} 
\begin{tcolorbox}[colback = red!20!white,title=Versión Inversas]
Determina si la matriz $A$ tiene inversa por la izquierda, por la derecha, por los dos lados o por ninguno, siendo $A$ la matriz 
\end{tcolorbox}
\end{minipage}
\[ A = \left[\begin{array}{rrrr}
13 & 7 & 6 & 0 \\
18 & 2 & 15 & 2 \\
20 & 30 & 8 & 19 \\
13 & 18 & 23 & 9
\end{array}\right] \in {\bf M}_{4\times 4}({\mathbb Z}_{31})\]
\end{ejer}

{\it Soluci\'on:}
% Escribe tu soluci\'on para el Ejercicio 23

\begin{sageblock}
matrix(Zmod(31),[[13,7,6,0],
[18,2,15,2],
[20,30,8,19],
[13,18,23,9]])
\end{sageblock}

% Fin del Ejercicio 23


\begin{ejer} Sea $K$ el cuerpo de los n\'umeros reales.
\newline
\noindent\begin{minipage}{\textwidth}
\begin{tcolorbox}[colback = green!20!white,title=Versión Aplicaciones]
Determina si la aplicaci\'on lineal asociada a la matriz $A$ es inyectiva, sobreyectiva, biyectiva o ninguna de esas cosas, siendo $A$ la matriz \end{tcolorbox}
\end{minipage} \newline
\noindent\begin{minipage}{\textwidth}
\begin{tcolorbox}[colback = blue!20!white,title=Versión Vectores]
Determina si las columnas de la matriz $A$ son linealmente idependientes, generadores o base de $K^{4}$, siendo $A$ la matriz \end{tcolorbox}
\end{minipage} \newline
\noindent\begin{minipage}{\textwidth} 
\begin{tcolorbox}[colback = red!20!white,title=Versión Inversas]
Determina si la matriz $A$ tiene inversa por la izquierda, por la derecha, por los dos lados o por ninguno, siendo $A$ la matriz 
\end{tcolorbox}
\end{minipage}
\[ A = \left[\begin{array}{rrrrr}
-1 & 3 & -2 & 4 & -2 \\
-2 & 5 & -3 & 5 & -5 \\
-2 & 2 & 1 & -7 & -9 \\
1 & -4 & 1 & 0 & 4
\end{array}\right] \in {\bf M}_{4\times 5}({\mathbb R})\]
\end{ejer}

{\it Soluci\'on:}
% Escribe tu soluci\'on para el Ejercicio 24

\begin{sageblock}
matrix(QQ,[[-1,3,-2,4,-2],
[-2,5,-3,5,-5],
[-2,2,1,-7,-9],
[1,-4,1,0,4]])
\end{sageblock}

% Fin del Ejercicio 24


\begin{ejer} Sea $K$ el cuerpo de 5 elementos.
\newline
\noindent\begin{minipage}{\textwidth}
\begin{tcolorbox}[colback = green!20!white,title=Versión Aplicaciones]
Determina si la aplicaci\'on lineal asociada a la matriz $A$ es inyectiva, sobreyectiva, biyectiva o ninguna de esas cosas, siendo $A$ la matriz \end{tcolorbox}
\end{minipage} \newline
\noindent\begin{minipage}{\textwidth}
\begin{tcolorbox}[colback = blue!20!white,title=Versión Vectores]
Determina si las columnas de la matriz $A$ son linealmente idependientes, generadores o base de $K^{3}$, siendo $A$ la matriz \end{tcolorbox}
\end{minipage} \newline
\noindent\begin{minipage}{\textwidth} 
\begin{tcolorbox}[colback = red!20!white,title=Versión Inversas]
Determina si la matriz $A$ tiene inversa por la izquierda, por la derecha, por los dos lados o por ninguno, siendo $A$ la matriz 
\end{tcolorbox}
\end{minipage}
\[ A = \left[\begin{array}{rrrr}
2 & 2 & 4 & 1 \\
3 & 1 & 3 & 2 \\
4 & 3 & 4 & 1
\end{array}\right] \in {\bf M}_{3\times 4}({\mathbb Z}_{5})\]
\end{ejer}

{\it Soluci\'on:}
% Escribe tu soluci\'on para el Ejercicio 25

\begin{sageblock}
matrix(Zmod(5),[[2,2,4,1],
[3,1,3,2],
[4,3,4,1]])
\end{sageblock}

% Fin del Ejercicio 25


\begin{ejer} Sea $K$ el cuerpo de los n\'umeros reales.
\newline
\noindent\begin{minipage}{\textwidth}
\begin{tcolorbox}[colback = green!20!white,title=Versión Aplicaciones]
Determina si la aplicaci\'on lineal asociada a la matriz $A$ es inyectiva, sobreyectiva, biyectiva o ninguna de esas cosas, siendo $A$ la matriz \end{tcolorbox}
\end{minipage} \newline
\noindent\begin{minipage}{\textwidth}
\begin{tcolorbox}[colback = blue!20!white,title=Versión Vectores]
Determina si las columnas de la matriz $A$ son linealmente idependientes, generadores o base de $K^{4}$, siendo $A$ la matriz \end{tcolorbox}
\end{minipage} \newline
\noindent\begin{minipage}{\textwidth} 
\begin{tcolorbox}[colback = red!20!white,title=Versión Inversas]
Determina si la matriz $A$ tiene inversa por la izquierda, por la derecha, por los dos lados o por ninguno, siendo $A$ la matriz 
\end{tcolorbox}
\end{minipage}
\[ A = \left[\begin{array}{rrrr}
1 & 3 & -1 & 4 \\
5 & 6 & -3 & 0 \\
-1 & -3 & -2 & -8 \\
0 & -2 & 0 & -5
\end{array}\right] \in {\bf M}_{4\times 4}({\mathbb R})\]
\end{ejer}

{\it Soluci\'on:}
% Escribe tu soluci\'on para el Ejercicio 26

\begin{sageblock}
matrix(QQ,[[1,3,-1,4],
[5,6,-3,0],
[-1,-3,-2,-8],
[0,-2,0,-5]])
\end{sageblock}

% Fin del Ejercicio 26


\begin{ejer} Sea $K$ el cuerpo de los n\'umeros reales.
\newline
\noindent\begin{minipage}{\textwidth}
\begin{tcolorbox}[colback = green!20!white,title=Versión Aplicaciones]
Determina si la aplicaci\'on lineal asociada a la matriz $A$ es inyectiva, sobreyectiva, biyectiva o ninguna de esas cosas, siendo $A$ la matriz \end{tcolorbox}
\end{minipage} \newline
\noindent\begin{minipage}{\textwidth}
\begin{tcolorbox}[colback = blue!20!white,title=Versión Vectores]
Determina si las columnas de la matriz $A$ son linealmente idependientes, generadores o base de $K^{4}$, siendo $A$ la matriz \end{tcolorbox}
\end{minipage} \newline
\noindent\begin{minipage}{\textwidth} 
\begin{tcolorbox}[colback = red!20!white,title=Versión Inversas]
Determina si la matriz $A$ tiene inversa por la izquierda, por la derecha, por los dos lados o por ninguno, siendo $A$ la matriz 
\end{tcolorbox}
\end{minipage}
\[ A = \left[\begin{array}{rrr}
2 & -2 & 8 \\
-1 & 3 & -5 \\
-1 & -1 & -4 \\
1 & 0 & 3
\end{array}\right] \in {\bf M}_{4\times 3}({\mathbb R})\]
\end{ejer}

{\it Soluci\'on:}
% Escribe tu soluci\'on para el Ejercicio 27

\begin{sageblock}
matrix(QQ,[[2,-2,8],
[-1,3,-5],
[-1,-1,-4],
[1,0,3]])
\end{sageblock}

% Fin del Ejercicio 27


\begin{ejer} Sea $K$ el cuerpo de 19 elementos.
\newline
\noindent\begin{minipage}{\textwidth}
\begin{tcolorbox}[colback = green!20!white,title=Versión Aplicaciones]
Determina si la aplicaci\'on lineal asociada a la matriz $A$ es inyectiva, sobreyectiva, biyectiva o ninguna de esas cosas, siendo $A$ la matriz \end{tcolorbox}
\end{minipage} \newline
\noindent\begin{minipage}{\textwidth}
\begin{tcolorbox}[colback = blue!20!white,title=Versión Vectores]
Determina si las columnas de la matriz $A$ son linealmente idependientes, generadores o base de $K^{3}$, siendo $A$ la matriz \end{tcolorbox}
\end{minipage} \newline
\noindent\begin{minipage}{\textwidth} 
\begin{tcolorbox}[colback = red!20!white,title=Versión Inversas]
Determina si la matriz $A$ tiene inversa por la izquierda, por la derecha, por los dos lados o por ninguno, siendo $A$ la matriz 
\end{tcolorbox}
\end{minipage}
\[ A = \left[\begin{array}{rrrrr}
16 & 4 & 18 & 16 & 9 \\
12 & 0 & 10 & 5 & 16 \\
5 & 4 & 10 & 18 & 0
\end{array}\right] \in {\bf M}_{3\times 5}({\mathbb Z}_{19})\]
\end{ejer}

{\it Soluci\'on:}
% Escribe tu soluci\'on para el Ejercicio 28

\begin{sageblock}
matrix(Zmod(19),[[16,4,18,16,9],
[12,0,10,5,16],
[5,4,10,18,0]])
\end{sageblock}

% Fin del Ejercicio 28


\begin{ejer} Sea $K$ el cuerpo de 37 elementos.
\newline
\noindent\begin{minipage}{\textwidth}
\begin{tcolorbox}[colback = green!20!white,title=Versión Aplicaciones]
Determina si la aplicaci\'on lineal asociada a la matriz $A$ es inyectiva, sobreyectiva, biyectiva o ninguna de esas cosas, siendo $A$ la matriz \end{tcolorbox}
\end{minipage} \newline
\noindent\begin{minipage}{\textwidth}
\begin{tcolorbox}[colback = blue!20!white,title=Versión Vectores]
Determina si las columnas de la matriz $A$ son linealmente idependientes, generadores o base de $K^{3}$, siendo $A$ la matriz \end{tcolorbox}
\end{minipage} \newline
\noindent\begin{minipage}{\textwidth} 
\begin{tcolorbox}[colback = red!20!white,title=Versión Inversas]
Determina si la matriz $A$ tiene inversa por la izquierda, por la derecha, por los dos lados o por ninguno, siendo $A$ la matriz 
\end{tcolorbox}
\end{minipage}
\[ A = \left[\begin{array}{rrr}
0 & 14 & 14 \\
15 & 2 & 11 \\
7 & 11 & 27
\end{array}\right] \in {\bf M}_{3\times 3}({\mathbb Z}_{37})\]
\end{ejer}

{\it Soluci\'on:}
% Escribe tu soluci\'on para el Ejercicio 29

\begin{sageblock}
matrix(Zmod(37),[[0,14,14],
[15,2,11],
[7,11,27]])
\end{sageblock}

% Fin del Ejercicio 29


\begin{ejer} Sea $K$ el cuerpo de 7 elementos.
\newline
\noindent\begin{minipage}{\textwidth}
\begin{tcolorbox}[colback = green!20!white,title=Versión Aplicaciones]
Determina si la aplicaci\'on lineal asociada a la matriz $A$ es inyectiva, sobreyectiva, biyectiva o ninguna de esas cosas, siendo $A$ la matriz \end{tcolorbox}
\end{minipage} \newline
\noindent\begin{minipage}{\textwidth}
\begin{tcolorbox}[colback = blue!20!white,title=Versión Vectores]
Determina si las columnas de la matriz $A$ son linealmente idependientes, generadores o base de $K^{5}$, siendo $A$ la matriz \end{tcolorbox}
\end{minipage} \newline
\noindent\begin{minipage}{\textwidth} 
\begin{tcolorbox}[colback = red!20!white,title=Versión Inversas]
Determina si la matriz $A$ tiene inversa por la izquierda, por la derecha, por los dos lados o por ninguno, siendo $A$ la matriz 
\end{tcolorbox}
\end{minipage}
\[ A = \left[\begin{array}{rrr}
1 & 5 & 0 \\
6 & 2 & 1 \\
6 & 2 & 4 \\
4 & 6 & 6 \\
5 & 4 & 5
\end{array}\right] \in {\bf M}_{5\times 3}({\mathbb Z}_{7})\]
\end{ejer}

{\it Soluci\'on:}
% Escribe tu soluci\'on para el Ejercicio 30

\begin{sageblock}
matrix(Zmod(7),[[1,5,0],
[6,2,1],
[6,2,4],
[4,6,6],
[5,4,5]])
\end{sageblock}

% Fin del Ejercicio 30


\begin{ejer} Sea $K$ el cuerpo de 11 elementos.
\newline
\noindent\begin{minipage}{\textwidth}
\begin{tcolorbox}[colback = green!20!white,title=Versión Aplicaciones]
Determina si la aplicaci\'on lineal asociada a la matriz $A$ es inyectiva, sobreyectiva, biyectiva o ninguna de esas cosas, siendo $A$ la matriz \end{tcolorbox}
\end{minipage} \newline
\noindent\begin{minipage}{\textwidth}
\begin{tcolorbox}[colback = blue!20!white,title=Versión Vectores]
Determina si las columnas de la matriz $A$ son linealmente idependientes, generadores o base de $K^{3}$, siendo $A$ la matriz \end{tcolorbox}
\end{minipage} \newline
\noindent\begin{minipage}{\textwidth} 
\begin{tcolorbox}[colback = red!20!white,title=Versión Inversas]
Determina si la matriz $A$ tiene inversa por la izquierda, por la derecha, por los dos lados o por ninguno, siendo $A$ la matriz 
\end{tcolorbox}
\end{minipage}
\[ A = \left[\begin{array}{rrr}
6 & 6 & 2 \\
7 & 9 & 6 \\
1 & 0 & 5
\end{array}\right] \in {\bf M}_{3\times 3}({\mathbb Z}_{11})\]
\end{ejer}

{\it Soluci\'on:}
% Escribe tu soluci\'on para el Ejercicio 31

\begin{sageblock}
matrix(Zmod(11),[[6,6,2],
[7,9,6],
[1,0,5]])
\end{sageblock}

% Fin del Ejercicio 31


\begin{ejer} Sea $K$ el cuerpo de 31 elementos.
\newline
\noindent\begin{minipage}{\textwidth}
\begin{tcolorbox}[colback = green!20!white,title=Versión Aplicaciones]
Determina si la aplicaci\'on lineal asociada a la matriz $A$ es inyectiva, sobreyectiva, biyectiva o ninguna de esas cosas, siendo $A$ la matriz \end{tcolorbox}
\end{minipage} \newline
\noindent\begin{minipage}{\textwidth}
\begin{tcolorbox}[colback = blue!20!white,title=Versión Vectores]
Determina si las columnas de la matriz $A$ son linealmente idependientes, generadores o base de $K^{5}$, siendo $A$ la matriz \end{tcolorbox}
\end{minipage} \newline
\noindent\begin{minipage}{\textwidth} 
\begin{tcolorbox}[colback = red!20!white,title=Versión Inversas]
Determina si la matriz $A$ tiene inversa por la izquierda, por la derecha, por los dos lados o por ninguno, siendo $A$ la matriz 
\end{tcolorbox}
\end{minipage}
\[ A = \left[\begin{array}{rrrr}
27 & 2 & 22 & 11 \\
14 & 16 & 19 & 26 \\
21 & 24 & 14 & 6 \\
21 & 24 & 1 & 2 \\
18 & 28 & 8 & 28
\end{array}\right] \in {\bf M}_{5\times 4}({\mathbb Z}_{31})\]
\end{ejer}

{\it Soluci\'on:}
% Escribe tu soluci\'on para el Ejercicio 32

\begin{sageblock}
matrix(Zmod(31),[[27,2,22,11],
[14,16,19,26],
[21,24,14,6],
[21,24,1,2],
[18,28,8,28]])
\end{sageblock}

% Fin del Ejercicio 32


\begin{ejer} Sea $K$ el cuerpo de los n\'umeros reales.
\newline
\noindent\begin{minipage}{\textwidth}
\begin{tcolorbox}[colback = green!20!white,title=Versión Aplicaciones]
Determina si la aplicaci\'on lineal asociada a la matriz $A$ es inyectiva, sobreyectiva, biyectiva o ninguna de esas cosas, siendo $A$ la matriz \end{tcolorbox}
\end{minipage} \newline
\noindent\begin{minipage}{\textwidth}
\begin{tcolorbox}[colback = blue!20!white,title=Versión Vectores]
Determina si las columnas de la matriz $A$ son linealmente idependientes, generadores o base de $K^{4}$, siendo $A$ la matriz \end{tcolorbox}
\end{minipage} \newline
\noindent\begin{minipage}{\textwidth} 
\begin{tcolorbox}[colback = red!20!white,title=Versión Inversas]
Determina si la matriz $A$ tiene inversa por la izquierda, por la derecha, por los dos lados o por ninguno, siendo $A$ la matriz 
\end{tcolorbox}
\end{minipage}
\[ A = \left[\begin{array}{rrrr}
1 & -3 & -8 & -2 \\
-2 & -1 & -8 & 7 \\
0 & -2 & -7 & 1 \\
-2 & -1 & -4 & 3
\end{array}\right] \in {\bf M}_{4\times 4}({\mathbb R})\]
\end{ejer}

{\it Soluci\'on:}
% Escribe tu soluci\'on para el Ejercicio 33

\begin{sageblock}
matrix(QQ,[[1,-3,-8,-2],
[-2,-1,-8,7],
[0,-2,-7,1],
[-2,-1,-4,3]])
\end{sageblock}

% Fin del Ejercicio 33


\begin{ejer} Sea $K$ el cuerpo de 47 elementos.
\newline
\noindent\begin{minipage}{\textwidth}
\begin{tcolorbox}[colback = green!20!white,title=Versión Aplicaciones]
Determina si la aplicaci\'on lineal asociada a la matriz $A$ es inyectiva, sobreyectiva, biyectiva o ninguna de esas cosas, siendo $A$ la matriz \end{tcolorbox}
\end{minipage} \newline
\noindent\begin{minipage}{\textwidth}
\begin{tcolorbox}[colback = blue!20!white,title=Versión Vectores]
Determina si las columnas de la matriz $A$ son linealmente idependientes, generadores o base de $K^{3}$, siendo $A$ la matriz \end{tcolorbox}
\end{minipage} \newline
\noindent\begin{minipage}{\textwidth} 
\begin{tcolorbox}[colback = red!20!white,title=Versión Inversas]
Determina si la matriz $A$ tiene inversa por la izquierda, por la derecha, por los dos lados o por ninguno, siendo $A$ la matriz 
\end{tcolorbox}
\end{minipage}
\[ A = \left[\begin{array}{rrr}
18 & 27 & 46 \\
31 & 10 & 26 \\
25 & 14 & 44
\end{array}\right] \in {\bf M}_{3\times 3}({\mathbb Z}_{47})\]
\end{ejer}

{\it Soluci\'on:}
% Escribe tu soluci\'on para el Ejercicio 34

\begin{sageblock}
matrix(Zmod(47),[[18,27,46],
[31,10,26],
[25,14,44]])
\end{sageblock}

% Fin del Ejercicio 34


\begin{ejer} Sea $K$ el cuerpo de los n\'umeros reales.
\newline
\noindent\begin{minipage}{\textwidth}
\begin{tcolorbox}[colback = green!20!white,title=Versión Aplicaciones]
Determina si la aplicaci\'on lineal asociada a la matriz $A$ es inyectiva, sobreyectiva, biyectiva o ninguna de esas cosas, siendo $A$ la matriz \end{tcolorbox}
\end{minipage} \newline
\noindent\begin{minipage}{\textwidth}
\begin{tcolorbox}[colback = blue!20!white,title=Versión Vectores]
Determina si las columnas de la matriz $A$ son linealmente idependientes, generadores o base de $K^{3}$, siendo $A$ la matriz \end{tcolorbox}
\end{minipage} \newline
\noindent\begin{minipage}{\textwidth} 
\begin{tcolorbox}[colback = red!20!white,title=Versión Inversas]
Determina si la matriz $A$ tiene inversa por la izquierda, por la derecha, por los dos lados o por ninguno, siendo $A$ la matriz 
\end{tcolorbox}
\end{minipage}
\[ A = \left[\begin{array}{rrrr}
-5 & -1 & -2 & -2 \\
-1 & 2 & 9 & 4 \\
2 & 0 & -1 & 0
\end{array}\right] \in {\bf M}_{3\times 4}({\mathbb R})\]
\end{ejer}

{\it Soluci\'on:}
% Escribe tu soluci\'on para el Ejercicio 35

\begin{sageblock}
matrix(QQ,[[-5,-1,-2,-2],
[-1,2,9,4],
[2,0,-1,0]])
\end{sageblock}

% Fin del Ejercicio 35


\begin{ejer} Sea $K$ el cuerpo de 17 elementos.
\newline
\noindent\begin{minipage}{\textwidth}
\begin{tcolorbox}[colback = green!20!white,title=Versión Aplicaciones]
Determina si la aplicaci\'on lineal asociada a la matriz $A$ es inyectiva, sobreyectiva, biyectiva o ninguna de esas cosas, siendo $A$ la matriz \end{tcolorbox}
\end{minipage} \newline
\noindent\begin{minipage}{\textwidth}
\begin{tcolorbox}[colback = blue!20!white,title=Versión Vectores]
Determina si las columnas de la matriz $A$ son linealmente idependientes, generadores o base de $K^{5}$, siendo $A$ la matriz \end{tcolorbox}
\end{minipage} \newline
\noindent\begin{minipage}{\textwidth} 
\begin{tcolorbox}[colback = red!20!white,title=Versión Inversas]
Determina si la matriz $A$ tiene inversa por la izquierda, por la derecha, por los dos lados o por ninguno, siendo $A$ la matriz 
\end{tcolorbox}
\end{minipage}
\[ A = \left[\begin{array}{rrr}
5 & 5 & 3 \\
16 & 10 & 10 \\
5 & 2 & 3 \\
10 & 5 & 1 \\
12 & 8 & 2
\end{array}\right] \in {\bf M}_{5\times 3}({\mathbb Z}_{17})\]
\end{ejer}

{\it Soluci\'on:}
% Escribe tu soluci\'on para el Ejercicio 36

\begin{sageblock}
matrix(Zmod(17),[[5,5,3],
[16,10,10],
[5,2,3],
[10,5,1],
[12,8,2]])
\end{sageblock}

% Fin del Ejercicio 36


\begin{ejer} Sea $K$ el cuerpo de 5 elementos.
\newline
\noindent\begin{minipage}{\textwidth}
\begin{tcolorbox}[colback = green!20!white,title=Versión Aplicaciones]
Determina si la aplicaci\'on lineal asociada a la matriz $A$ es inyectiva, sobreyectiva, biyectiva o ninguna de esas cosas, siendo $A$ la matriz \end{tcolorbox}
\end{minipage} \newline
\noindent\begin{minipage}{\textwidth}
\begin{tcolorbox}[colback = blue!20!white,title=Versión Vectores]
Determina si las columnas de la matriz $A$ son linealmente idependientes, generadores o base de $K^{4}$, siendo $A$ la matriz \end{tcolorbox}
\end{minipage} \newline
\noindent\begin{minipage}{\textwidth} 
\begin{tcolorbox}[colback = red!20!white,title=Versión Inversas]
Determina si la matriz $A$ tiene inversa por la izquierda, por la derecha, por los dos lados o por ninguno, siendo $A$ la matriz 
\end{tcolorbox}
\end{minipage}
\[ A = \left[\begin{array}{rrrr}
1 & 4 & 2 & 2 \\
0 & 1 & 2 & 0 \\
3 & 1 & 0 & 2 \\
1 & 1 & 3 & 0
\end{array}\right] \in {\bf M}_{4\times 4}({\mathbb Z}_{5})\]
\end{ejer}

{\it Soluci\'on:}
% Escribe tu soluci\'on para el Ejercicio 37

\begin{sageblock}
matrix(Zmod(5),[[1,4,2,2],
[0,1,2,0],
[3,1,0,2],
[1,1,3,0]])
\end{sageblock}

% Fin del Ejercicio 37


\begin{ejer} Sea $K$ el cuerpo de 7 elementos.
\newline
\noindent\begin{minipage}{\textwidth}
\begin{tcolorbox}[colback = green!20!white,title=Versión Aplicaciones]
Determina si la aplicaci\'on lineal asociada a la matriz $A$ es inyectiva, sobreyectiva, biyectiva o ninguna de esas cosas, siendo $A$ la matriz \end{tcolorbox}
\end{minipage} \newline
\noindent\begin{minipage}{\textwidth}
\begin{tcolorbox}[colback = blue!20!white,title=Versión Vectores]
Determina si las columnas de la matriz $A$ son linealmente idependientes, generadores o base de $K^{4}$, siendo $A$ la matriz \end{tcolorbox}
\end{minipage} \newline
\noindent\begin{minipage}{\textwidth} 
\begin{tcolorbox}[colback = red!20!white,title=Versión Inversas]
Determina si la matriz $A$ tiene inversa por la izquierda, por la derecha, por los dos lados o por ninguno, siendo $A$ la matriz 
\end{tcolorbox}
\end{minipage}
\[ A = \left[\begin{array}{rrrr}
6 & 1 & 0 & 2 \\
3 & 0 & 1 & 0 \\
5 & 6 & 1 & 4 \\
4 & 3 & 4 & 5
\end{array}\right] \in {\bf M}_{4\times 4}({\mathbb Z}_{7})\]
\end{ejer}

{\it Soluci\'on:}
% Escribe tu soluci\'on para el Ejercicio 38

\begin{sageblock}
A = matrix(Zmod(7),[[6,1,0,2],
[3,0,1,0],
[5,6,1,4],
[4,3,4,5]])
Ar = A.echelon_form()
\end{sageblock}

$$
	R = \sage{Ar}
$$

La aplicación lineal biyectiva, sus columnas son base y su matriz tiene inversa por los dos lados

% Fin del Ejercicio 38


\begin{ejer} Sea $K$ el cuerpo de los n\'umeros reales.
\newline
\noindent\begin{minipage}{\textwidth}
\begin{tcolorbox}[colback = green!20!white,title=Versión Aplicaciones]
Determina si la aplicaci\'on lineal asociada a la matriz $A$ es inyectiva, sobreyectiva, biyectiva o ninguna de esas cosas, siendo $A$ la matriz \end{tcolorbox}
\end{minipage} \newline
\noindent\begin{minipage}{\textwidth}
\begin{tcolorbox}[colback = blue!20!white,title=Versión Vectores]
Determina si las columnas de la matriz $A$ son linealmente idependientes, generadores o base de $K^{3}$, siendo $A$ la matriz \end{tcolorbox}
\end{minipage} \newline
\noindent\begin{minipage}{\textwidth} 
\begin{tcolorbox}[colback = red!20!white,title=Versión Inversas]
Determina si la matriz $A$ tiene inversa por la izquierda, por la derecha, por los dos lados o por ninguno, siendo $A$ la matriz 
\end{tcolorbox}
\end{minipage}
\[ A = \left[\begin{array}{rrr}
1 & 0 & 3 \\
2 & 1 & 7 \\
0 & 0 & 1
\end{array}\right] \in {\bf M}_{3\times 3}({\mathbb R})\]
\end{ejer}

{\it Soluci\'on:}
% Escribe tu soluci\'on para el Ejercicio 39

\begin{sageblock}
matrix(QQ,[[1,0,3],
[2,1,7],
[0,0,1]])
\end{sageblock}

% Fin del Ejercicio 39


\begin{ejer} Sea $K$ el cuerpo de los n\'umeros reales.
\newline
\noindent\begin{minipage}{\textwidth}
\begin{tcolorbox}[colback = green!20!white,title=Versión Aplicaciones]
Determina si la aplicaci\'on lineal asociada a la matriz $A$ es inyectiva, sobreyectiva, biyectiva o ninguna de esas cosas, siendo $A$ la matriz \end{tcolorbox}
\end{minipage} \newline
\noindent\begin{minipage}{\textwidth}
\begin{tcolorbox}[colback = blue!20!white,title=Versión Vectores]
Determina si las columnas de la matriz $A$ son linealmente idependientes, generadores o base de $K^{4}$, siendo $A$ la matriz \end{tcolorbox}
\end{minipage} \newline
\noindent\begin{minipage}{\textwidth} 
\begin{tcolorbox}[colback = red!20!white,title=Versión Inversas]
Determina si la matriz $A$ tiene inversa por la izquierda, por la derecha, por los dos lados o por ninguno, siendo $A$ la matriz 
\end{tcolorbox}
\end{minipage}
\[ A = \left[\begin{array}{rrr}
3 & 1 & -6 \\
2 & -1 & -9 \\
1 & -1 & -5 \\
1 & 1 & 0
\end{array}\right] \in {\bf M}_{4\times 3}({\mathbb R})\]
\end{ejer}

{\it Soluci\'on:}
% Escribe tu soluci\'on para el Ejercicio 40

\begin{sageblock}
matrix(QQ,[[3,1,-6],
[2,-1,-9],
[1,-1,-5],
[1,1,0]])
\end{sageblock}

% Fin del Ejercicio 40


\begin{ejer} Sea $K$ el cuerpo de 11 elementos.
\newline
\noindent\begin{minipage}{\textwidth}
\begin{tcolorbox}[colback = green!20!white,title=Versión Núcleo]
Determina de entre los vectores columna de la matriz $A$, cuales de ellos están en el núcleo de la aplicación lineal $f:K^{2} \to K^{3}$ siendo  $$ M(f) = \left[\begin{array}{rr}
7 & 9 \\
6 & 3 \\
10 & 5
\end{array}\right] $$ y $A$ la matriz que se da a continuación:\end{tcolorbox}
\end{minipage} \newline
\noindent\begin{minipage}{\textwidth}
\begin{tcolorbox}[colback = blue!20!white,title=Versión Anulador]
Determina de entre los vectores columna de la matriz $A$, cuales de ellos están en el anulador por la derecha de la matriz $$ \left[\begin{array}{rr}
7 & 9 \\
6 & 3 \\
10 & 5
\end{array}\right] $$ siendo $A$ la matriz que se da a continuación:\end{tcolorbox}
\end{minipage} \newline
\noindent\begin{minipage}{\textwidth} 
\begin{tcolorbox}[colback = red!20!white,title=Versión Ecuaciones Implícitas]
Determina de entre los vectores columna de la matriz $A$, cuales de ellos están en espacio vectorial definido en forma implícita por las siguientes ecuaciones:
\[ 7 x_{0} + 9 x_{1} = 0 \]
\[ 6 x_{0} + 3 x_{1} = 0 \]
\[ -x_{0} + 5 x_{1} = 0 \]
Siendo $A$ la matriz que se da a continuación:
\end{tcolorbox}
\end{minipage}
\[ A = \left[\begin{array}{rrrrrrrrr}
10 & 6 & 1 & 3 & 5 & 6 & 4 & 9 & 0 \\
0 & 0 & 9 & 5 & 4 & 10 & 3 & 3 & 0
\end{array}\right] \in {\bf M}_{2\times 9}({\mathbb Z}_{11})\]
\end{ejer}

{\it Soluci\'on:}
% Escribe tu soluci\'on para el Ejercicio 41

\begin{sageblock}
matrix(Zmod(11),[[7,9],
[6,3],
[10,5]])
matrix(Zmod(11),[[10,6,1,3,5,6,4,9,0],
[0,0,9,5,4,10,3,3,0]])
\end{sageblock}

% Fin del Ejercicio 41


\begin{ejer} Sea $K$ el cuerpo de los n\'umeros reales.
\newline
\noindent\begin{minipage}{\textwidth}
\begin{tcolorbox}[colback = green!20!white,title=Versión Núcleo]
Determina de entre los vectores columna de la matriz $A$, cuales de ellos están en el núcleo de la aplicación lineal $f:K^{3} \to K^{5}$ siendo  $$ M(f) = \left[\begin{array}{rrr}
-2 & 8 & 4 \\
2 & -8 & -9 \\
1 & -4 & -3 \\
-1 & 4 & 0 \\
-1 & 4 & 8
\end{array}\right] $$ y $A$ la matriz que se da a continuación:\end{tcolorbox}
\end{minipage} \newline
\noindent\begin{minipage}{\textwidth}
\begin{tcolorbox}[colback = blue!20!white,title=Versión Anulador]
Determina de entre los vectores columna de la matriz $A$, cuales de ellos están en el anulador por la derecha de la matriz $$ \left[\begin{array}{rrr}
-2 & 8 & 4 \\
2 & -8 & -9 \\
1 & -4 & -3 \\
-1 & 4 & 0 \\
-1 & 4 & 8
\end{array}\right] $$ siendo $A$ la matriz que se da a continuación:\end{tcolorbox}
\end{minipage} \newline
\noindent\begin{minipage}{\textwidth} 
\begin{tcolorbox}[colback = red!20!white,title=Versión Ecuaciones Implícitas]
Determina de entre los vectores columna de la matriz $A$, cuales de ellos están en espacio vectorial definido en forma implícita por las siguientes ecuaciones:
\[ -2 x_{0} + 8 x_{1} + 4 x_{2} = 0 \]
\[ 2 x_{0} - 8 x_{1} - 9 x_{2} = 0 \]
\[ x_{0} - 4 x_{1} - 3 x_{2} = 0 \]
\[ -x_{0} + 4 x_{1} = 0 \]
\[ -x_{0} + 4 x_{1} + 8 x_{2} = 0 \]
Siendo $A$ la matriz que se da a continuación:
\end{tcolorbox}
\end{minipage}
\[ A = \left[\begin{array}{rrrrrrrrrr}
-4 & \frac{1}{8} & 2 & 2 & -3 & -1 & 1 & -\frac{1}{2} & \frac{1}{4} & 1 \\
1 & \frac{1}{32} & 2 & \frac{1}{2} & -\frac{1}{29} & -\frac{1}{4} & \frac{1}{4} & -\frac{1}{8} & 0 & \frac{1}{3} \\
-2 & 0 & \frac{2}{5} & 0 & \frac{1}{2} & 0 & 0 & 0 & -\frac{1}{3} & -\frac{1}{2}
\end{array}\right] \in {\bf M}_{3\times 10}({\mathbb R})\]
\end{ejer}

{\it Soluci\'on:}
% Escribe tu soluci\'on para el Ejercicio 42

\begin{sageblock}
M = matrix(QQ,[[-2,8,4],
[2,-8,-9],
[1,-4,-3],
[-1,4,0],
[-1,4,8]])
A = matrix(QQ,[[-4,1/8,2,2,-3,-1,1,-1/2,1/4,1],
[1,1/32,2,1/2,-1/29,-1/4,1/4,-1/8,0,1/3],
[-2,0,2/5,0,1/2,0,0,0,-1/3,-1/2]])
\end{sageblock}

La matriz reducida
$$
	\sage{M.echelon_form()}
$$

\begin{align*}
	x_1 &= 
	x_2 &= 
	x_3 &= 
\end{align*}

% Fin del Ejercicio 42


\begin{ejer} Sea $K$ el cuerpo de 29 elementos.
\newline
\noindent\begin{minipage}{\textwidth}
\begin{tcolorbox}[colback = green!20!white,title=Versión Núcleo]
Determina de entre los vectores columna de la matriz $A$, cuales de ellos están en el núcleo de la aplicación lineal $f:K^{3} \to K^{4}$ siendo  $$ M(f) = \left[\begin{array}{rrr}
21 & 5 & 6 \\
1 & 28 & 23 \\
3 & 4 & 22 \\
23 & 24 & 27
\end{array}\right] $$ y $A$ la matriz que se da a continuación:\end{tcolorbox}
\end{minipage} \newline
\noindent\begin{minipage}{\textwidth}
\begin{tcolorbox}[colback = blue!20!white,title=Versión Anulador]
Determina de entre los vectores columna de la matriz $A$, cuales de ellos están en el anulador por la derecha de la matriz $$ \left[\begin{array}{rrr}
21 & 5 & 6 \\
1 & 28 & 23 \\
3 & 4 & 22 \\
23 & 24 & 27
\end{array}\right] $$ siendo $A$ la matriz que se da a continuación:\end{tcolorbox}
\end{minipage} \newline
\noindent\begin{minipage}{\textwidth} 
\begin{tcolorbox}[colback = red!20!white,title=Versión Ecuaciones Implícitas]
Determina de entre los vectores columna de la matriz $A$, cuales de ellos están en espacio vectorial definido en forma implícita por las siguientes ecuaciones:
\[ 21 x_{0} + 5 x_{1} + 6 x_{2} = 0 \]
\[ x_{0} - x_{1} + 23 x_{2} = 0 \]
\[ 3 x_{0} + 4 x_{1} + 22 x_{2} = 0 \]
\[ 23 x_{0} + 24 x_{1} + 27 x_{2} = 0 \]
Siendo $A$ la matriz que se da a continuación:
\end{tcolorbox}
\end{minipage}
\[ A = \left[\begin{array}{rrrrrrrrrr}
19 & 13 & 0 & 15 & 22 & 5 & 2 & 3 & 18 & 9 \\
26 & 16 & 18 & 19 & 22 & 21 & 18 & 27 & 21 & 23 \\
23 & 11 & 22 & 9 & 4 & 27 & 7 & 25 & 23 & 17
\end{array}\right] \in {\bf M}_{3\times 10}({\mathbb Z}_{29})\]
\end{ejer}

{\it Soluci\'on:}
% Escribe tu soluci\'on para el Ejercicio 43

\begin{sageblock}
matrix(Zmod(29),[[21,5,6],
[1,28,23],
[3,4,22],
[23,24,27]])
matrix(Zmod(29),[[19,13,0,15,22,5,2,3,18,9],
[26,16,18,19,22,21,18,27,21,23],
[23,11,22,9,4,27,7,25,23,17]])
\end{sageblock}

% Fin del Ejercicio 43


\begin{ejer} Sea $K$ el cuerpo de 11 elementos.
\newline
\noindent\begin{minipage}{\textwidth}
\begin{tcolorbox}[colback = green!20!white,title=Versión Núcleo]
Determina de entre los vectores columna de la matriz $A$, cuales de ellos están en el núcleo de la aplicación lineal $f:K^{4} \to K^{3}$ siendo  $$ M(f) = \left[\begin{array}{rrrr}
7 & 1 & 0 & 1 \\
10 & 0 & 7 & 10 \\
3 & 7 & 3 & 5
\end{array}\right] $$ y $A$ la matriz que se da a continuación:\end{tcolorbox}
\end{minipage} \newline
\noindent\begin{minipage}{\textwidth}
\begin{tcolorbox}[colback = blue!20!white,title=Versión Anulador]
Determina de entre los vectores columna de la matriz $A$, cuales de ellos están en el anulador por la derecha de la matriz $$ \left[\begin{array}{rrrr}
7 & 1 & 0 & 1 \\
10 & 0 & 7 & 10 \\
3 & 7 & 3 & 5
\end{array}\right] $$ siendo $A$ la matriz que se da a continuación:\end{tcolorbox}
\end{minipage} \newline
\noindent\begin{minipage}{\textwidth} 
\begin{tcolorbox}[colback = red!20!white,title=Versión Ecuaciones Implícitas]
Determina de entre los vectores columna de la matriz $A$, cuales de ellos están en espacio vectorial definido en forma implícita por las siguientes ecuaciones:
\[ 7 x_{0} + x_{1} + x_{3} = 0 \]
\[ -x_{0} + 7 x_{2} - x_{3} = 0 \]
\[ 3 x_{0} + 7 x_{1} + 3 x_{2} + 5 x_{3} = 0 \]
Siendo $A$ la matriz que se da a continuación:
\end{tcolorbox}
\end{minipage}
\[ A = \left[\begin{array}{rrrrrrrrr}
6 & 2 & 3 & 2 & 1 & 10 & 9 & 9 & 4 \\
0 & 5 & 0 & 7 & 1 & 1 & 3 & 0 & 2 \\
9 & 2 & 9 & 2 & 7 & 7 & 6 & 9 & 5 \\
2 & 4 & 5 & 1 & 5 & 6 & 0 & 10 & 7
\end{array}\right] \in {\bf M}_{4\times 9}({\mathbb Z}_{11})\]
\end{ejer}

{\it Soluci\'on:}
% Escribe tu soluci\'on para el Ejercicio 44

\begin{sageblock}
matrix(Zmod(11),[[7,1,0,1],
[10,0,7,10],
[3,7,3,5]])
matrix(Zmod(11),[[6,2,3,2,1,10,9,9,4],
[0,5,0,7,1,1,3,0,2],
[9,2,9,2,7,7,6,9,5],
[2,4,5,1,5,6,0,10,7]])
\end{sageblock}

% Fin del Ejercicio 44


\begin{ejer} Sea $K$ el cuerpo de 19 elementos.
\newline
\noindent\begin{minipage}{\textwidth}
\begin{tcolorbox}[colback = green!20!white,title=Versión Núcleo]
Determina de entre los vectores columna de la matriz $A$, cuales de ellos están en el núcleo de la aplicación lineal $f:K^{3} \to K^{5}$ siendo  $$ M(f) = \left[\begin{array}{rrr}
0 & 0 & 16 \\
6 & 0 & 9 \\
17 & 0 & 5 \\
5 & 0 & 9 \\
11 & 0 & 11
\end{array}\right] $$ y $A$ la matriz que se da a continuación:\end{tcolorbox}
\end{minipage} \newline
\noindent\begin{minipage}{\textwidth}
\begin{tcolorbox}[colback = blue!20!white,title=Versión Anulador]
Determina de entre los vectores columna de la matriz $A$, cuales de ellos están en el anulador por la derecha de la matriz $$ \left[\begin{array}{rrr}
0 & 0 & 16 \\
6 & 0 & 9 \\
17 & 0 & 5 \\
5 & 0 & 9 \\
11 & 0 & 11
\end{array}\right] $$ siendo $A$ la matriz que se da a continuación:\end{tcolorbox}
\end{minipage} \newline
\noindent\begin{minipage}{\textwidth} 
\begin{tcolorbox}[colback = red!20!white,title=Versión Ecuaciones Implícitas]
Determina de entre los vectores columna de la matriz $A$, cuales de ellos están en espacio vectorial definido en forma implícita por las siguientes ecuaciones:
\[ 16 x_{2} = 0 \]
\[ 6 x_{0} + 9 x_{2} = 0 \]
\[ 17 x_{0} + 5 x_{2} = 0 \]
\[ 5 x_{0} + 9 x_{2} = 0 \]
\[ 11 x_{0} + 11 x_{2} = 0 \]
Siendo $A$ la matriz que se da a continuación:
\end{tcolorbox}
\end{minipage}
\[ A = \left[\begin{array}{rrrrrrrrrr}
15 & 0 & 13 & 0 & 0 & 1 & 0 & 0 & 2 & 5 \\
13 & 2 & 18 & 18 & 4 & 16 & 6 & 1 & 10 & 7 \\
12 & 0 & 13 & 0 & 0 & 8 & 0 & 0 & 5 & 17
\end{array}\right] \in {\bf M}_{3\times 10}({\mathbb Z}_{19})\]
\end{ejer}

{\it Soluci\'on:}
% Escribe tu soluci\'on para el Ejercicio 45

\begin{sageblock}
matrix(Zmod(19),[[0,0,16],
[6,0,9],
[17,0,5],
[5,0,9],
[11,0,11]])
matrix(Zmod(19),[[15,0,13,0,0,1,0,0,2,5],
[13,2,18,18,4,16,6,1,10,7],
[12,0,13,0,0,8,0,0,5,17]])
\end{sageblock}

% Fin del Ejercicio 45


\begin{ejer} Sea $K$ el cuerpo de 31 elementos.
\newline
\noindent\begin{minipage}{\textwidth}
\begin{tcolorbox}[colback = green!20!white,title=Versión Núcleo]
Determina de entre los vectores columna de la matriz $A$, cuales de ellos están en el núcleo de la aplicación lineal $f:K^{4} \to K^{3}$ siendo  $$ M(f) = \left[\begin{array}{rrrr}
11 & 23 & 9 & 23 \\
22 & 15 & 18 & 15 \\
29 & 24 & 24 & 27
\end{array}\right] $$ y $A$ la matriz que se da a continuación:\end{tcolorbox}
\end{minipage} \newline
\noindent\begin{minipage}{\textwidth}
\begin{tcolorbox}[colback = blue!20!white,title=Versión Anulador]
Determina de entre los vectores columna de la matriz $A$, cuales de ellos están en el anulador por la derecha de la matriz $$ \left[\begin{array}{rrrr}
11 & 23 & 9 & 23 \\
22 & 15 & 18 & 15 \\
29 & 24 & 24 & 27
\end{array}\right] $$ siendo $A$ la matriz que se da a continuación:\end{tcolorbox}
\end{minipage} \newline
\noindent\begin{minipage}{\textwidth} 
\begin{tcolorbox}[colback = red!20!white,title=Versión Ecuaciones Implícitas]
Determina de entre los vectores columna de la matriz $A$, cuales de ellos están en espacio vectorial definido en forma implícita por las siguientes ecuaciones:
\[ 11 x_{0} + 23 x_{1} + 9 x_{2} + 23 x_{3} = 0 \]
\[ 22 x_{0} + 15 x_{1} + 18 x_{2} + 15 x_{3} = 0 \]
\[ 29 x_{0} + 24 x_{1} + 24 x_{2} + 27 x_{3} = 0 \]
Siendo $A$ la matriz que se da a continuación:
\end{tcolorbox}
\end{minipage}
\[ A = \left[\begin{array}{rrrrrrrrrr}
14 & 12 & 27 & 15 & 26 & 5 & 2 & 12 & 12 & 18 \\
23 & 14 & 0 & 1 & 23 & 26 & 4 & 7 & 21 & 22 \\
28 & 2 & 5 & 19 & 4 & 10 & 28 & 20 & 15 & 11 \\
20 & 28 & 9 & 12 & 25 & 13 & 0 & 1 & 24 & 25
\end{array}\right] \in {\bf M}_{4\times 10}({\mathbb Z}_{31})\]
\end{ejer}

{\it Soluci\'on:}
% Escribe tu soluci\'on para el Ejercicio 46

\begin{sageblock}
matrix(Zmod(31),[[11,23,9,23],
[22,15,18,15],
[29,24,24,27]])
matrix(Zmod(31),[[14,12,27,15,26,5,2,12,12,18],
[23,14,0,1,23,26,4,7,21,22],
[28,2,5,19,4,10,28,20,15,11],
[20,28,9,12,25,13,0,1,24,25]])
\end{sageblock}

% Fin del Ejercicio 46


\begin{ejer} Sea $K$ el cuerpo de los n\'umeros reales.
\newline
\noindent\begin{minipage}{\textwidth}
\begin{tcolorbox}[colback = green!20!white,title=Versión Núcleo]
Determina de entre los vectores columna de la matriz $A$, cuales de ellos están en el núcleo de la aplicación lineal $f:K^{2} \to K^{4}$ siendo  $$ M(f) = \left[\begin{array}{rr}
4 & 8 \\
4 & 8 \\
-1 & -2 \\
-2 & -4
\end{array}\right] $$ y $A$ la matriz que se da a continuación:\end{tcolorbox}
\end{minipage} \newline
\noindent\begin{minipage}{\textwidth}
\begin{tcolorbox}[colback = blue!20!white,title=Versión Anulador]
Determina de entre los vectores columna de la matriz $A$, cuales de ellos están en el anulador por la derecha de la matriz $$ \left[\begin{array}{rr}
4 & 8 \\
4 & 8 \\
-1 & -2 \\
-2 & -4
\end{array}\right] $$ siendo $A$ la matriz que se da a continuación:\end{tcolorbox}
\end{minipage} \newline
\noindent\begin{minipage}{\textwidth} 
\begin{tcolorbox}[colback = red!20!white,title=Versión Ecuaciones Implícitas]
Determina de entre los vectores columna de la matriz $A$, cuales de ellos están en espacio vectorial definido en forma implícita por las siguientes ecuaciones:
\[ 4 x_{0} + 8 x_{1} = 0 \]
\[ 4 x_{0} + 8 x_{1} = 0 \]
\[ -x_{0} - 2 x_{1} = 0 \]
\[ -2 x_{0} - 4 x_{1} = 0 \]
Siendo $A$ la matriz que se da a continuación:
\end{tcolorbox}
\end{minipage}
\[ A = \left[\begin{array}{rrrrrrrrr}
\frac{1}{26} & -\frac{49}{8} & \frac{2}{3} & 50 & -\frac{1}{8} & -\frac{1}{3} & \frac{2}{5} & -64 & -1 \\
-1 & -\frac{1}{3} & -\frac{1}{3} & -25 & \frac{1}{16} & 1 & 1 & \frac{1}{2} & \frac{1}{2}
\end{array}\right] \in {\bf M}_{2\times 9}({\mathbb R})\]
\end{ejer}

{\it Soluci\'on:}
% Escribe tu soluci\'on para el Ejercicio 47

\begin{sageblock}
matrix(QQ,[[4,8],
[4,8],
[-1,-2],
[-2,-4]])
matrix(QQ,[[1/26,-49/8,2/3,50,-1/8,-1/3,2/5,-64,-1],
[-1,-1/3,-1/3,-25,1/16,1,1,1/2,1/2]])
\end{sageblock}

% Fin del Ejercicio 47


\begin{ejer} Sea $K$ el cuerpo de 29 elementos.
\newline
\noindent\begin{minipage}{\textwidth}
\begin{tcolorbox}[colback = green!20!white,title=Versión Núcleo]
Determina de entre los vectores columna de la matriz $A$, cuales de ellos están en el núcleo de la aplicación lineal $f:K^{2} \to K^{3}$ siendo  $$ M(f) = \left[\begin{array}{rr}
9 & 12 \\
4 & 15 \\
9 & 12
\end{array}\right] $$ y $A$ la matriz que se da a continuación:\end{tcolorbox}
\end{minipage} \newline
\noindent\begin{minipage}{\textwidth}
\begin{tcolorbox}[colback = blue!20!white,title=Versión Anulador]
Determina de entre los vectores columna de la matriz $A$, cuales de ellos están en el anulador por la derecha de la matriz $$ \left[\begin{array}{rr}
9 & 12 \\
4 & 15 \\
9 & 12
\end{array}\right] $$ siendo $A$ la matriz que se da a continuación:\end{tcolorbox}
\end{minipage} \newline
\noindent\begin{minipage}{\textwidth} 
\begin{tcolorbox}[colback = red!20!white,title=Versión Ecuaciones Implícitas]
Determina de entre los vectores columna de la matriz $A$, cuales de ellos están en espacio vectorial definido en forma implícita por las siguientes ecuaciones:
\[ 9 x_{0} + 12 x_{1} = 0 \]
\[ 4 x_{0} + 15 x_{1} = 0 \]
\[ 9 x_{0} + 12 x_{1} = 0 \]
Siendo $A$ la matriz que se da a continuación:
\end{tcolorbox}
\end{minipage}
\[ A = \left[\begin{array}{rrrrrrrrr}
19 & 12 & 7 & 18 & 9 & 10 & 22 & 1 & 12 \\
25 & 15 & 25 & 0 & 27 & 7 & 27 & 21 & 20
\end{array}\right] \in {\bf M}_{2\times 9}({\mathbb Z}_{29})\]
\end{ejer}

{\it Soluci\'on:}
% Escribe tu soluci\'on para el Ejercicio 48

\begin{sageblock}
matrix(Zmod(29),[[9,12],
[4,15],
[9,12]])
matrix(Zmod(29),[[19,12,7,18,9,10,22,1,12],
[25,15,25,0,27,7,27,21,20]])
\end{sageblock}

% Fin del Ejercicio 48


\begin{ejer} Sea $K$ el cuerpo de 5 elementos.
\newline
\noindent\begin{minipage}{\textwidth}
\begin{tcolorbox}[colback = green!20!white,title=Versión Núcleo]
Determina de entre los vectores columna de la matriz $A$, cuales de ellos están en el núcleo de la aplicación lineal $f:K^{3} \to K^{2}$ siendo  $$ M(f) = \left[\begin{array}{rrr}
1 & 0 & 4 \\
4 & 0 & 1
\end{array}\right] $$ y $A$ la matriz que se da a continuación:\end{tcolorbox}
\end{minipage} \newline
\noindent\begin{minipage}{\textwidth}
\begin{tcolorbox}[colback = blue!20!white,title=Versión Anulador]
Determina de entre los vectores columna de la matriz $A$, cuales de ellos están en el anulador por la derecha de la matriz $$ \left[\begin{array}{rrr}
1 & 0 & 4 \\
4 & 0 & 1
\end{array}\right] $$ siendo $A$ la matriz que se da a continuación:\end{tcolorbox}
\end{minipage} \newline
\noindent\begin{minipage}{\textwidth} 
\begin{tcolorbox}[colback = red!20!white,title=Versión Ecuaciones Implícitas]
Determina de entre los vectores columna de la matriz $A$, cuales de ellos están en espacio vectorial definido en forma implícita por las siguientes ecuaciones:
\[ x_{0} - x_{2} = 0 \]
\[ -x_{0} + x_{2} = 0 \]
Siendo $A$ la matriz que se da a continuación:
\end{tcolorbox}
\end{minipage}
\[ A = \left[\begin{array}{rrrrrrrrrr}
2 & 1 & 3 & 1 & 4 & 0 & 0 & 4 & 0 & 4 \\
4 & 4 & 4 & 4 & 2 & 2 & 2 & 1 & 1 & 1 \\
4 & 1 & 2 & 3 & 4 & 1 & 0 & 0 & 0 & 4
\end{array}\right] \in {\bf M}_{3\times 10}({\mathbb Z}_{5})\]
\end{ejer}

{\it Soluci\'on:}
% Escribe tu soluci\'on para el Ejercicio 49

\begin{sageblock}
matrix(Zmod(5),[[1,0,4],
[4,0,1]])
matrix(Zmod(5),[[2,1,3,1,4,0,0,4,0,4],
[4,4,4,4,2,2,2,1,1,1],
[4,1,2,3,4,1,0,0,0,4]])
\end{sageblock}

% Fin del Ejercicio 49


\begin{ejer} Sea $K$ el cuerpo de 43 elementos.
\newline
\noindent\begin{minipage}{\textwidth}
\begin{tcolorbox}[colback = green!20!white,title=Versión Núcleo]
Determina de entre los vectores columna de la matriz $A$, cuales de ellos están en el núcleo de la aplicación lineal $f:K^{3} \to K^{4}$ siendo  $$ M(f) = \left[\begin{array}{rrr}
30 & 27 & 35 \\
40 & 9 & 22 \\
16 & 33 & 14 \\
22 & 23 & 15
\end{array}\right] $$ y $A$ la matriz que se da a continuación:\end{tcolorbox}
\end{minipage} \newline
\noindent\begin{minipage}{\textwidth}
\begin{tcolorbox}[colback = blue!20!white,title=Versión Anulador]
Determina de entre los vectores columna de la matriz $A$, cuales de ellos están en el anulador por la derecha de la matriz $$ \left[\begin{array}{rrr}
30 & 27 & 35 \\
40 & 9 & 22 \\
16 & 33 & 14 \\
22 & 23 & 15
\end{array}\right] $$ siendo $A$ la matriz que se da a continuación:\end{tcolorbox}
\end{minipage} \newline
\noindent\begin{minipage}{\textwidth} 
\begin{tcolorbox}[colback = red!20!white,title=Versión Ecuaciones Implícitas]
Determina de entre los vectores columna de la matriz $A$, cuales de ellos están en espacio vectorial definido en forma implícita por las siguientes ecuaciones:
\[ 30 x_{0} + 27 x_{1} + 35 x_{2} = 0 \]
\[ 40 x_{0} + 9 x_{1} + 22 x_{2} = 0 \]
\[ 16 x_{0} + 33 x_{1} + 14 x_{2} = 0 \]
\[ 22 x_{0} + 23 x_{1} + 15 x_{2} = 0 \]
Siendo $A$ la matriz que se da a continuación:
\end{tcolorbox}
\end{minipage}
\[ A = \left[\begin{array}{rrrrrrrrrr}
25 & 34 & 11 & 11 & 10 & 38 & 12 & 26 & 29 & 0 \\
30 & 33 & 22 & 17 & 26 & 4 & 33 & 32 & 37 & 0 \\
32 & 40 & 16 & 18 & 29 & 27 & 38 & 1 & 24 & 0
\end{array}\right] \in {\bf M}_{3\times 10}({\mathbb Z}_{43})\]
\end{ejer}

{\it Soluci\'on:}
% Escribe tu soluci\'on para el Ejercicio 50

\begin{sageblock}
matrix(Zmod(43),[[30,27,35],
[40,9,22],
[16,33,14],
[22,23,15]])
matrix(Zmod(43),[[25,34,11,11,10,38,12,26,29,0],
[30,33,22,17,26,4,33,32,37,0],
[32,40,16,18,29,27,38,1,24,0]])
\end{sageblock}

% Fin del Ejercicio 50


\begin{ejer} Sea $K$ el cuerpo de 19 elementos.
\newline
\noindent\begin{minipage}{\textwidth}
\begin{tcolorbox}[colback = green!20!white,title=Versión Núcleo]
Determina de entre los vectores columna de la matriz $A$, cuales de ellos están en el núcleo de la aplicación lineal $f:K^{5} \to K^{3}$ siendo  $$ M(f) = \left[\begin{array}{rrrrr}
11 & 16 & 14 & 10 & 6 \\
9 & 8 & 8 & 5 & 8 \\
6 & 1 & 18 & 3 & 10
\end{array}\right] $$ y $A$ la matriz que se da a continuación:\end{tcolorbox}
\end{minipage} \newline
\noindent\begin{minipage}{\textwidth}
\begin{tcolorbox}[colback = blue!20!white,title=Versión Anulador]
Determina de entre los vectores columna de la matriz $A$, cuales de ellos están en el anulador por la derecha de la matriz $$ \left[\begin{array}{rrrrr}
11 & 16 & 14 & 10 & 6 \\
9 & 8 & 8 & 5 & 8 \\
6 & 1 & 18 & 3 & 10
\end{array}\right] $$ siendo $A$ la matriz que se da a continuación:\end{tcolorbox}
\end{minipage} \newline
\noindent\begin{minipage}{\textwidth} 
\begin{tcolorbox}[colback = red!20!white,title=Versión Ecuaciones Implícitas]
Determina de entre los vectores columna de la matriz $A$, cuales de ellos están en espacio vectorial definido en forma implícita por las siguientes ecuaciones:
\[ 11 x_{0} + 16 x_{1} + 14 x_{2} + 10 x_{3} + 6 x_{4} = 0 \]
\[ 9 x_{0} + 8 x_{1} + 8 x_{2} + 5 x_{3} + 8 x_{4} = 0 \]
\[ 6 x_{0} + x_{1} - x_{2} + 3 x_{3} + 10 x_{4} = 0 \]
Siendo $A$ la matriz que se da a continuación:
\end{tcolorbox}
\end{minipage}
\[ A = \left[\begin{array}{rrrrrrrrrr}
1 & 9 & 12 & 8 & 7 & 9 & 16 & 0 & 9 & 14 \\
1 & 5 & 10 & 6 & 15 & 7 & 12 & 10 & 9 & 18 \\
1 & 3 & 11 & 3 & 16 & 8 & 17 & 9 & 6 & 5 \\
8 & 2 & 18 & 9 & 7 & 15 & 17 & 0 & 6 & 10 \\
13 & 13 & 16 & 14 & 9 & 13 & 5 & 7 & 2 & 12
\end{array}\right] \in {\bf M}_{5\times 10}({\mathbb Z}_{19})\]
\end{ejer}

{\it Soluci\'on:}
% Escribe tu soluci\'on para el Ejercicio 51

\begin{sageblock}
matrix(Zmod(19),[[11,16,14,10,6],
[9,8,8,5,8],
[6,1,18,3,10]])
matrix(Zmod(19),[[1,9,12,8,7,9,16,0,9,14],
[1,5,10,6,15,7,12,10,9,18],
[1,3,11,3,16,8,17,9,6,5],
[8,2,18,9,7,15,17,0,6,10],
[13,13,16,14,9,13,5,7,2,12]])
\end{sageblock}

% Fin del Ejercicio 51


\begin{ejer} Sea $K$ el cuerpo de 31 elementos.
\newline
\noindent\begin{minipage}{\textwidth}
\begin{tcolorbox}[colback = green!20!white,title=Versión Núcleo]
Determina de entre los vectores columna de la matriz $A$, cuales de ellos están en el núcleo de la aplicación lineal $f:K^{3} \to K^{5}$ siendo  $$ M(f) = \left[\begin{array}{rrr}
27 & 14 & 23 \\
26 & 16 & 2 \\
10 & 10 & 1 \\
18 & 27 & 29 \\
8 & 10 & 22
\end{array}\right] $$ y $A$ la matriz que se da a continuación:\end{tcolorbox}
\end{minipage} \newline
\noindent\begin{minipage}{\textwidth}
\begin{tcolorbox}[colback = blue!20!white,title=Versión Anulador]
Determina de entre los vectores columna de la matriz $A$, cuales de ellos están en el anulador por la derecha de la matriz $$ \left[\begin{array}{rrr}
27 & 14 & 23 \\
26 & 16 & 2 \\
10 & 10 & 1 \\
18 & 27 & 29 \\
8 & 10 & 22
\end{array}\right] $$ siendo $A$ la matriz que se da a continuación:\end{tcolorbox}
\end{minipage} \newline
\noindent\begin{minipage}{\textwidth} 
\begin{tcolorbox}[colback = red!20!white,title=Versión Ecuaciones Implícitas]
Determina de entre los vectores columna de la matriz $A$, cuales de ellos están en espacio vectorial definido en forma implícita por las siguientes ecuaciones:
\[ 27 x_{0} + 14 x_{1} + 23 x_{2} = 0 \]
\[ 26 x_{0} + 16 x_{1} + 2 x_{2} = 0 \]
\[ 10 x_{0} + 10 x_{1} + x_{2} = 0 \]
\[ 18 x_{0} + 27 x_{1} + 29 x_{2} = 0 \]
\[ 8 x_{0} + 10 x_{1} + 22 x_{2} = 0 \]
Siendo $A$ la matriz que se da a continuación:
\end{tcolorbox}
\end{minipage}
\[ A = \left[\begin{array}{rrrrrrrrrr}
24 & 1 & 17 & 15 & 21 & 23 & 2 & 3 & 30 & 26 \\
12 & 15 & 22 & 7 & 14 & 19 & 3 & 20 & 14 & 7 \\
12 & 5 & 26 & 28 & 0 & 14 & 12 & 18 & 25 & 29
\end{array}\right] \in {\bf M}_{3\times 10}({\mathbb Z}_{31})\]
\end{ejer}

{\it Soluci\'on:}
% Escribe tu soluci\'on para el Ejercicio 52

\begin{sageblock}
matrix(Zmod(31),[[27,14,23],
[26,16,2],
[10,10,1],
[18,27,29],
[8,10,22]])
matrix(Zmod(31),[[24,1,17,15,21,23,2,3,30,26],
[12,15,22,7,14,19,3,20,14,7],
[12,5,26,28,0,14,12,18,25,29]])
\end{sageblock}

% Fin del Ejercicio 52


\begin{ejer} Sea $K$ el cuerpo de 5 elementos.
\newline
\noindent\begin{minipage}{\textwidth}
\begin{tcolorbox}[colback = green!20!white,title=Versión Núcleo]
Determina de entre los vectores columna de la matriz $A$, cuales de ellos están en el núcleo de la aplicación lineal $f:K^{2} \to K^{4}$ siendo  $$ M(f) = \left[\begin{array}{rr}
1 & 3 \\
1 & 3 \\
0 & 0 \\
2 & 1
\end{array}\right] $$ y $A$ la matriz que se da a continuación:\end{tcolorbox}
\end{minipage} \newline
\noindent\begin{minipage}{\textwidth}
\begin{tcolorbox}[colback = blue!20!white,title=Versión Anulador]
Determina de entre los vectores columna de la matriz $A$, cuales de ellos están en el anulador por la derecha de la matriz $$ \left[\begin{array}{rr}
1 & 3 \\
1 & 3 \\
0 & 0 \\
2 & 1
\end{array}\right] $$ siendo $A$ la matriz que se da a continuación:\end{tcolorbox}
\end{minipage} \newline
\noindent\begin{minipage}{\textwidth} 
\begin{tcolorbox}[colback = red!20!white,title=Versión Ecuaciones Implícitas]
Determina de entre los vectores columna de la matriz $A$, cuales de ellos están en espacio vectorial definido en forma implícita por las siguientes ecuaciones:
\[ x_{0} + 3 x_{1} = 0 \]
\[ x_{0} + 3 x_{1} = 0 \]
\[ 0 = 0 \]
\[ 2 x_{0} + x_{1} = 0 \]
Siendo $A$ la matriz que se da a continuación:
\end{tcolorbox}
\end{minipage}
\[ A = \left[\begin{array}{rrrrrrrr}
1 & 2 & 3 & 0 & 2 & 4 & 3 & 4 \\
3 & 2 & 4 & 1 & 1 & 1 & 3 & 2
\end{array}\right] \in {\bf M}_{2\times 8}({\mathbb Z}_{5})\]
\end{ejer}

{\it Soluci\'on:}
% Escribe tu soluci\'on para el Ejercicio 53

\begin{sageblock}
matrix(Zmod(5),[[1,3],
[1,3],
[0,0],
[2,1]])
matrix(Zmod(5),[[1,2,3,0,2,4,3,4],
[3,2,4,1,1,1,3,2]])
\end{sageblock}

% Fin del Ejercicio 53


\begin{ejer} Sea $K$ el cuerpo de 19 elementos.
\newline
\noindent\begin{minipage}{\textwidth}
\begin{tcolorbox}[colback = green!20!white,title=Versión Núcleo]
Determina de entre los vectores columna de la matriz $A$, cuales de ellos están en el núcleo de la aplicación lineal $f:K^{2} \to K^{4}$ siendo  $$ M(f) = \left[\begin{array}{rr}
10 & 4 \\
11 & 12 \\
15 & 6 \\
10 & 4
\end{array}\right] $$ y $A$ la matriz que se da a continuación:\end{tcolorbox}
\end{minipage} \newline
\noindent\begin{minipage}{\textwidth}
\begin{tcolorbox}[colback = blue!20!white,title=Versión Anulador]
Determina de entre los vectores columna de la matriz $A$, cuales de ellos están en el anulador por la derecha de la matriz $$ \left[\begin{array}{rr}
10 & 4 \\
11 & 12 \\
15 & 6 \\
10 & 4
\end{array}\right] $$ siendo $A$ la matriz que se da a continuación:\end{tcolorbox}
\end{minipage} \newline
\noindent\begin{minipage}{\textwidth} 
\begin{tcolorbox}[colback = red!20!white,title=Versión Ecuaciones Implícitas]
Determina de entre los vectores columna de la matriz $A$, cuales de ellos están en espacio vectorial definido en forma implícita por las siguientes ecuaciones:
\[ 10 x_{0} + 4 x_{1} = 0 \]
\[ 11 x_{0} + 12 x_{1} = 0 \]
\[ 15 x_{0} + 6 x_{1} = 0 \]
\[ 10 x_{0} + 4 x_{1} = 0 \]
Siendo $A$ la matriz que se da a continuación:
\end{tcolorbox}
\end{minipage}
\[ A = \left[\begin{array}{rrrrrrrrr}
10 & 13 & 1 & 1 & 8 & 9 & 15 & 3 & 9 \\
4 & 18 & 1 & 7 & 18 & 12 & 13 & 2 & 6
\end{array}\right] \in {\bf M}_{2\times 9}({\mathbb Z}_{19})\]
\end{ejer}

{\it Soluci\'on:}
% Escribe tu soluci\'on para el Ejercicio 54

\begin{sageblock}
matrix(Zmod(19),[[10,4],
[11,12],
[15,6],
[10,4]])
matrix(Zmod(19),[[10,13,1,1,8,9,15,3,9],
[4,18,1,7,18,12,13,2,6]])
\end{sageblock}

% Fin del Ejercicio 54


\begin{ejer} Sea $K$ el cuerpo de 47 elementos.
\newline
\noindent\begin{minipage}{\textwidth}
\begin{tcolorbox}[colback = green!20!white,title=Versión Núcleo]
Determina de entre los vectores columna de la matriz $A$, cuales de ellos están en el núcleo de la aplicación lineal $f:K^{4} \to K^{3}$ siendo  $$ M(f) = \left[\begin{array}{rrrr}
7 & 37 & 8 & 0 \\
7 & 3 & 35 & 34 \\
21 & 11 & 26 & 6
\end{array}\right] $$ y $A$ la matriz que se da a continuación:\end{tcolorbox}
\end{minipage} \newline
\noindent\begin{minipage}{\textwidth}
\begin{tcolorbox}[colback = blue!20!white,title=Versión Anulador]
Determina de entre los vectores columna de la matriz $A$, cuales de ellos están en el anulador por la derecha de la matriz $$ \left[\begin{array}{rrrr}
7 & 37 & 8 & 0 \\
7 & 3 & 35 & 34 \\
21 & 11 & 26 & 6
\end{array}\right] $$ siendo $A$ la matriz que se da a continuación:\end{tcolorbox}
\end{minipage} \newline
\noindent\begin{minipage}{\textwidth} 
\begin{tcolorbox}[colback = red!20!white,title=Versión Ecuaciones Implícitas]
Determina de entre los vectores columna de la matriz $A$, cuales de ellos están en espacio vectorial definido en forma implícita por las siguientes ecuaciones:
\[ 7 x_{0} + 37 x_{1} + 8 x_{2} = 0 \]
\[ 7 x_{0} + 3 x_{1} + 35 x_{2} + 34 x_{3} = 0 \]
\[ 21 x_{0} + 11 x_{1} + 26 x_{2} + 6 x_{3} = 0 \]
Siendo $A$ la matriz que se da a continuación:
\end{tcolorbox}
\end{minipage}
\[ A = \left[\begin{array}{rrrrrrrrrr}
28 & 41 & 45 & 46 & 40 & 8 & 40 & 44 & 42 & 23 \\
5 & 31 & 40 & 38 & 1 & 10 & 27 & 46 & 22 & 28 \\
17 & 44 & 40 & 10 & 34 & 29 & 34 & 12 & 1 & 46 \\
15 & 32 & 11 & 8 & 11 & 16 & 0 & 25 & 8 & 11
\end{array}\right] \in {\bf M}_{4\times 10}({\mathbb Z}_{47})\]
\end{ejer}

{\it Soluci\'on:}
% Escribe tu soluci\'on para el Ejercicio 55

\begin{sageblock}
matrix(Zmod(47),[[7,37,8,0],
[7,3,35,34],
[21,11,26,6]])
matrix(Zmod(47),[[28,41,45,46,40,8,40,44,42,23],
[5,31,40,38,1,10,27,46,22,28],
[17,44,40,10,34,29,34,12,1,46],
[15,32,11,8,11,16,0,25,8,11]])
\end{sageblock}

% Fin del Ejercicio 55


\begin{ejer} Sea $K$ el cuerpo de 17 elementos.
\newline
\noindent\begin{minipage}{\textwidth}
\begin{tcolorbox}[colback = green!20!white,title=Versión Núcleo]
Determina de entre los vectores columna de la matriz $A$, cuales de ellos están en el núcleo de la aplicación lineal $f:K^{3} \to K^{5}$ siendo  $$ M(f) = \left[\begin{array}{rrr}
8 & 9 & 3 \\
10 & 5 & 6 \\
12 & 6 & 14 \\
0 & 10 & 10 \\
2 & 16 & 6
\end{array}\right] $$ y $A$ la matriz que se da a continuación:\end{tcolorbox}
\end{minipage} \newline
\noindent\begin{minipage}{\textwidth}
\begin{tcolorbox}[colback = blue!20!white,title=Versión Anulador]
Determina de entre los vectores columna de la matriz $A$, cuales de ellos están en el anulador por la derecha de la matriz $$ \left[\begin{array}{rrr}
8 & 9 & 3 \\
10 & 5 & 6 \\
12 & 6 & 14 \\
0 & 10 & 10 \\
2 & 16 & 6
\end{array}\right] $$ siendo $A$ la matriz que se da a continuación:\end{tcolorbox}
\end{minipage} \newline
\noindent\begin{minipage}{\textwidth} 
\begin{tcolorbox}[colback = red!20!white,title=Versión Ecuaciones Implícitas]
Determina de entre los vectores columna de la matriz $A$, cuales de ellos están en espacio vectorial definido en forma implícita por las siguientes ecuaciones:
\[ 8 x_{0} + 9 x_{1} + 3 x_{2} = 0 \]
\[ 10 x_{0} + 5 x_{1} + 6 x_{2} = 0 \]
\[ 12 x_{0} + 6 x_{1} + 14 x_{2} = 0 \]
\[ 10 x_{1} + 10 x_{2} = 0 \]
\[ 2 x_{0} - x_{1} + 6 x_{2} = 0 \]
Siendo $A$ la matriz que se da a continuación:
\end{tcolorbox}
\end{minipage}
\[ A = \left[\begin{array}{rrrrrrrrrr}
15 & 5 & 1 & 0 & 14 & 13 & 9 & 13 & 9 & 6 \\
14 & 3 & 13 & 0 & 4 & 11 & 5 & 6 & 11 & 5 \\
3 & 5 & 12 & 0 & 13 & 6 & 12 & 1 & 16 & 14
\end{array}\right] \in {\bf M}_{3\times 10}({\mathbb Z}_{17})\]
\end{ejer}

{\it Soluci\'on:}
% Escribe tu soluci\'on para el Ejercicio 56

\begin{sageblock}
matrix(Zmod(17),[[8,9,3],
[10,5,6],
[12,6,14],
[0,10,10],
[2,16,6]])
matrix(Zmod(17),[[15,5,1,0,14,13,9,13,9,6],
[14,3,13,0,4,11,5,6,11,5],
[3,5,12,0,13,6,12,1,16,14]])
\end{sageblock}

% Fin del Ejercicio 56


\begin{ejer} Sea $K$ el cuerpo de 41 elementos.
\newline
\noindent\begin{minipage}{\textwidth}
\begin{tcolorbox}[colback = green!20!white,title=Versión Núcleo]
Determina de entre los vectores columna de la matriz $A$, cuales de ellos están en el núcleo de la aplicación lineal $f:K^{2} \to K^{5}$ siendo  $$ M(f) = \left[\begin{array}{rr}
27 & 24 \\
1 & 10 \\
22 & 15 \\
26 & 14 \\
16 & 37
\end{array}\right] $$ y $A$ la matriz que se da a continuación:\end{tcolorbox}
\end{minipage} \newline
\noindent\begin{minipage}{\textwidth}
\begin{tcolorbox}[colback = blue!20!white,title=Versión Anulador]
Determina de entre los vectores columna de la matriz $A$, cuales de ellos están en el anulador por la derecha de la matriz $$ \left[\begin{array}{rr}
27 & 24 \\
1 & 10 \\
22 & 15 \\
26 & 14 \\
16 & 37
\end{array}\right] $$ siendo $A$ la matriz que se da a continuación:\end{tcolorbox}
\end{minipage} \newline
\noindent\begin{minipage}{\textwidth} 
\begin{tcolorbox}[colback = red!20!white,title=Versión Ecuaciones Implícitas]
Determina de entre los vectores columna de la matriz $A$, cuales de ellos están en espacio vectorial definido en forma implícita por las siguientes ecuaciones:
\[ 27 x_{0} + 24 x_{1} = 0 \]
\[ x_{0} + 10 x_{1} = 0 \]
\[ 22 x_{0} + 15 x_{1} = 0 \]
\[ 26 x_{0} + 14 x_{1} = 0 \]
\[ 16 x_{0} + 37 x_{1} = 0 \]
Siendo $A$ la matriz que se da a continuación:
\end{tcolorbox}
\end{minipage}
\[ A = \left[\begin{array}{rrrrrrrrrr}
35 & 39 & 40 & 6 & 35 & 25 & 35 & 22 & 27 & 3 \\
17 & 33 & 37 & 39 & 13 & 32 & 0 & 6 & 28 & 12
\end{array}\right] \in {\bf M}_{2\times 10}({\mathbb Z}_{41})\]
\end{ejer}

{\it Soluci\'on:}
% Escribe tu soluci\'on para el Ejercicio 57

\begin{sageblock}
matrix(Zmod(41),[[27,24],
[1,10],
[22,15],
[26,14],
[16,37]])
matrix(Zmod(41),[[35,39,40,6,35,25,35,22,27,3],
[17,33,37,39,13,32,0,6,28,12]])
\end{sageblock}

% Fin del Ejercicio 57


\begin{ejer} Sea $K$ el cuerpo de los n\'umeros reales.
\newline
\noindent\begin{minipage}{\textwidth}
\begin{tcolorbox}[colback = green!20!white,title=Versión Núcleo]
Determina de entre los vectores columna de la matriz $A$, cuales de ellos están en el núcleo de la aplicación lineal $f:K^{3} \to K^{5}$ siendo  $$ M(f) = \left[\begin{array}{rrr}
-2 & 7 & -1 \\
0 & 1 & -1 \\
1 & -1 & -2 \\
0 & 4 & -4 \\
-1 & 1 & 2
\end{array}\right] $$ y $A$ la matriz que se da a continuación:\end{tcolorbox}
\end{minipage} \newline
\noindent\begin{minipage}{\textwidth}
\begin{tcolorbox}[colback = blue!20!white,title=Versión Anulador]
Determina de entre los vectores columna de la matriz $A$, cuales de ellos están en el anulador por la derecha de la matriz $$ \left[\begin{array}{rrr}
-2 & 7 & -1 \\
0 & 1 & -1 \\
1 & -1 & -2 \\
0 & 4 & -4 \\
-1 & 1 & 2
\end{array}\right] $$ siendo $A$ la matriz que se da a continuación:\end{tcolorbox}
\end{minipage} \newline
\noindent\begin{minipage}{\textwidth} 
\begin{tcolorbox}[colback = red!20!white,title=Versión Ecuaciones Implícitas]
Determina de entre los vectores columna de la matriz $A$, cuales de ellos están en espacio vectorial definido en forma implícita por las siguientes ecuaciones:
\[ -2 x_{0} + 7 x_{1} - x_{2} = 0 \]
\[ x_{1} - x_{2} = 0 \]
\[ x_{0} - x_{1} - 2 x_{2} = 0 \]
\[ 4 x_{1} - 4 x_{2} = 0 \]
\[ -x_{0} + x_{1} + 2 x_{2} = 0 \]
Siendo $A$ la matriz que se da a continuación:
\end{tcolorbox}
\end{minipage}
\[ A = \left[\begin{array}{rrrrrrrr}
0 & \frac{2}{3} & 0 & -\frac{1}{3} & -\frac{13}{8} & 2 & 0 & -3 \\
0 & -\frac{1}{250} & 1 & 2 & -\frac{13}{24} & -5 & 0 & -1 \\
-\frac{2}{5} & \frac{1}{15} & \frac{3}{4} & 0 & -\frac{13}{24} & -3 & 0 & -1
\end{array}\right] \in {\bf M}_{3\times 8}({\mathbb R})\]
\end{ejer}

{\it Soluci\'on:}
% Escribe tu soluci\'on para el Ejercicio 58

\begin{sageblock}
matrix(QQ,[[-2,7,-1],
[0,1,-1],
[1,-1,-2],
[0,4,-4],
[-1,1,2]])
matrix(QQ,[[0,2/3,0,-1/3,-13/8,2,0,-3],
[0,-1/250,1,2,-13/24,-5,0,-1],
[-2/5,1/15,3/4,0,-13/24,-3,0,-1]])
\end{sageblock}

% Fin del Ejercicio 58


\begin{ejer} Sea $K$ el cuerpo de los n\'umeros reales.
\newline
\noindent\begin{minipage}{\textwidth}
\begin{tcolorbox}[colback = green!20!white,title=Versión Núcleo]
Determina de entre los vectores columna de la matriz $A$, cuales de ellos están en el núcleo de la aplicación lineal $f:K^{3} \to K^{5}$ siendo  $$ M(f) = \left[\begin{array}{rrr}
0 & 0 & -1 \\
1 & -4 & -1 \\
0 & 0 & 4 \\
0 & 0 & 4 \\
2 & -8 & -3
\end{array}\right] $$ y $A$ la matriz que se da a continuación:\end{tcolorbox}
\end{minipage} \newline
\noindent\begin{minipage}{\textwidth}
\begin{tcolorbox}[colback = blue!20!white,title=Versión Anulador]
Determina de entre los vectores columna de la matriz $A$, cuales de ellos están en el anulador por la derecha de la matriz $$ \left[\begin{array}{rrr}
0 & 0 & -1 \\
1 & -4 & -1 \\
0 & 0 & 4 \\
0 & 0 & 4 \\
2 & -8 & -3
\end{array}\right] $$ siendo $A$ la matriz que se da a continuación:\end{tcolorbox}
\end{minipage} \newline
\noindent\begin{minipage}{\textwidth} 
\begin{tcolorbox}[colback = red!20!white,title=Versión Ecuaciones Implícitas]
Determina de entre los vectores columna de la matriz $A$, cuales de ellos están en espacio vectorial definido en forma implícita por las siguientes ecuaciones:
\[ -x_{2} = 0 \]
\[ x_{0} - 4 x_{1} - x_{2} = 0 \]
\[ 4 x_{2} = 0 \]
\[ 4 x_{2} = 0 \]
\[ 2 x_{0} - 8 x_{1} - 3 x_{2} = 0 \]
Siendo $A$ la matriz que se da a continuación:
\end{tcolorbox}
\end{minipage}
\[ A = \left[\begin{array}{rrrrrrrrrr}
-\frac{1}{2} & -1 & -1 & 3 & 0 & -\frac{1}{3} & -\frac{2}{3} & -21 & \frac{33}{2} & \frac{1}{14} \\
-\frac{1}{8} & -\frac{1}{4} & -\frac{1}{2} & \frac{3}{4} & -\frac{1}{3} & -1 & -\frac{1}{6} & 0 & \frac{33}{8} & -\frac{1}{3} \\
0 & 0 & -1 & 0 & 13 & -15 & 0 & -\frac{5}{2} & 0 & \frac{1}{3}
\end{array}\right] \in {\bf M}_{3\times 10}({\mathbb R})\]
\end{ejer}

{\it Soluci\'on:}
% Escribe tu soluci\'on para el Ejercicio 59

\begin{sageblock}
matrix(QQ,[[0,0,-1],
[1,-4,-1],
[0,0,4],
[0,0,4],
[2,-8,-3]])
matrix(QQ,[[-1/2,-1,-1,3,0,-1/3,-2/3,-21,33/2,1/14],
[-1/8,-1/4,-1/2,3/4,-1/3,-1,-1/6,0,33/8,-1/3],
[0,0,-1,0,13,-15,0,-5/2,0,1/3]])
\end{sageblock}

% Fin del Ejercicio 59


\begin{ejer} Sea $K$ el cuerpo de 41 elementos.
\newline
\noindent\begin{minipage}{\textwidth}
\begin{tcolorbox}[colback = green!20!white,title=Versión Núcleo]
Determina de entre los vectores columna de la matriz $A$, cuales de ellos están en el núcleo de la aplicación lineal $f:K^{3} \to K^{4}$ siendo  $$ M(f) = \left[\begin{array}{rrr}
38 & 5 & 28 \\
29 & 10 & 12 \\
17 & 30 & 1 \\
6 & 17 & 9
\end{array}\right] $$ y $A$ la matriz que se da a continuación:\end{tcolorbox}
\end{minipage} \newline
\noindent\begin{minipage}{\textwidth}
\begin{tcolorbox}[colback = blue!20!white,title=Versión Anulador]
Determina de entre los vectores columna de la matriz $A$, cuales de ellos están en el anulador por la derecha de la matriz $$ \left[\begin{array}{rrr}
38 & 5 & 28 \\
29 & 10 & 12 \\
17 & 30 & 1 \\
6 & 17 & 9
\end{array}\right] $$ siendo $A$ la matriz que se da a continuación:\end{tcolorbox}
\end{minipage} \newline
\noindent\begin{minipage}{\textwidth} 
\begin{tcolorbox}[colback = red!20!white,title=Versión Ecuaciones Implícitas]
Determina de entre los vectores columna de la matriz $A$, cuales de ellos están en espacio vectorial definido en forma implícita por las siguientes ecuaciones:
\[ 38 x_{0} + 5 x_{1} + 28 x_{2} = 0 \]
\[ 29 x_{0} + 10 x_{1} + 12 x_{2} = 0 \]
\[ 17 x_{0} + 30 x_{1} + x_{2} = 0 \]
\[ 6 x_{0} + 17 x_{1} + 9 x_{2} = 0 \]
Siendo $A$ la matriz que se da a continuación:
\end{tcolorbox}
\end{minipage}
\[ A = \left[\begin{array}{rrrrrrrrrr}
10 & 3 & 7 & 27 & 27 & 2 & 24 & 13 & 21 & 16 \\
36 & 0 & 31 & 35 & 7 & 40 & 15 & 14 & 8 & 33 \\
21 & 16 & 13 & 19 & 28 & 37 & 2 & 15 & 10 & 9
\end{array}\right] \in {\bf M}_{3\times 10}({\mathbb Z}_{41})\]
\end{ejer}

{\it Soluci\'on:}
% Escribe tu soluci\'on para el Ejercicio 60

\begin{sageblock}
matrix(Zmod(41),[[38,5,28],
[29,10,12],
[17,30,1],
[6,17,9]])
matrix(Zmod(41),[[10,3,7,27,27,2,24,13,21,16],
[36,0,31,35,7,40,15,14,8,33],
[21,16,13,19,28,37,2,15,10,9]])
\end{sageblock}

% Fin del Ejercicio 60


\begin{ejer} Sea $K$ el cuerpo de 37 elementos.
\newline
\noindent\begin{minipage}{\textwidth}
\begin{tcolorbox}[colback = green!20!white,title=Versión Núcleo]
Determina de entre los vectores columna de la matriz $A$, cuales de ellos están en la imagen de la aplicación lineal $f:K^{3} \to K^{4}$ siendo  $$ M(f) = \left[\begin{array}{rrr}
21 & 3 & 27 \\
23 & 5 & 20 \\
8 & 14 & 31 \\
29 & 8 & 25
\end{array}\right] $$ y $A$ la matriz que se da a continuación:\end{tcolorbox}
\end{minipage} \newline
\noindent\begin{minipage}{\textwidth}
\begin{tcolorbox}[colback = blue!20!white,title=Versión Anulador]
Determina de entre los vectores columna de la matriz $A$, cuales de ellos están en el espacio generado por las columnas de la matriz $$ \left[\begin{array}{rrr}
21 & 3 & 27 \\
23 & 5 & 20 \\
8 & 14 & 31 \\
29 & 8 & 25
\end{array}\right] $$ siendo $A$ la matriz que se da a continuación:\end{tcolorbox}
\end{minipage} \newline
\noindent\begin{minipage}{\textwidth} 
\begin{tcolorbox}[colback = red!20!white,title=Versión Ecuaciones Implícitas]
Para cada una de las columnas $B$ de la matriz $A$, determina qué sistemas de ecuaciones tienen solución, siendo $$ \left[\begin{array}{rrr}
21 & 3 & 27 \\
23 & 5 & 20 \\
8 & 14 & 31 \\
29 & 8 & 25
\end{array}\right] \left[\begin{array}{r}
x_{0} \\
x_{1} \\
x_{2}
\end{array}\right] = B$$ siendo $A$ la matriz que se da a continuación:
\end{tcolorbox}
\end{minipage}
\[ A = \left[\begin{array}{rrrrrrrrrr}
1 & 16 & 36 & 35 & 13 & 10 & 24 & 14 & 10 & 33 \\
28 & 30 & 6 & 18 & 18 & 35 & 16 & 30 & 11 & 3 \\
20 & 30 & 22 & 6 & 24 & 30 & 6 & 29 & 35 & 23 \\
3 & 33 & 30 & 11 & 1 & 34 & 10 & 2 & 17 & 8
\end{array}\right] \in {\bf M}_{4\times 10}({\mathbb Z}_{37})\]
\end{ejer}

{\it Soluci\'on:}
% Escribe tu soluci\'on para el Ejercicio 61

\begin{sageblock}
matrix(Zmod(37),[[21,3,27],
[23,5,20],
[8,14,31],
[29,8,25]])
matrix(Zmod(37),[[1,16,36,35,13,10,24,14,10,33],
[28,30,6,18,18,35,16,30,11,3],
[20,30,22,6,24,30,6,29,35,23],
[3,33,30,11,1,34,10,2,17,8]])
\end{sageblock}

% Fin del Ejercicio 61


\begin{ejer} Sea $K$ el cuerpo de 43 elementos.
\newline
\noindent\begin{minipage}{\textwidth}
\begin{tcolorbox}[colback = green!20!white,title=Versión Núcleo]
Determina de entre los vectores columna de la matriz $A$, cuales de ellos están en la imagen de la aplicación lineal $f:K^{2} \to K^{3}$ siendo  $$ M(f) = \left[\begin{array}{rr}
25 & 31 \\
19 & 27 \\
32 & 7
\end{array}\right] $$ y $A$ la matriz que se da a continuación:\end{tcolorbox}
\end{minipage} \newline
\noindent\begin{minipage}{\textwidth}
\begin{tcolorbox}[colback = blue!20!white,title=Versión Anulador]
Determina de entre los vectores columna de la matriz $A$, cuales de ellos están en el espacio generado por las columnas de la matriz $$ \left[\begin{array}{rr}
25 & 31 \\
19 & 27 \\
32 & 7
\end{array}\right] $$ siendo $A$ la matriz que se da a continuación:\end{tcolorbox}
\end{minipage} \newline
\noindent\begin{minipage}{\textwidth} 
\begin{tcolorbox}[colback = red!20!white,title=Versión Ecuaciones Implícitas]
Para cada una de las columnas $B$ de la matriz $A$, determina qué sistemas de ecuaciones tienen solución, siendo $$ \left[\begin{array}{rr}
25 & 31 \\
19 & 27 \\
32 & 7
\end{array}\right] \left[\begin{array}{r}
x_{0} \\
x_{1}
\end{array}\right] = B$$ siendo $A$ la matriz que se da a continuación:
\end{tcolorbox}
\end{minipage}
\[ A = \left[\begin{array}{rrrrrrrrrr}
2 & 19 & 31 & 36 & 37 & 35 & 26 & 14 & 4 & 28 \\
17 & 32 & 27 & 4 & 35 & 17 & 6 & 17 & 0 & 23 \\
6 & 18 & 7 & 12 & 25 & 16 & 35 & 5 & 16 & 41
\end{array}\right] \in {\bf M}_{3\times 10}({\mathbb Z}_{43})\]
\end{ejer}

{\it Soluci\'on:}
% Escribe tu soluci\'on para el Ejercicio 62

\begin{sageblock}
matrix(Zmod(43),[[25,31],
[19,27],
[32,7]])
matrix(Zmod(43),[[2,19,31,36,37,35,26,14,4,28],
[17,32,27,4,35,17,6,17,0,23],
[6,18,7,12,25,16,35,5,16,41]])
\end{sageblock}

% Fin del Ejercicio 62


\begin{ejer} Sea $K$ el cuerpo de 29 elementos.
\newline
\noindent\begin{minipage}{\textwidth}
\begin{tcolorbox}[colback = green!20!white,title=Versión Núcleo]
Determina de entre los vectores columna de la matriz $A$, cuales de ellos están en la imagen de la aplicación lineal $f:K^{3} \to K^{5}$ siendo  $$ M(f) = \left[\begin{array}{rrr}
0 & 11 & 15 \\
28 & 11 & 4 \\
12 & 23 & 21 \\
18 & 6 & 19 \\
15 & 4 & 7
\end{array}\right] $$ y $A$ la matriz que se da a continuación:\end{tcolorbox}
\end{minipage} \newline
\noindent\begin{minipage}{\textwidth}
\begin{tcolorbox}[colback = blue!20!white,title=Versión Anulador]
Determina de entre los vectores columna de la matriz $A$, cuales de ellos están en el espacio generado por las columnas de la matriz $$ \left[\begin{array}{rrr}
0 & 11 & 15 \\
28 & 11 & 4 \\
12 & 23 & 21 \\
18 & 6 & 19 \\
15 & 4 & 7
\end{array}\right] $$ siendo $A$ la matriz que se da a continuación:\end{tcolorbox}
\end{minipage} \newline
\noindent\begin{minipage}{\textwidth} 
\begin{tcolorbox}[colback = red!20!white,title=Versión Ecuaciones Implícitas]
Para cada una de las columnas $B$ de la matriz $A$, determina qué sistemas de ecuaciones tienen solución, siendo $$ \left[\begin{array}{rrr}
0 & 11 & 15 \\
28 & 11 & 4 \\
12 & 23 & 21 \\
18 & 6 & 19 \\
15 & 4 & 7
\end{array}\right] \left[\begin{array}{r}
x_{0} \\
x_{1} \\
x_{2}
\end{array}\right] = B$$ siendo $A$ la matriz que se da a continuación:
\end{tcolorbox}
\end{minipage}
\[ A = \left[\begin{array}{rrrrrrrrrr}
13 & 23 & 21 & 13 & 4 & 1 & 17 & 13 & 7 & 21 \\
1 & 2 & 20 & 21 & 9 & 2 & 15 & 19 & 8 & 27 \\
13 & 1 & 19 & 13 & 9 & 18 & 20 & 19 & 9 & 22 \\
28 & 22 & 11 & 13 & 15 & 15 & 11 & 2 & 17 & 1 \\
16 & 6 & 20 & 11 & 24 & 15 & 23 & 9 & 3 & 2
\end{array}\right] \in {\bf M}_{5\times 10}({\mathbb Z}_{29})\]
\end{ejer}

{\it Soluci\'on:}
% Escribe tu soluci\'on para el Ejercicio 63

\begin{sageblock}
matrix(Zmod(29),[[0,11,15],
[28,11,4],
[12,23,21],
[18,6,19],
[15,4,7]])
matrix(Zmod(29),[[13,23,21,13,4,1,17,13,7,21],
[1,2,20,21,9,2,15,19,8,27],
[13,1,19,13,9,18,20,19,9,22],
[28,22,11,13,15,15,11,2,17,1],
[16,6,20,11,24,15,23,9,3,2]])
\end{sageblock}

% Fin del Ejercicio 63


\begin{ejer} Sea $K$ el cuerpo de 17 elementos.
\newline
\noindent\begin{minipage}{\textwidth}
\begin{tcolorbox}[colback = green!20!white,title=Versión Núcleo]
Determina de entre los vectores columna de la matriz $A$, cuales de ellos están en la imagen de la aplicación lineal $f:K^{3} \to K^{4}$ siendo  $$ M(f) = \left[\begin{array}{rrr}
9 & 6 & 15 \\
2 & 9 & 5 \\
9 & 3 & 4 \\
4 & 4 & 4
\end{array}\right] $$ y $A$ la matriz que se da a continuación:\end{tcolorbox}
\end{minipage} \newline
\noindent\begin{minipage}{\textwidth}
\begin{tcolorbox}[colback = blue!20!white,title=Versión Anulador]
Determina de entre los vectores columna de la matriz $A$, cuales de ellos están en el espacio generado por las columnas de la matriz $$ \left[\begin{array}{rrr}
9 & 6 & 15 \\
2 & 9 & 5 \\
9 & 3 & 4 \\
4 & 4 & 4
\end{array}\right] $$ siendo $A$ la matriz que se da a continuación:\end{tcolorbox}
\end{minipage} \newline
\noindent\begin{minipage}{\textwidth} 
\begin{tcolorbox}[colback = red!20!white,title=Versión Ecuaciones Implícitas]
Para cada una de las columnas $B$ de la matriz $A$, determina qué sistemas de ecuaciones tienen solución, siendo $$ \left[\begin{array}{rrr}
9 & 6 & 15 \\
2 & 9 & 5 \\
9 & 3 & 4 \\
4 & 4 & 4
\end{array}\right] \left[\begin{array}{r}
x_{0} \\
x_{1} \\
x_{2}
\end{array}\right] = B$$ siendo $A$ la matriz que se da a continuación:
\end{tcolorbox}
\end{minipage}
\[ A = \left[\begin{array}{rrrrrrrrrr}
4 & 1 & 2 & 14 & 12 & 3 & 11 & 8 & 10 & 10 \\
16 & 8 & 12 & 2 & 11 & 16 & 0 & 9 & 9 & 11 \\
4 & 16 & 13 & 10 & 8 & 14 & 2 & 0 & 15 & 2 \\
15 & 1 & 8 & 8 & 9 & 4 & 15 & 9 & 10 & 3
\end{array}\right] \in {\bf M}_{4\times 10}({\mathbb Z}_{17})\]
\end{ejer}

{\it Soluci\'on:}
% Escribe tu soluci\'on para el Ejercicio 64

\begin{sageblock}
matrix(Zmod(17),[[9,6,15],
[2,9,5],
[9,3,4],
[4,4,4]])
matrix(Zmod(17),[[4,1,2,14,12,3,11,8,10,10],
[16,8,12,2,11,16,0,9,9,11],
[4,16,13,10,8,14,2,0,15,2],
[15,1,8,8,9,4,15,9,10,3]])
\end{sageblock}

% Fin del Ejercicio 64


\begin{ejer} Sea $K$ el cuerpo de los n\'umeros reales.
\newline
\noindent\begin{minipage}{\textwidth}
\begin{tcolorbox}[colback = green!20!white,title=Versión Núcleo]
Determina de entre los vectores columna de la matriz $A$, cuales de ellos están en la imagen de la aplicación lineal $f:K^{3} \to K^{4}$ siendo  $$ M(f) = \left[\begin{array}{rrr}
1 & -4 & 2 \\
1 & -3 & 1 \\
3 & -7 & 1 \\
-2 & 5 & -1
\end{array}\right] $$ y $A$ la matriz que se da a continuación:\end{tcolorbox}
\end{minipage} \newline
\noindent\begin{minipage}{\textwidth}
\begin{tcolorbox}[colback = blue!20!white,title=Versión Anulador]
Determina de entre los vectores columna de la matriz $A$, cuales de ellos están en el espacio generado por las columnas de la matriz $$ \left[\begin{array}{rrr}
1 & -4 & 2 \\
1 & -3 & 1 \\
3 & -7 & 1 \\
-2 & 5 & -1
\end{array}\right] $$ siendo $A$ la matriz que se da a continuación:\end{tcolorbox}
\end{minipage} \newline
\noindent\begin{minipage}{\textwidth} 
\begin{tcolorbox}[colback = red!20!white,title=Versión Ecuaciones Implícitas]
Para cada una de las columnas $B$ de la matriz $A$, determina qué sistemas de ecuaciones tienen solución, siendo $$ \left[\begin{array}{rrr}
1 & -4 & 2 \\
1 & -3 & 1 \\
3 & -7 & 1 \\
-2 & 5 & -1
\end{array}\right] \left[\begin{array}{r}
x_{0} \\
x_{1} \\
x_{2}
\end{array}\right] = B$$ siendo $A$ la matriz que se da a continuación:
\end{tcolorbox}
\end{minipage}
\[ A = \left[\begin{array}{rrrrrrrrr}
\frac{11}{94} & 0 & \frac{3}{2} & 288 & \frac{8}{9} & 0 & 0 & -2 & -4 \\
-1 & 1 & -\frac{1}{2} & \frac{1}{20} & -\frac{1}{2} & -1 & 6 & 0 & 0 \\
-\frac{246}{47} & 5 & -\frac{5}{3} & \frac{1}{5} & 3 & -5 & 2 & 4 & -1 \\
\frac{293}{94} & -3 & \frac{2}{7} & -1 & -1 & 3 & 0 & -2 & -5
\end{array}\right] \in {\bf M}_{4\times 9}({\mathbb R})\]
\end{ejer}

{\it Soluci\'on:}
% Escribe tu soluci\'on para el Ejercicio 65

\begin{sageblock}
matrix(QQ,[[1,-4,2],
[1,-3,1],
[3,-7,1],
[-2,5,-1]])
matrix(QQ,[[11/94,0,3/2,288,8/9,0,0,-2,-4],
[-1,1,-1/2,1/20,-1/2,-1,6,0,0],
[-246/47,5,-5/3,1/5,3,-5,2,4,-1],
[293/94,-3,2/7,-1,-1,3,0,-2,-5]])
\end{sageblock}

% Fin del Ejercicio 65


\begin{ejer} Sea $K$ el cuerpo de 29 elementos.
\newline
\noindent\begin{minipage}{\textwidth}
\begin{tcolorbox}[colback = green!20!white,title=Versión Núcleo]
Determina de entre los vectores columna de la matriz $A$, cuales de ellos están en la imagen de la aplicación lineal $f:K^{5} \to K^{2}$ siendo  $$ M(f) = \left[\begin{array}{rrrrr}
10 & 7 & 5 & 14 & 16 \\
3 & 5 & 16 & 10 & 28
\end{array}\right] $$ y $A$ la matriz que se da a continuación:\end{tcolorbox}
\end{minipage} \newline
\noindent\begin{minipage}{\textwidth}
\begin{tcolorbox}[colback = blue!20!white,title=Versión Anulador]
Determina de entre los vectores columna de la matriz $A$, cuales de ellos están en el espacio generado por las columnas de la matriz $$ \left[\begin{array}{rrrrr}
10 & 7 & 5 & 14 & 16 \\
3 & 5 & 16 & 10 & 28
\end{array}\right] $$ siendo $A$ la matriz que se da a continuación:\end{tcolorbox}
\end{minipage} \newline
\noindent\begin{minipage}{\textwidth} 
\begin{tcolorbox}[colback = red!20!white,title=Versión Ecuaciones Implícitas]
Para cada una de las columnas $B$ de la matriz $A$, determina qué sistemas de ecuaciones tienen solución, siendo $$ \left[\begin{array}{rrrrr}
10 & 7 & 5 & 14 & 16 \\
3 & 5 & 16 & 10 & 28
\end{array}\right] \left[\begin{array}{r}
x_{0} \\
x_{1} \\
x_{2} \\
x_{3} \\
x_{4}
\end{array}\right] = B$$ siendo $A$ la matriz que se da a continuación:
\end{tcolorbox}
\end{minipage}
\[ A = \left[\begin{array}{rrrrrrrrr}
17 & 5 & 14 & 6 & 22 & 21 & 7 & 9 & 26 \\
8 & 4 & 1 & 20 & 24 & 7 & 5 & 23 & 14
\end{array}\right] \in {\bf M}_{2\times 9}({\mathbb Z}_{29})\]
\end{ejer}

{\it Soluci\'on:}
% Escribe tu soluci\'on para el Ejercicio 66

\begin{sageblock}
matrix(Zmod(29),[[10,7,5,14,16],
[3,5,16,10,28]])
matrix(Zmod(29),[[17,5,14,6,22,21,7,9,26],
[8,4,1,20,24,7,5,23,14]])
\end{sageblock}

% Fin del Ejercicio 66


\begin{ejer} Sea $K$ el cuerpo de 13 elementos.
\newline
\noindent\begin{minipage}{\textwidth}
\begin{tcolorbox}[colback = green!20!white,title=Versión Núcleo]
Determina de entre los vectores columna de la matriz $A$, cuales de ellos están en la imagen de la aplicación lineal $f:K^{2} \to K^{3}$ siendo  $$ M(f) = \left[\begin{array}{rr}
4 & 11 \\
10 & 8 \\
7 & 3
\end{array}\right] $$ y $A$ la matriz que se da a continuación:\end{tcolorbox}
\end{minipage} \newline
\noindent\begin{minipage}{\textwidth}
\begin{tcolorbox}[colback = blue!20!white,title=Versión Anulador]
Determina de entre los vectores columna de la matriz $A$, cuales de ellos están en el espacio generado por las columnas de la matriz $$ \left[\begin{array}{rr}
4 & 11 \\
10 & 8 \\
7 & 3
\end{array}\right] $$ siendo $A$ la matriz que se da a continuación:\end{tcolorbox}
\end{minipage} \newline
\noindent\begin{minipage}{\textwidth} 
\begin{tcolorbox}[colback = red!20!white,title=Versión Ecuaciones Implícitas]
Para cada una de las columnas $B$ de la matriz $A$, determina qué sistemas de ecuaciones tienen solución, siendo $$ \left[\begin{array}{rr}
4 & 11 \\
10 & 8 \\
7 & 3
\end{array}\right] \left[\begin{array}{r}
x_{0} \\
x_{1}
\end{array}\right] = B$$ siendo $A$ la matriz que se da a continuación:
\end{tcolorbox}
\end{minipage}
\[ A = \left[\begin{array}{rrrrrrrrrr}
12 & 9 & 1 & 11 & 8 & 5 & 7 & 10 & 10 & 2 \\
4 & 3 & 9 & 8 & 7 & 9 & 3 & 12 & 8 & 11 \\
8 & 6 & 5 & 3 & 1 & 6 & 12 & 11 & 3 & 3
\end{array}\right] \in {\bf M}_{3\times 10}({\mathbb Z}_{13})\]
\end{ejer}

{\it Soluci\'on:}
% Escribe tu soluci\'on para el Ejercicio 67

\begin{sageblock}
matrix(Zmod(13),[[4,11],
[10,8],
[7,3]])
matrix(Zmod(13),[[12,9,1,11,8,5,7,10,10,2],
[4,3,9,8,7,9,3,12,8,11],
[8,6,5,3,1,6,12,11,3,3]])
\end{sageblock}

% Fin del Ejercicio 67


\begin{ejer} Sea $K$ el cuerpo de 19 elementos.
\newline
\noindent\begin{minipage}{\textwidth}
\begin{tcolorbox}[colback = green!20!white,title=Versión Núcleo]
Determina de entre los vectores columna de la matriz $A$, cuales de ellos están en la imagen de la aplicación lineal $f:K^{3} \to K^{4}$ siendo  $$ M(f) = \left[\begin{array}{rrr}
10 & 4 & 15 \\
15 & 15 & 9 \\
4 & 2 & 13 \\
14 & 18 & 10
\end{array}\right] $$ y $A$ la matriz que se da a continuación:\end{tcolorbox}
\end{minipage} \newline
\noindent\begin{minipage}{\textwidth}
\begin{tcolorbox}[colback = blue!20!white,title=Versión Anulador]
Determina de entre los vectores columna de la matriz $A$, cuales de ellos están en el espacio generado por las columnas de la matriz $$ \left[\begin{array}{rrr}
10 & 4 & 15 \\
15 & 15 & 9 \\
4 & 2 & 13 \\
14 & 18 & 10
\end{array}\right] $$ siendo $A$ la matriz que se da a continuación:\end{tcolorbox}
\end{minipage} \newline
\noindent\begin{minipage}{\textwidth} 
\begin{tcolorbox}[colback = red!20!white,title=Versión Ecuaciones Implícitas]
Para cada una de las columnas $B$ de la matriz $A$, determina qué sistemas de ecuaciones tienen solución, siendo $$ \left[\begin{array}{rrr}
10 & 4 & 15 \\
15 & 15 & 9 \\
4 & 2 & 13 \\
14 & 18 & 10
\end{array}\right] \left[\begin{array}{r}
x_{0} \\
x_{1} \\
x_{2}
\end{array}\right] = B$$ siendo $A$ la matriz que se da a continuación:
\end{tcolorbox}
\end{minipage}
\[ A = \left[\begin{array}{rrrrrrrrrr}
16 & 12 & 13 & 8 & 18 & 18 & 15 & 7 & 7 & 12 \\
0 & 6 & 6 & 3 & 6 & 5 & 18 & 16 & 1 & 5 \\
18 & 3 & 9 & 18 & 16 & 12 & 0 & 8 & 18 & 0 \\
2 & 18 & 3 & 14 & 14 & 12 & 0 & 7 & 6 & 1
\end{array}\right] \in {\bf M}_{4\times 10}({\mathbb Z}_{19})\]
\end{ejer}

{\it Soluci\'on:}
% Escribe tu soluci\'on para el Ejercicio 68

\begin{sageblock}
matrix(Zmod(19),[[10,4,15],
[15,15,9],
[4,2,13],
[14,18,10]])
matrix(Zmod(19),[[16,12,13,8,18,18,15,7,7,12],
[0,6,6,3,6,5,18,16,1,5],
[18,3,9,18,16,12,0,8,18,0],
[2,18,3,14,14,12,0,7,6,1]])
\end{sageblock}

% Fin del Ejercicio 68


\begin{ejer} Sea $K$ el cuerpo de 47 elementos.
\newline
\noindent\begin{minipage}{\textwidth}
\begin{tcolorbox}[colback = green!20!white,title=Versión Núcleo]
Determina de entre los vectores columna de la matriz $A$, cuales de ellos están en la imagen de la aplicación lineal $f:K^{4} \to K^{3}$ siendo  $$ M(f) = \left[\begin{array}{rrrr}
20 & 11 & 22 & 21 \\
34 & 14 & 28 & 31 \\
6 & 31 & 43 & 0
\end{array}\right] $$ y $A$ la matriz que se da a continuación:\end{tcolorbox}
\end{minipage} \newline
\noindent\begin{minipage}{\textwidth}
\begin{tcolorbox}[colback = blue!20!white,title=Versión Anulador]
Determina de entre los vectores columna de la matriz $A$, cuales de ellos están en el espacio generado por las columnas de la matriz $$ \left[\begin{array}{rrrr}
20 & 11 & 22 & 21 \\
34 & 14 & 28 & 31 \\
6 & 31 & 43 & 0
\end{array}\right] $$ siendo $A$ la matriz que se da a continuación:\end{tcolorbox}
\end{minipage} \newline
\noindent\begin{minipage}{\textwidth} 
\begin{tcolorbox}[colback = red!20!white,title=Versión Ecuaciones Implícitas]
Para cada una de las columnas $B$ de la matriz $A$, determina qué sistemas de ecuaciones tienen solución, siendo $$ \left[\begin{array}{rrrr}
20 & 11 & 22 & 21 \\
34 & 14 & 28 & 31 \\
6 & 31 & 43 & 0
\end{array}\right] \left[\begin{array}{r}
x_{0} \\
x_{1} \\
x_{2} \\
x_{3}
\end{array}\right] = B$$ siendo $A$ la matriz que se da a continuación:
\end{tcolorbox}
\end{minipage}
\[ A = \left[\begin{array}{rrrrrrrrrr}
44 & 40 & 37 & 11 & 39 & 16 & 16 & 45 & 39 & 19 \\
25 & 28 & 30 & 40 & 24 & 46 & 46 & 37 & 24 & 40 \\
13 & 16 & 19 & 16 & 45 & 14 & 6 & 39 & 25 & 30
\end{array}\right] \in {\bf M}_{3\times 10}({\mathbb Z}_{47})\]
\end{ejer}

{\it Soluci\'on:}
% Escribe tu soluci\'on para el Ejercicio 69

\begin{sageblock}
matrix(Zmod(47),[[20,11,22,21],
[34,14,28,31],
[6,31,43,0]])
matrix(Zmod(47),[[44,40,37,11,39,16,16,45,39,19],
[25,28,30,40,24,46,46,37,24,40],
[13,16,19,16,45,14,6,39,25,30]])
\end{sageblock}

% Fin del Ejercicio 69


\begin{ejer} Sea $K$ el cuerpo de 31 elementos.
\newline
\noindent\begin{minipage}{\textwidth}
\begin{tcolorbox}[colback = green!20!white,title=Versión Núcleo]
Determina de entre los vectores columna de la matriz $A$, cuales de ellos están en la imagen de la aplicación lineal $f:K^{3} \to K^{5}$ siendo  $$ M(f) = \left[\begin{array}{rrr}
19 & 2 & 4 \\
11 & 11 & 16 \\
30 & 30 & 7 \\
0 & 18 & 30 \\
21 & 12 & 24
\end{array}\right] $$ y $A$ la matriz que se da a continuación:\end{tcolorbox}
\end{minipage} \newline
\noindent\begin{minipage}{\textwidth}
\begin{tcolorbox}[colback = blue!20!white,title=Versión Anulador]
Determina de entre los vectores columna de la matriz $A$, cuales de ellos están en el espacio generado por las columnas de la matriz $$ \left[\begin{array}{rrr}
19 & 2 & 4 \\
11 & 11 & 16 \\
30 & 30 & 7 \\
0 & 18 & 30 \\
21 & 12 & 24
\end{array}\right] $$ siendo $A$ la matriz que se da a continuación:\end{tcolorbox}
\end{minipage} \newline
\noindent\begin{minipage}{\textwidth} 
\begin{tcolorbox}[colback = red!20!white,title=Versión Ecuaciones Implícitas]
Para cada una de las columnas $B$ de la matriz $A$, determina qué sistemas de ecuaciones tienen solución, siendo $$ \left[\begin{array}{rrr}
19 & 2 & 4 \\
11 & 11 & 16 \\
30 & 30 & 7 \\
0 & 18 & 30 \\
21 & 12 & 24
\end{array}\right] \left[\begin{array}{r}
x_{0} \\
x_{1} \\
x_{2}
\end{array}\right] = B$$ siendo $A$ la matriz que se da a continuación:
\end{tcolorbox}
\end{minipage}
\[ A = \left[\begin{array}{rrrrrrrrrr}
27 & 21 & 3 & 15 & 21 & 2 & 7 & 24 & 18 & 21 \\
6 & 6 & 29 & 15 & 11 & 7 & 6 & 20 & 30 & 19 \\
22 & 22 & 3 & 3 & 12 & 5 & 23 & 1 & 30 & 21 \\
23 & 2 & 24 & 16 & 0 & 14 & 19 & 11 & 6 & 24 \\
7 & 2 & 18 & 6 & 8 & 12 & 30 & 20 & 29 & 2
\end{array}\right] \in {\bf M}_{5\times 10}({\mathbb Z}_{31})\]
\end{ejer}

{\it Soluci\'on:}
% Escribe tu soluci\'on para el Ejercicio 70

\begin{sageblock}
matrix(Zmod(31),[[19,2,4],
[11,11,16],
[30,30,7],
[0,18,30],
[21,12,24]])
matrix(Zmod(31),[[27,21,3,15,21,2,7,24,18,21],
[6,6,29,15,11,7,6,20,30,19],
[22,22,3,3,12,5,23,1,30,21],
[23,2,24,16,0,14,19,11,6,24],
[7,2,18,6,8,12,30,20,29,2]])
\end{sageblock}

% Fin del Ejercicio 70


\begin{ejer} Sea $K$ el cuerpo de los n\'umeros reales.
\newline
\noindent\begin{minipage}{\textwidth}
\begin{tcolorbox}[colback = green!20!white,title=Versión Núcleo]
Determina de entre los vectores columna de la matriz $A$, cuales de ellos están en la imagen de la aplicación lineal $f:K^{5} \to K^{3}$ siendo  $$ M(f) = \left[\begin{array}{rrrrr}
1 & 2 & 1 & -7 & 0 \\
-1 & -1 & 0 & 4 & 1 \\
0 & -1 & -1 & 3 & -1
\end{array}\right] $$ y $A$ la matriz que se da a continuación:\end{tcolorbox}
\end{minipage} \newline
\noindent\begin{minipage}{\textwidth}
\begin{tcolorbox}[colback = blue!20!white,title=Versión Anulador]
Determina de entre los vectores columna de la matriz $A$, cuales de ellos están en el espacio generado por las columnas de la matriz $$ \left[\begin{array}{rrrrr}
1 & 2 & 1 & -7 & 0 \\
-1 & -1 & 0 & 4 & 1 \\
0 & -1 & -1 & 3 & -1
\end{array}\right] $$ siendo $A$ la matriz que se da a continuación:\end{tcolorbox}
\end{minipage} \newline
\noindent\begin{minipage}{\textwidth} 
\begin{tcolorbox}[colback = red!20!white,title=Versión Ecuaciones Implícitas]
Para cada una de las columnas $B$ de la matriz $A$, determina qué sistemas de ecuaciones tienen solución, siendo $$ \left[\begin{array}{rrrrr}
1 & 2 & 1 & -7 & 0 \\
-1 & -1 & 0 & 4 & 1 \\
0 & -1 & -1 & 3 & -1
\end{array}\right] \left[\begin{array}{r}
x_{0} \\
x_{1} \\
x_{2} \\
x_{3} \\
x_{4}
\end{array}\right] = B$$ siendo $A$ la matriz que se da a continuación:
\end{tcolorbox}
\end{minipage}
\[ A = \left[\begin{array}{rrrrrrrrrr}
-\frac{1}{2} & -\frac{1}{2} & 0 & 1 & 1 & -\frac{13}{3} & 0 & -3 & 2 & \frac{1}{13} \\
-2 & -\frac{7}{3} & -\frac{21}{32} & \frac{1}{2} & 0 & -4 & 2 & -\frac{1}{20} & -4 & 0 \\
0 & \frac{17}{6} & 5 & -\frac{3}{2} & -1 & 14 & -\frac{335}{3} & \frac{48}{11} & 2 & -\frac{1}{13}
\end{array}\right] \in {\bf M}_{3\times 10}({\mathbb R})\]
\end{ejer}

{\it Soluci\'on:}
% Escribe tu soluci\'on para el Ejercicio 71

\begin{sageblock}
matrix(QQ,[[1,2,1,-7,0],
[-1,-1,0,4,1],
[0,-1,-1,3,-1]])
matrix(QQ,[[-1/2,-1/2,0,1,1,-13/3,0,-3,2,1/13],
[-2,-7/3,-21/32,1/2,0,-4,2,-1/20,-4,0],
[0,17/6,5,-3/2,-1,14,-335/3,48/11,2,-1/13]])
\end{sageblock}

% Fin del Ejercicio 71


\begin{ejer} Sea $K$ el cuerpo de 29 elementos.
\newline
\noindent\begin{minipage}{\textwidth}
\begin{tcolorbox}[colback = green!20!white,title=Versión Núcleo]
Determina de entre los vectores columna de la matriz $A$, cuales de ellos están en la imagen de la aplicación lineal $f:K^{4} \to K^{2}$ siendo  $$ M(f) = \left[\begin{array}{rrrr}
10 & 1 & 0 & 10 \\
19 & 28 & 0 & 19
\end{array}\right] $$ y $A$ la matriz que se da a continuación:\end{tcolorbox}
\end{minipage} \newline
\noindent\begin{minipage}{\textwidth}
\begin{tcolorbox}[colback = blue!20!white,title=Versión Anulador]
Determina de entre los vectores columna de la matriz $A$, cuales de ellos están en el espacio generado por las columnas de la matriz $$ \left[\begin{array}{rrrr}
10 & 1 & 0 & 10 \\
19 & 28 & 0 & 19
\end{array}\right] $$ siendo $A$ la matriz que se da a continuación:\end{tcolorbox}
\end{minipage} \newline
\noindent\begin{minipage}{\textwidth} 
\begin{tcolorbox}[colback = red!20!white,title=Versión Ecuaciones Implícitas]
Para cada una de las columnas $B$ de la matriz $A$, determina qué sistemas de ecuaciones tienen solución, siendo $$ \left[\begin{array}{rrrr}
10 & 1 & 0 & 10 \\
19 & 28 & 0 & 19
\end{array}\right] \left[\begin{array}{r}
x_{0} \\
x_{1} \\
x_{2} \\
x_{3}
\end{array}\right] = B$$ siendo $A$ la matriz que se da a continuación:
\end{tcolorbox}
\end{minipage}
\[ A = \left[\begin{array}{rrrrrrrrrr}
14 & 13 & 12 & 15 & 14 & 13 & 25 & 23 & 26 & 4 \\
28 & 15 & 6 & 14 & 15 & 16 & 4 & 19 & 4 & 25
\end{array}\right] \in {\bf M}_{2\times 10}({\mathbb Z}_{29})\]
\end{ejer}

{\it Soluci\'on:}
% Escribe tu soluci\'on para el Ejercicio 72

\begin{sageblock}
matrix(Zmod(29),[[10,1,0,10],
[19,28,0,19]])
matrix(Zmod(29),[[14,13,12,15,14,13,25,23,26,4],
[28,15,6,14,15,16,4,19,4,25]])
\end{sageblock}

% Fin del Ejercicio 72


\begin{ejer} Sea $K$ el cuerpo de los n\'umeros reales.
\newline
\noindent\begin{minipage}{\textwidth}
\begin{tcolorbox}[colback = green!20!white,title=Versión Núcleo]
Determina de entre los vectores columna de la matriz $A$, cuales de ellos están en la imagen de la aplicación lineal $f:K^{3} \to K^{5}$ siendo  $$ M(f) = \left[\begin{array}{rrr}
6 & -7 & -7 \\
-5 & 6 & 6 \\
1 & 3 & 3 \\
4 & -1 & -1 \\
5 & -1 & -1
\end{array}\right] $$ y $A$ la matriz que se da a continuación:\end{tcolorbox}
\end{minipage} \newline
\noindent\begin{minipage}{\textwidth}
\begin{tcolorbox}[colback = blue!20!white,title=Versión Anulador]
Determina de entre los vectores columna de la matriz $A$, cuales de ellos están en el espacio generado por las columnas de la matriz $$ \left[\begin{array}{rrr}
6 & -7 & -7 \\
-5 & 6 & 6 \\
1 & 3 & 3 \\
4 & -1 & -1 \\
5 & -1 & -1
\end{array}\right] $$ siendo $A$ la matriz que se da a continuación:\end{tcolorbox}
\end{minipage} \newline
\noindent\begin{minipage}{\textwidth} 
\begin{tcolorbox}[colback = red!20!white,title=Versión Ecuaciones Implícitas]
Para cada una de las columnas $B$ de la matriz $A$, determina qué sistemas de ecuaciones tienen solución, siendo $$ \left[\begin{array}{rrr}
6 & -7 & -7 \\
-5 & 6 & 6 \\
1 & 3 & 3 \\
4 & -1 & -1 \\
5 & -1 & -1
\end{array}\right] \left[\begin{array}{r}
x_{0} \\
x_{1} \\
x_{2}
\end{array}\right] = B$$ siendo $A$ la matriz que se da a continuación:
\end{tcolorbox}
\end{minipage}
\[ A = \left[\begin{array}{rrrrrrrrrr}
\frac{3}{2} & 0 & 0 & 23 & -3 & -1 & -\frac{1}{3} & 4 & -6 & 1 \\
2 & -2 & \frac{8}{3} & \frac{2}{3} & 1 & -2 & -2 & 0 & 0 & 1 \\
\frac{163}{2} & -50 & \frac{200}{3} & 0 & -\frac{1}{4} & -\frac{1}{2} & -1 & 3 & -126 & 46 \\
\frac{145}{2} & -44 & \frac{176}{3} & -\frac{11}{2} & \frac{2}{3} & \frac{1}{3} & -1 & -3 & -114 & 41 \\
\frac{191}{2} & -58 & \frac{232}{3} & 2 & -19 & -5 & \frac{1}{3} & -10 & -150 & 54
\end{array}\right] \in {\bf M}_{5\times 10}({\mathbb R})\]
\end{ejer}

{\it Soluci\'on:}
% Escribe tu soluci\'on para el Ejercicio 73

\begin{sageblock}
matrix(QQ,[[6,-7,-7],
[-5,6,6],
[1,3,3],
[4,-1,-1],
[5,-1,-1]])
matrix(QQ,[[3/2,0,0,23,-3,-1,-1/3,4,-6,1],
[2,-2,8/3,2/3,1,-2,-2,0,0,1],
[163/2,-50,200/3,0,-1/4,-1/2,-1,3,-126,46],
[145/2,-44,176/3,-11/2,2/3,1/3,-1,-3,-114,41],
[191/2,-58,232/3,2,-19,-5,1/3,-10,-150,54]])
\end{sageblock}

% Fin del Ejercicio 73


\begin{ejer} Sea $K$ el cuerpo de 41 elementos.
\newline
\noindent\begin{minipage}{\textwidth}
\begin{tcolorbox}[colback = green!20!white,title=Versión Núcleo]
Determina de entre los vectores columna de la matriz $A$, cuales de ellos están en la imagen de la aplicación lineal $f:K^{5} \to K^{2}$ siendo  $$ M(f) = \left[\begin{array}{rrrrr}
40 & 21 & 29 & 34 & 21 \\
3 & 19 & 36 & 21 & 19
\end{array}\right] $$ y $A$ la matriz que se da a continuación:\end{tcolorbox}
\end{minipage} \newline
\noindent\begin{minipage}{\textwidth}
\begin{tcolorbox}[colback = blue!20!white,title=Versión Anulador]
Determina de entre los vectores columna de la matriz $A$, cuales de ellos están en el espacio generado por las columnas de la matriz $$ \left[\begin{array}{rrrrr}
40 & 21 & 29 & 34 & 21 \\
3 & 19 & 36 & 21 & 19
\end{array}\right] $$ siendo $A$ la matriz que se da a continuación:\end{tcolorbox}
\end{minipage} \newline
\noindent\begin{minipage}{\textwidth} 
\begin{tcolorbox}[colback = red!20!white,title=Versión Ecuaciones Implícitas]
Para cada una de las columnas $B$ de la matriz $A$, determina qué sistemas de ecuaciones tienen solución, siendo $$ \left[\begin{array}{rrrrr}
40 & 21 & 29 & 34 & 21 \\
3 & 19 & 36 & 21 & 19
\end{array}\right] \left[\begin{array}{r}
x_{0} \\
x_{1} \\
x_{2} \\
x_{3} \\
x_{4}
\end{array}\right] = B$$ siendo $A$ la matriz que se da a continuación:
\end{tcolorbox}
\end{minipage}
\[ A = \left[\begin{array}{rrrrrrrrrr}
35 & 37 & 18 & 39 & 7 & 25 & 13 & 40 & 19 & 17 \\
18 & 6 & 28 & 16 & 33 & 7 & 10 & 34 & 25 & 31
\end{array}\right] \in {\bf M}_{2\times 10}({\mathbb Z}_{41})\]
\end{ejer}

{\it Soluci\'on:}
% Escribe tu soluci\'on para el Ejercicio 74

\begin{sageblock}
matrix(Zmod(41),[[40,21,29,34,21],
[3,19,36,21,19]])
matrix(Zmod(41),[[35,37,18,39,7,25,13,40,19,17],
[18,6,28,16,33,7,10,34,25,31]])
\end{sageblock}

% Fin del Ejercicio 74


\begin{ejer} Sea $K$ el cuerpo de los n\'umeros reales.
\newline
\noindent\begin{minipage}{\textwidth}
\begin{tcolorbox}[colback = green!20!white,title=Versión Núcleo]
Determina de entre los vectores columna de la matriz $A$, cuales de ellos están en la imagen de la aplicación lineal $f:K^{2} \to K^{5}$ siendo  $$ M(f) = \left[\begin{array}{rr}
7 & -7 \\
-3 & 3 \\
-3 & 3 \\
-2 & 2 \\
5 & -5
\end{array}\right] $$ y $A$ la matriz que se da a continuación:\end{tcolorbox}
\end{minipage} \newline
\noindent\begin{minipage}{\textwidth}
\begin{tcolorbox}[colback = blue!20!white,title=Versión Anulador]
Determina de entre los vectores columna de la matriz $A$, cuales de ellos están en el espacio generado por las columnas de la matriz $$ \left[\begin{array}{rr}
7 & -7 \\
-3 & 3 \\
-3 & 3 \\
-2 & 2 \\
5 & -5
\end{array}\right] $$ siendo $A$ la matriz que se da a continuación:\end{tcolorbox}
\end{minipage} \newline
\noindent\begin{minipage}{\textwidth} 
\begin{tcolorbox}[colback = red!20!white,title=Versión Ecuaciones Implícitas]
Para cada una de las columnas $B$ de la matriz $A$, determina qué sistemas de ecuaciones tienen solución, siendo $$ \left[\begin{array}{rr}
7 & -7 \\
-3 & 3 \\
-3 & 3 \\
-2 & 2 \\
5 & -5
\end{array}\right] \left[\begin{array}{r}
x_{0} \\
x_{1}
\end{array}\right] = B$$ siendo $A$ la matriz que se da a continuación:
\end{tcolorbox}
\end{minipage}
\[ A = \left[\begin{array}{rrrrrrrrr}
-\frac{4}{5} & \frac{1}{3} & 0 & -1 & 1 & 0 & -3 & -\frac{1}{35} & 0 \\
-\frac{1}{5} & -\frac{1}{7} & 0 & \frac{3}{7} & -\frac{3}{7} & 0 & -\frac{1}{4} & -2 & -3 \\
2 & -\frac{1}{7} & 1 & \frac{3}{7} & -\frac{3}{7} & 0 & -\frac{2}{67} & \frac{1}{24} & -\frac{1}{4} \\
-\frac{1}{5} & -\frac{2}{21} & -\frac{1}{4} & \frac{2}{7} & -\frac{2}{7} & 0 & 11 & \frac{4}{29} & 0 \\
0 & \frac{5}{21} & 4 & -\frac{5}{7} & \frac{5}{7} & 0 & -4 & 0 & 1
\end{array}\right] \in {\bf M}_{5\times 9}({\mathbb R})\]
\end{ejer}

{\it Soluci\'on:}
% Escribe tu soluci\'on para el Ejercicio 75

\begin{sageblock}
matrix(QQ,[[7,-7],
[-3,3],
[-3,3],
[-2,2],
[5,-5]])
matrix(QQ,[[-4/5,1/3,0,-1,1,0,-3,-1/35,0],
[-1/5,-1/7,0,3/7,-3/7,0,-1/4,-2,-3],
[2,-1/7,1,3/7,-3/7,0,-2/67,1/24,-1/4],
[-1/5,-2/21,-1/4,2/7,-2/7,0,11,4/29,0],
[0,5/21,4,-5/7,5/7,0,-4,0,1]])
\end{sageblock}

% Fin del Ejercicio 75


\begin{ejer} Sea $K$ el cuerpo de 7 elementos.
\newline
\noindent\begin{minipage}{\textwidth}
\begin{tcolorbox}[colback = green!20!white,title=Versión Núcleo]
Determina de entre los vectores columna de la matriz $A$, cuales de ellos están en la imagen de la aplicación lineal $f:K^{3} \to K^{5}$ siendo  $$ M(f) = \left[\begin{array}{rrr}
0 & 5 & 3 \\
6 & 3 & 6 \\
3 & 0 & 0 \\
2 & 5 & 3 \\
4 & 5 & 3
\end{array}\right] $$ y $A$ la matriz que se da a continuación:\end{tcolorbox}
\end{minipage} \newline
\noindent\begin{minipage}{\textwidth}
\begin{tcolorbox}[colback = blue!20!white,title=Versión Anulador]
Determina de entre los vectores columna de la matriz $A$, cuales de ellos están en el espacio generado por las columnas de la matriz $$ \left[\begin{array}{rrr}
0 & 5 & 3 \\
6 & 3 & 6 \\
3 & 0 & 0 \\
2 & 5 & 3 \\
4 & 5 & 3
\end{array}\right] $$ siendo $A$ la matriz que se da a continuación:\end{tcolorbox}
\end{minipage} \newline
\noindent\begin{minipage}{\textwidth} 
\begin{tcolorbox}[colback = red!20!white,title=Versión Ecuaciones Implícitas]
Para cada una de las columnas $B$ de la matriz $A$, determina qué sistemas de ecuaciones tienen solución, siendo $$ \left[\begin{array}{rrr}
0 & 5 & 3 \\
6 & 3 & 6 \\
3 & 0 & 0 \\
2 & 5 & 3 \\
4 & 5 & 3
\end{array}\right] \left[\begin{array}{r}
x_{0} \\
x_{1} \\
x_{2}
\end{array}\right] = B$$ siendo $A$ la matriz que se da a continuación:
\end{tcolorbox}
\end{minipage}
\[ A = \left[\begin{array}{rrrrrrrrrr}
2 & 2 & 4 & 2 & 0 & 3 & 1 & 6 & 1 & 0 \\
1 & 2 & 6 & 5 & 1 & 2 & 1 & 5 & 4 & 0 \\
5 & 6 & 6 & 0 & 4 & 3 & 5 & 0 & 1 & 0 \\
6 & 0 & 1 & 5 & 5 & 2 & 5 & 6 & 4 & 0 \\
1 & 1 & 5 & 5 & 3 & 3 & 4 & 6 & 0 & 0
\end{array}\right] \in {\bf M}_{5\times 10}({\mathbb Z}_{7})\]
\end{ejer}

{\it Soluci\'on:}
% Escribe tu soluci\'on para el Ejercicio 76

\begin{sageblock}
matrix(Zmod(7),[[0,5,3],
[6,3,6],
[3,0,0],
[2,5,3],
[4,5,3]])
matrix(Zmod(7),[[2,2,4,2,0,3,1,6,1,0],
[1,2,6,5,1,2,1,5,4,0],
[5,6,6,0,4,3,5,0,1,0],
[6,0,1,5,5,2,5,6,4,0],
[1,1,5,5,3,3,4,6,0,0]])
\end{sageblock}

% Fin del Ejercicio 76


\begin{ejer} Sea $K$ el cuerpo de 41 elementos.
\newline
\noindent\begin{minipage}{\textwidth}
\begin{tcolorbox}[colback = green!20!white,title=Versión Núcleo]
Determina de entre los vectores columna de la matriz $A$, cuales de ellos están en la imagen de la aplicación lineal $f:K^{3} \to K^{4}$ siendo  $$ M(f) = \left[\begin{array}{rrr}
0 & 13 & 33 \\
12 & 12 & 6 \\
8 & 40 & 19 \\
4 & 30 & 27
\end{array}\right] $$ y $A$ la matriz que se da a continuación:\end{tcolorbox}
\end{minipage} \newline
\noindent\begin{minipage}{\textwidth}
\begin{tcolorbox}[colback = blue!20!white,title=Versión Anulador]
Determina de entre los vectores columna de la matriz $A$, cuales de ellos están en el espacio generado por las columnas de la matriz $$ \left[\begin{array}{rrr}
0 & 13 & 33 \\
12 & 12 & 6 \\
8 & 40 & 19 \\
4 & 30 & 27
\end{array}\right] $$ siendo $A$ la matriz que se da a continuación:\end{tcolorbox}
\end{minipage} \newline
\noindent\begin{minipage}{\textwidth} 
\begin{tcolorbox}[colback = red!20!white,title=Versión Ecuaciones Implícitas]
Para cada una de las columnas $B$ de la matriz $A$, determina qué sistemas de ecuaciones tienen solución, siendo $$ \left[\begin{array}{rrr}
0 & 13 & 33 \\
12 & 12 & 6 \\
8 & 40 & 19 \\
4 & 30 & 27
\end{array}\right] \left[\begin{array}{r}
x_{0} \\
x_{1} \\
x_{2}
\end{array}\right] = B$$ siendo $A$ la matriz que se da a continuación:
\end{tcolorbox}
\end{minipage}
\[ A = \left[\begin{array}{rrrrrrrrrr}
9 & 34 & 13 & 36 & 4 & 24 & 12 & 12 & 25 & 5 \\
17 & 14 & 38 & 14 & 18 & 22 & 13 & 36 & 25 & 14 \\
3 & 37 & 30 & 38 & 15 & 38 & 5 & 22 & 33 & 21 \\
10 & 27 & 25 & 35 & 37 & 28 & 36 & 36 & 31 & 20
\end{array}\right] \in {\bf M}_{4\times 10}({\mathbb Z}_{41})\]
\end{ejer}

{\it Soluci\'on:}
% Escribe tu soluci\'on para el Ejercicio 77

\begin{sageblock}
matrix(Zmod(41),[[0,13,33],
[12,12,6],
[8,40,19],
[4,30,27]])
matrix(Zmod(41),[[9,34,13,36,4,24,12,12,25,5],
[17,14,38,14,18,22,13,36,25,14],
[3,37,30,38,15,38,5,22,33,21],
[10,27,25,35,37,28,36,36,31,20]])
\end{sageblock}

% Fin del Ejercicio 77


\begin{ejer} Sea $K$ el cuerpo de 31 elementos.
\newline
\noindent\begin{minipage}{\textwidth}
\begin{tcolorbox}[colback = green!20!white,title=Versión Núcleo]
Determina de entre los vectores columna de la matriz $A$, cuales de ellos están en la imagen de la aplicación lineal $f:K^{5} \to K^{3}$ siendo  $$ M(f) = \left[\begin{array}{rrrrr}
17 & 17 & 28 & 30 & 19 \\
6 & 6 & 30 & 24 & 29 \\
15 & 15 & 9 & 16 & 29
\end{array}\right] $$ y $A$ la matriz que se da a continuación:\end{tcolorbox}
\end{minipage} \newline
\noindent\begin{minipage}{\textwidth}
\begin{tcolorbox}[colback = blue!20!white,title=Versión Anulador]
Determina de entre los vectores columna de la matriz $A$, cuales de ellos están en el espacio generado por las columnas de la matriz $$ \left[\begin{array}{rrrrr}
17 & 17 & 28 & 30 & 19 \\
6 & 6 & 30 & 24 & 29 \\
15 & 15 & 9 & 16 & 29
\end{array}\right] $$ siendo $A$ la matriz que se da a continuación:\end{tcolorbox}
\end{minipage} \newline
\noindent\begin{minipage}{\textwidth} 
\begin{tcolorbox}[colback = red!20!white,title=Versión Ecuaciones Implícitas]
Para cada una de las columnas $B$ de la matriz $A$, determina qué sistemas de ecuaciones tienen solución, siendo $$ \left[\begin{array}{rrrrr}
17 & 17 & 28 & 30 & 19 \\
6 & 6 & 30 & 24 & 29 \\
15 & 15 & 9 & 16 & 29
\end{array}\right] \left[\begin{array}{r}
x_{0} \\
x_{1} \\
x_{2} \\
x_{3} \\
x_{4}
\end{array}\right] = B$$ siendo $A$ la matriz que se da a continuación:
\end{tcolorbox}
\end{minipage}
\[ A = \left[\begin{array}{rrrrrrrrrr}
16 & 29 & 18 & 2 & 19 & 19 & 10 & 6 & 11 & 10 \\
19 & 3 & 24 & 12 & 8 & 10 & 17 & 7 & 24 & 16 \\
23 & 19 & 7 & 22 & 2 & 18 & 10 & 26 & 24 & 4
\end{array}\right] \in {\bf M}_{3\times 10}({\mathbb Z}_{31})\]
\end{ejer}

{\it Soluci\'on:}
% Escribe tu soluci\'on para el Ejercicio 78

\begin{sageblock}
matrix(Zmod(31),[[17,17,28,30,19],
[6,6,30,24,29],
[15,15,9,16,29]])
matrix(Zmod(31),[[16,29,18,2,19,19,10,6,11,10],
[19,3,24,12,8,10,17,7,24,16],
[23,19,7,22,2,18,10,26,24,4]])
\end{sageblock}

% Fin del Ejercicio 78


\begin{ejer} Sea $K$ el cuerpo de 5 elementos.
\newline
\noindent\begin{minipage}{\textwidth}
\begin{tcolorbox}[colback = green!20!white,title=Versión Núcleo]
Determina de entre los vectores columna de la matriz $A$, cuales de ellos están en la imagen de la aplicación lineal $f:K^{2} \to K^{5}$ siendo  $$ M(f) = \left[\begin{array}{rr}
2 & 0 \\
4 & 0 \\
1 & 0 \\
2 & 0 \\
3 & 0
\end{array}\right] $$ y $A$ la matriz que se da a continuación:\end{tcolorbox}
\end{minipage} \newline
\noindent\begin{minipage}{\textwidth}
\begin{tcolorbox}[colback = blue!20!white,title=Versión Anulador]
Determina de entre los vectores columna de la matriz $A$, cuales de ellos están en el espacio generado por las columnas de la matriz $$ \left[\begin{array}{rr}
2 & 0 \\
4 & 0 \\
1 & 0 \\
2 & 0 \\
3 & 0
\end{array}\right] $$ siendo $A$ la matriz que se da a continuación:\end{tcolorbox}
\end{minipage} \newline
\noindent\begin{minipage}{\textwidth} 
\begin{tcolorbox}[colback = red!20!white,title=Versión Ecuaciones Implícitas]
Para cada una de las columnas $B$ de la matriz $A$, determina qué sistemas de ecuaciones tienen solución, siendo $$ \left[\begin{array}{rr}
2 & 0 \\
4 & 0 \\
1 & 0 \\
2 & 0 \\
3 & 0
\end{array}\right] \left[\begin{array}{r}
x_{0} \\
x_{1}
\end{array}\right] = B$$ siendo $A$ la matriz que se da a continuación:
\end{tcolorbox}
\end{minipage}
\[ A = \left[\begin{array}{rrrrrrrr}
1 & 4 & 2 & 4 & 0 & 4 & 1 & 2 \\
2 & 1 & 0 & 1 & 0 & 3 & 3 & 1 \\
3 & 2 & 3 & 1 & 0 & 2 & 4 & 0 \\
1 & 0 & 4 & 1 & 0 & 4 & 0 & 2 \\
4 & 4 & 4 & 3 & 0 & 1 & 2 & 0
\end{array}\right] \in {\bf M}_{5\times 8}({\mathbb Z}_{5})\]
\end{ejer}

{\it Soluci\'on:}
% Escribe tu soluci\'on para el Ejercicio 79

\begin{sageblock}
matrix(Zmod(5),[[2,0],
[4,0],
[1,0],
[2,0],
[3,0]])
matrix(Zmod(5),[[1,4,2,4,0,4,1,2],
[2,1,0,1,0,3,3,1],
[3,2,3,1,0,2,4,0],
[1,0,4,1,0,4,0,2],
[4,4,4,3,0,1,2,0]])
\end{sageblock}

% Fin del Ejercicio 79


\begin{ejer} Sea $K$ el cuerpo de 19 elementos.
\newline
\noindent\begin{minipage}{\textwidth}
\begin{tcolorbox}[colback = green!20!white,title=Versión Núcleo]
Determina de entre los vectores columna de la matriz $A$, cuales de ellos están en la imagen de la aplicación lineal $f:K^{5} \to K^{3}$ siendo  $$ M(f) = \left[\begin{array}{rrrrr}
1 & 15 & 5 & 14 & 1 \\
0 & 1 & 8 & 1 & 9 \\
17 & 12 & 3 & 14 & 15
\end{array}\right] $$ y $A$ la matriz que se da a continuación:\end{tcolorbox}
\end{minipage} \newline
\noindent\begin{minipage}{\textwidth}
\begin{tcolorbox}[colback = blue!20!white,title=Versión Anulador]
Determina de entre los vectores columna de la matriz $A$, cuales de ellos están en el espacio generado por las columnas de la matriz $$ \left[\begin{array}{rrrrr}
1 & 15 & 5 & 14 & 1 \\
0 & 1 & 8 & 1 & 9 \\
17 & 12 & 3 & 14 & 15
\end{array}\right] $$ siendo $A$ la matriz que se da a continuación:\end{tcolorbox}
\end{minipage} \newline
\noindent\begin{minipage}{\textwidth} 
\begin{tcolorbox}[colback = red!20!white,title=Versión Ecuaciones Implícitas]
Para cada una de las columnas $B$ de la matriz $A$, determina qué sistemas de ecuaciones tienen solución, siendo $$ \left[\begin{array}{rrrrr}
1 & 15 & 5 & 14 & 1 \\
0 & 1 & 8 & 1 & 9 \\
17 & 12 & 3 & 14 & 15
\end{array}\right] \left[\begin{array}{r}
x_{0} \\
x_{1} \\
x_{2} \\
x_{3} \\
x_{4}
\end{array}\right] = B$$ siendo $A$ la matriz que se da a continuación:
\end{tcolorbox}
\end{minipage}
\[ A = \left[\begin{array}{rrrrrrrrrr}
0 & 10 & 3 & 5 & 6 & 1 & 5 & 16 & 7 & 2 \\
9 & 15 & 6 & 0 & 9 & 5 & 11 & 16 & 0 & 17 \\
5 & 2 & 11 & 9 & 6 & 18 & 6 & 9 & 5 & 7
\end{array}\right] \in {\bf M}_{3\times 10}({\mathbb Z}_{19})\]
\end{ejer}

{\it Soluci\'on:}
% Escribe tu soluci\'on para el Ejercicio 80

\begin{sageblock}
matrix(Zmod(19),[[1,15,5,14,1],
[0,1,8,1,9],
[17,12,3,14,15]])
matrix(Zmod(19),[[0,10,3,5,6,1,5,16,7,2],
[9,15,6,0,9,5,11,16,0,17],
[5,2,11,9,6,18,6,9,5,7]])
\end{sageblock}

% Fin del Ejercicio 80

\end{document}
