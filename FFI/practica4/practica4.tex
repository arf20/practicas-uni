\documentclass[12pt, letterpaper, twoside]{article}

\usepackage{amsmath}
\usepackage{siunitx}

\title{Tarea Practica 4}
\author{Ángel Ruiz Fernández 1º G4.1 FFI}
\date{10 Marzo 2024}

\begin{document}
	\maketitle
	
	\section{}
	En la descarga de un condensador, la diferencia de potencial entre sus	terminales viene dada por la ecuación $V = V_0 e^{-0.0613t}$. Si la resistencia del circuito es $R = 15 \si K \Omega$. Calcule: La capacidad del condensador C en $\si{\mu F}$
	
	\begin{align*}
		-\frac{t}{RC} &= -\frac{t}{15000 C} = -0.0613t \\
		\frac{1}{15000 C} &= 0.0613 \\
		15000 C &= \frac{1}{0.0613} \\
		C &= \frac{1}{15000 \cdot 0.0613} = 0.0010875 \si{F} = \boxed{ 1087.5 \si{\mu F}} \\
	\end{align*}
	
\end{document}