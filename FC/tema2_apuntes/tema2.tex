\documentclass[12pt, letterpaper, twoside]{article}

\usepackage{amsmath}

\title{FC Tema 2: Apuntes}
\author{Ángel Ruiz Fernández 1º G4.1}
\date{Enero 2024}

\usepackage[a4paper, total={6in, 9in}]{geometry}

\newcommand{\dd}[1]{\mathrm{d}#1}

%\usepackage[showframe,paper=a4paper,margin=1in]{geometry}
%\setlength{\droptitle}{-0.5in}


\begin{document}
	\maketitle
	
	\section{Representación de enteros}
	Representaciones comunes
	\begin{itemize}
		\item Base 2: Binario {0, 1}
		\item Base 8: Octal {0-7}
		\item Base 10: Decimal {0-9}
		\item Base 16: Hexadecimal {0-9A-F}
	\end{itemize}

	\subsection{Binario}
	\begin{itemize}
		\item $000)_2$ = $0)_{10}$
		\item $001)_2$ = $1)_{10}$
		\item $010)_2$ = $2)_{10}$
		\item $011)_2$ = $3)_{10}$
		\item $100)_2$ = $4)_{10}$
		\item $101)_2$ = $5)_{10}$
		\item $110)_2$ = $6)_{10}$
		\item $111)_2$ = $7)_{10}$
	\end{itemize}
	
	El numero de numeros posibles en $n$ bits es $2^n$
	
	
	
\end{document}